\documentclass[11pt,letterpaper]{article}
\usepackage[utf8]{inputenc}
\usepackage[spanish]{babel}
\usepackage{amsmath}
\usepackage{amsfonts}
\usepackage{amssymb}
\usepackage{xcolor}
\usepackage{makeidx}
\usepackage{graphicx}

\usepackage{physics}
\usepackage{dsfont}

\renewcommand{\theenumi}{\alph{enumi}}

\usepackage[draft,inline,nomargin]{fixme} \fxsetup{theme=color}
\definecolor{jacolor}{RGB}{200,40,0} 
\FXRegisterAuthor{ja}{aja}{\color{jacolor}JA}

\usepackage[left=2cm,right=2cm,top=2cm,bottom=2cm]{geometry}
\author{José Alfredo de León}
\title{Hoja de trabajo 2}
\begin{document}
\maketitle

\textbf{Instrucciones:} Resuelva de forma clara y ordenada (a papel y lapiz o 
utilizando \LaTeX) los ejercicios y suba su solución a más tardar para la fecha
XXX a la plataforma de Google Classroom. La calificación de esta hoja de trabajo
será por cantidad de ejercicios \textbf{resueltos completamente}.

\textbf{Se recomienda fuertemente utilizar algún software, como Python
o Mathematica, para resolver operaciones entre matrices, calcular eigenvalores 
y eigenvectores o cualquier otro procedimiento algebraico repetitivo y largo.}

\subsubsection*{Ejercicio 1. Estados estacionarios y evolución temporal.}
Se denota como $\psi(n)$ a los estados estacionarios 
ortonormales de un sistema cuya energ\'ia correspondiente es $E_n$. 
Supongamos que la funci\'on de onda normalizada
del sistema en el tiempo $t=0$ es $\Psi(x,0)$ y supongamos que una medici\'on
de la energ\'ia da como resultado el valor $E_1$ con probabilidad $1/2$, $E_2$
con probabilidad $3/8$ y $E_3$ con probabilidad $1/8$. 
\begin{enumerate}
\item ¿Qu\'e significa que un estado sea estacionario?
\item ¿Escriba la expansi\'on m\'as general para $\Psi(x,0)$ consistente con 
esta informaci\'on.
\item ¿Cu\'al es la expansi\'on para la funci\'on de onda del sistema para un tiempo 
$t>0$, $\Psi(x,t)$?
\item ¿Muestre que el valor esperado del Hamiltoniano no cambia en el tiempo.
\end{enumerate}

\subsubsection*{Ejercicio 2. Medición de la energía.}
Un sistema compuesto por dos átomos de dos niveles (estado fundamental + 
estado excitado) es un sistema cuántico de 4 niveles, por consiguiente, su 
espacio de estados es $\mathcal{H}=\mathbb{C}^4$. Supongamos un estado
inicial, en $t=0$, de la forma
\begin{align}
\ket{\psi(0)}&=\frac{1}{\sqrt{3}}\ket{\phi_1}+\frac{1}{2}\ket{\phi_2}
+\frac{1}{\sqrt{6}}\ket{\phi_3}+\frac{1}{2}\ket{\phi_4},
\end{align}
donde $\{ \ket{\phi_i}\}$ son los eigenestados ortonormales del Hamiltoniano
$\hat H$ del sistema. 
\begin{enumerate}
\item Si los eigenestados $\ket{\phi_i}$ del Hamiltoniano tienen
energías $E_i$, respectivamente, escriba el estado
$\ket{\psi(t)}$, para $t>0$.
\item ¿Cuáles son los posibles resultados de medir la energía de este sistema
en el estado $\ket{\psi(t)}$ y con qué probabilidad ocurrirán?
\item Encuentre el valor esperado del Hamiltoniano del sistema en $t=0$
y $t=10$ s. ¿De qué forma tendría que ser el estado $\ket{\psi(0)}$ para que 
el valor esperado del Hamiltoniano $\hat H$ sea constante en el tiempo?
\end{enumerate}

\subsubsection*{Ejercicio 3. Conjunto completo de operadores 
que conmutan I.}
El fotón (partícula de espín 1), cuando sólo consideramos sus grados de 
libertad de espín,
es un sistema cuántico de 3 niveles, \textit{i.e.} $\mathcal{H}=\mathbb{C}^3$.
\begin{enumerate}
\item ¿Cuál es la dimensión de las matrices que representan a los observables que 
actúan sobre el espacio de estados del fotón? Convánzase de que la dimensión 
debe ser $3\times 3$.
\end{enumerate}
Considere que el Hamiltoniano $\hat H$ y un operador $\hat A$ que actúan 
sobre el espacio 
de estados del fotón están dados por 
\begin{align}
\hat H&=\epsilon_0\mqty(-2&0&0\\ 0&1&0\\ 0&0&1),
&
\hat A&=a_0\mqty(5&0&0\\ 0&0&2\\ 0&2&0),
\end{align}
donde $\epsilon_0$ tiene dimensionales de energ\'ia.
\begin{enumerate}
\item ¿Los operadores $\hat H$ y $\hat A$ conmutan? Si es as\'i
encuentre la base de eigenvectores en com\'un.
\item ¿Cu\'al de los siguientes conjuntos de operadores forman un conjunto 
completo de operadores que conmutan: $\{ \hat H\}$, $\{ \hat A\}$, 
$\{ \hat H, \hat A\}$, $\{ \hat H^2, \hat A \}$?
\end{enumerate}

\subsubsection*{Ejercicio 4. Medición de la energía. CSCO y 
mediciones consecutivas.}
Consideremos de nuevo a un sistema cuántico compuesto por un fotón
cuyo Hamiltoniano $\hat H$ y estado inicial $\ket{\psi(0)}$ son
\begin{align}
\hat H&=\epsilon_0\mqty(1&-1&0\\-1&1&0\\0&0&-1),
&
\ket{\psi(0)}=\frac{1}{\sqrt{6}}\mqty(1\\1\\2),
\end{align}
donde $\epsilon_0$ tiene dimensionales de energía. 
\begin{enumerate}
\item ¿Cuáles valores se obtendrían al medir la energía del estado 
$\ket{\psi(0)}$ y con qué probabilidades?
\item Calcule el valor esperado del Hamiltoniano $\hat H$ en el estado 
$\ket{\psi(0)}$.
\end{enumerate}

\subsubsection*{Ejercicio 5}
De nuevo, consideremos a nuestro buen fotón como sistema cuántico,
cuyo estado $\ket{\psi}$ y dos observables $A$ y $B$ están dados por 
\begin{align}
\ket{\psi}&=\frac{1}{3}\mqty(-i\\2\\0),&
A&=\frac{1}{\sqrt{2}}\mqty(1&i&1\\-i&0&0\\1&0&0),&
B&=\mqty(3&0&0\\0&1&i\\0&-i&1).
\end{align}
\begin{enumerate}
\item ¿Los operadores $A$ y $B$ son compatibles? ¿Cuáles de los conjuntos 
$\{ \hat A\}$, $\{ \hat B\}$,$\{ \hat A,\hat B\}$ forman un conjunto completo 
de operadores que conmutan?
\item Se mide $\hat A$ primero e inmediatamente después $\hat B$. Encuentre
la probabilidad de obtener el valor $-1$ para $\hat A$ y el valor $3$ para $B$.
\item Ahora se mide en el orden inverso que en el inciso anterior. Encuentre
la probabilidad de obtener el valor $3$ para $B$ y el valor $-1$ para $\hat A$.
\item Compare y explique los resultados de los incisos b y c.
\end{enumerate}

\subsubsection*{Ejercicio 6}
De nuevo, consideremos a nuestro buen fotón como sistema cuántico,
cuyo estado $\ket{\psi}$ y dos observables $A$ y $B$ están dados por 
\begin{align}
\ket{\psi}&=\mqty(5\\1\\3),&
A&=\frac{1}{\sqrt{2}}\mqty(2&0&0\\0&1&1\\0&1&1),&
B&=\mqty(1&0&0\\0&0&1\\0&1&0).
\end{align}
\begin{enumerate}
\item ¿Los operadores $A$ y $B$ son compatibles? 
\item ¿Cuáles de los conjuntos $\{ \hat A\}$, $\{ \hat B\}$,$\{ \hat A,\hat B\}$ 
forman un conjunto completo de operadores que conmutan?
\end{enumerate}
\textbf{Advertencia:} la respuesta de los inciso a. y b. son suficientes
para contestar a los insisos siguientes sin tener que hacer un sólo 
cálculo. Tómese unos minutos para meditar la respuesta y discútalo con
sus compañeros. Si decide hacer los cálculos y no los hará con ayuda del algún 
software, ni siquiera intente hacerlos a mano, sólo perderá su tiempo y se frustrará.
\begin{enumerate}
\item[c.] Se realiza una medición donde $A$ se mide primero y luego $B$ 
inmediatamente 
después. Encuentre la probabilidad de obtener el valor de $\sqrt{2}$ para $A$
y el valor de $-1$ para $B$.
\item[d.] Ahora se mide $B$ primero e inmediatamente después $A$. Encuentre 
la probabilidad de obtener el valor $-1$ para $B$ y el valor $\sqrt{2}$ para $A$.
\item[e.] Compare los resultados de los incisos a y b. Explique.
\end{enumerate}

\subsubsection*{Ejercicio 7}
Considere un sistema cuántico de 3 niveles que tiene tres observables
que se representan por las siguientes matrices:
\begin{align}
A&=\mqty(1&0&0\\0&0&1\\0&1&0),&
B&=\mqty(0&0&-1\\0&0&i\\-1&-i&4),&
C&=\mqty(2&0&0\\0&1&3\\0&3&1).
\end{align}
\begin{enumerate}
\item Encuentre los posibles resultados de medir cada uno de los observables 
$A$, $B$ y $C$.
\item De los 3 observables, ¿cuáles de ellos son compatibles?
Encuentre la base de eigenvectores en común a ese/esos conjunto/os 
xde observables. 
\item ¿Cuáles de los conjuntos de operadores $\{ \hat A\}$, $\{ \hat B\}$, 
$\{ \hat C\}$
$\{ \hat A, \hat B\}$, $\{ \hat B, \hat C\}$, $\{ \hat A, \hat C\}$ y
$\{ \hat A, \hat B, \hat C\}$ forman un conjunto completo de observables que 
conmutan? 
\end{enumerate}


\end{document}