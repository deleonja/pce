\documentclass[11pt,letterpaper]{article}
\usepackage[utf8]{inputenc}
\usepackage[spanish]{babel}
\usepackage{amsmath}
\usepackage{amsfonts}
\usepackage{amssymb}
\usepackage{xcolor}
\usepackage{makeidx}
\usepackage{graphicx}

\usepackage{physics}
\usepackage{dsfont}

\renewcommand{\theenumi}{\alph{enumi}}

\usepackage[draft,inline,nomargin]{fixme} \fxsetup{theme=color}
\definecolor{jacolor}{RGB}{200,40,0} \FXRegisterAuthor{ja}{aja}{\color{jacolor}JA}

\usepackage[left=2cm,right=2cm,top=2cm,bottom=2cm]{geometry}
\author{José Alfredo de León}
\title{Hoja de trabajo 1}
\begin{document}
\maketitle

\textbf{Instrucciones:} Resuelva de forma clara y ordenada (a papel y lapiz o 
utilizando \LaTeX) los ejercicios y suba su solución a más tardar para la fecha
XXX a la plataforma de Google Classroom. La calificación de esta hoja de trabajo
será por cantidad de ejercicios \textbf{resueltos completamente}.

\subsubsection*{Ejercicio 1. Notación de Dirac.}
Si consideramos únicamente los grados de libertad de espín, el fotón es un 
sistema cuántico de 3 niveles. Es decir, el espacio de Hilbert de un fotón es 
$\mathcal{H}=\mathbb{C}^3$. Consideremos ahora una base ortonormal 
de este espacio $\qty{\ket{\phi_i}}$ y también dos estados del 
fotón $\ket{\psi_1}=i\ket{\phi_1}+3i\ket{\phi_2}-\ket{\phi_3}$
y $\ket{\psi_2}=\ket{\phi_1}-i\ket{\phi_2}+5i\ket{\phi_3}$, donde $\qty{\phi_i}$.
\begin{enumerate}
\item El producto interior, llamado ``bracket'' en la notación de Dirac, 
cuántifica la proyección de un estado (vector) sobre otro estado
(vector). Calcule los brackets
$\braket{\phi_i}{\phi_j}$, cuando $i=j$ y cuando $i\ne j$.
\item Calcule $\braket{\psi_i}{\psi_j}$ para $i,j=1,2$ y, a partir de estos
resultados, calcule 
$\Big(\bra{\psi_1}+\bra{\psi_2}\Big)\Big(\ket{(\psi_1}+\ket{\psi_2}\Big)$.
\item En general, para cualesquiera dos estados $\ket{\psi_1}$ y
$\ket{\psi_2}$, ¿los brackets $\braket{\psi_1}{\psi_2}$ y
$\braket{\psi_2}{\psi_1}$ son iguales? ¿Por qué?
\end{enumerate}
\textit{Checkpoint: (medí cuánta física has aprendido)} Cualquier estado $\ket{\psi}$
del fotón puede escribirse como
\begin{align}
\ket{\psi}=\sum_{i=1}^3c_i\ket{\varphi_i}
\end{align}
Matemáticamente (y en notación de Dirac), ¿qué es lo que debés calcular si
te preguntan por la proyección del estado $\ket{\psi}$ sobre cada uno de los 
elementos de la base $\qty{\ket{\varphi_i}}$?
\begin{enumerate}
\item[d.]
Los operadores que actúan sobre $\mathcal{H}$ también se pueden 
escribir en la notación de Dirac como productos externos de ket's con bra's.
Calcule $\dyad{\psi_1}{\psi_2}$ y $\dyad{\psi_2}{\psi_1}$.
\item[e.] Dada una base ortonormal $\qty{\ket{\phi_i}}$
y un operador $\Lambda$, la traza de $\Lambda$ se calcula como
\begin{align}
\Tr (\Lambda)&=\sum_i \matrixel{\phi_i}{\Lambda}{\phi_i}
\end{align}
Calcule la traza de los operadores en (d).
\item 
Repita los cálculos de los incisos (b), (c), (d) y (e) usando la representación 
matricial. \textit{Ayuda para repetir (d):} calcule $\dyad{\phi_1}{\phi_1}$, 
$\dyad{\phi_2}{\phi_1}$ y $\dyad{\phi_3}{\phi_2}$ y note la relación entre 
los índices $i$ y $j$ del producto $\dyad{\phi_i}{\phi_j}$ y las posiciones del
elemento no nulo en la matriz de $3\times 3$.
\item[f.] Escriba los conjugados Hermíticos de $\ket{\psi_1}$, $\ket{\psi_2}$,
$\dyad{\psi_1}{\psi_2}$ y $\dyad{\psi_2}{\psi_1}$ en notación de Dirac
y en su representación matricial asumiendo que $\qty{\ket{\phi_i}}$ es la base
canónica.
\end{enumerate}

\subsection*{Ejercicio 2. De la notación de Dirac a la representación
matricial.}
Repita los incisos d, e y f del ejercicio 1 calculando la representación matricial 
de $\dyad{\psi_1}{\psi_2}$ y $\dyad{\psi_2}{\psi_1}$.\newline

\textit{Ayuda:} Los indices $i,j$ del operador $\lambda \dyad{\phi_i}{\phi_j}$ se 
corresponden con la posición del escalar $\lambda$ en la matriz de $3\times 3$.
Por ejemplo, 
\begin{align}
3\dyad{\phi_1}{\phi_3}=\mqty(0&0&3\\0&0&0\\0&0&0).
\end{align}

\subsection*{Ejercicio 3. Operadores I.}
Considere un sistema de dos niveles, como una partícula de espín $1/2$ (un 
electrón, por ejemplo). Considere la matriz de Pauli $\hat \sigma_z$ que se define 
como $\hat \sigma_z\ket{0}=\ket{0}$ y $\hat \sigma_z\ket{1}=-\ket{1}$; $\ket{0}$ 
y $\ket{1}$ son ortonormales,
\begin{enumerate}
\item ¿Los estados $\ket{0}$ y $\ket{1}$ forman una base del espacio
de estados? ¿Por qué?
%\item[(b)] Considere el operador $\hat \Lambda=\dyad{0}{1}$. ¿Es $\hat \Lambda$ hermítico?
\item Muestre que $\hat \Lambda^2=0$.
\item Un operador $\hat P$ que es un proyector satisface la condición
$\hat P^2=\hat P$. Muestre que $\hat \Lambda \hat \Lambda^{\dagger}$ y 
$\hat \Lambda^{\dagger} \hat \Lambda$ son proyectores.
\item Los operadores unitarios son importantes en física porque describen 
la evolución de los sistemas cuánticos cerrados. Por otro lado, el conmutador 
de dos operadores en mecánica cuántica nos proporciona información sobre 
la implicación que podría tener alterar el orden de dos mediciones sucesivas.

Un operador $\hat U$ se dice que es unitario si satisface la condición
$\hat U\hat U^{\dagger}=\hat U^{\dagger}\hat U=\mathds{1}$ ($\mathds{1}$ el operador identidad).
El conmutador de dos operadores $\hat \Lambda$ y $\hat B$ se define como
$\comm{\hat \Lambda}{\hat B}=\hat \Lambda\hat B-\hat B\hat \Lambda$. 
Muestre que el operador 
$\hat \Lambda\hat \Lambda^{\dagger}-\hat \Lambda^{\dagger}\hat \Lambda$ es unitario.
\end{enumerate}

\subsection*{Ejercicio 4. Operadores II.}
Considere el estado $\ket{\psi}$ del fotón, que está dado en términos de la 
base ortonormal $\qty{\ket{\phi_k}}$, de la forma
\begin{align}
\ket{\psi}=a\bigg(\frac{i}{\sqrt{15}}\ket{\phi_1}+
\frac{1}{\sqrt{3}}\ket{\phi_2}-
\frac{i}{\sqrt{5}}\ket{\phi_3}\bigg),
\end{align}
donde $a\in \mathbb{R}$ y $\ket{\phi_k}$ son eigenestados del 
operador $\hat B$, tal que satisface la ecuación de eigenvalores siguiente,
\begin{align}
\hat B\ket{\phi_k}=k\ket{\phi_k}.
\end{align}
\begin{enumerate}
\item Encuentre la constante de normalización $a$ del estado $\ket{\psi}$. 
\item Encuentre el valor de expectación de $\hat B$ para el estado $\ket{\psi}$.
\item Encuentre el valor de expectación de $\hat B^2$ para el estado 
$\ket{\psi}$. 
\end{enumerate}

\subsection*{Ejercicio 5. Operadores III.}
Considere el operador
\begin{align}
\hat A=\dyad{\phi_1}{\phi_1}+\dyad{\phi_2}{\phi_2}+\dyad{\phi_3}{\phi_3}
-i\dyad{\phi_1}{\phi_2}-\dyad{\phi_1}{\phi_3}+i\dyad{\phi_2}{\phi_1}
-\dyad{\phi_3}{\phi_1},
\end{align}
donde $\qty{\ket{\phi_i}}$ forma una base ortonormal y completa.
\begin{enumerate}
\item ¿Es $\hat A$ hermítico?
\item ¿Es $\hat A$ un proyector?
\item Encuentre la representación matricial de $\hat A$ en la base $\qty{\ket{\phi}}$.
\item Encuentre los eigenvalores y eigenvectores de la matriz. 
\end{enumerate}

\subsection*{Ejercicio 6. Representación en bases discretas.}
\janote{Hacer este ejercicio para $\Tr (AB)=\Tr (BA)$ en el taller.}
Supongamos tres operadores $\hat \Lambda$, $\hat B$ y $\hat C$ que actúan sobre 
el espacio de Hilbert de un sistema cuántico de dimensión finita. Utilizando
notación de Dirac, muestre que
\begin{align}
\Tr \big(\hat \Lambda\hat B\hat C \big)=\Tr \big(\hat C\hat \Lambda\hat B \big)
=\Tr \big(\hat B\hat C\hat \Lambda \big).
\end{align}
Esta se conoce como la ``propiedad cíclica'' de la traza y es generalizable
para cualquier cantidad de operadores. La traza de la multiplicación de $n$
operadores es invariante ante permutaciones de cíclicas del producto de los
operadores.

\subsection*{Ejercicio 7. Representación en bases continuas.}
\janote{Dar pistas en el taller. Quizás hacer el caso inverso?}
Utilizando la notación de Dirac muestre que la función de onda $\psi(x)$ es 
la transformada de Fourier de la función de onda $\Psi (p)$. Utilice el hecho 
de que el cambio de base del espacio de los momenta al de posiciones está 
dado por
\begin{align}
\braket{x}{p}=\frac{1}{\qty(2\pi\hbar)^{3/2}}e^{ipx/\hbar}.
\end{align}

\subsection*{Ejercicio 8. Mecánica matricial.}
El Hamiltoniano de un sistema de dos niveles está dado por 
\begin{align}
\hat H=E\qty(\dyad{\phi_1}{\phi_1}-\dyad{\phi_2}{\phi_2}-i\dyad{\phi_1}{\phi_2}
+i\dyad{\phi_2}{\phi_1}),
\end{align}
donde $\ket{\phi_1}$ y $\ket{\phi_2}$ forman una base ortonormal y completa, y
$E$ es una constante real que tiene dimensionales de energía.
\begin{enumerate}
\item ¿Es $\hat H$ Hermítico? Utilizando la notación de Dirac, 
calcule la traza de $\hat H$.
\item Encuentra la representación matricial de $\hat H$ en la base 
$\ket{\phi_1}$ y $\ket{\phi_2}$ y calcule los eigenvalores y eigenvectores
de la matriz. Ahora que tiene la matriz que representa a $\hat H$ calcule de nuevo
la traza (sumando los elementos de la diagonal) y comparelo con el resultado
del inciso (a).
\item Calcule $\comm{\hat H}{\dyad{\phi_1}{\phi_1}}$, $\comm{\hat H}
{\dyad{\phi_2}{\phi_2}}$ y $\comm{\hat H}{\dyad{\phi_1}{\phi_2}}$.
\end{enumerate}

\subsection*{Respuestas}
\begin{enumerate}
\item[2.a.] $a=\sqrt{5/3}$
\item[2.b.] $\matrixel{\psi}{\hat B}{\psi}=20/9$
\item[2.c.] $\matrixel{\psi}{\hat B^2}{\psi}=16/3$
\end{enumerate}
\end{document}