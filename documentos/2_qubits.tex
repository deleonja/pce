\section{2-qubits PCE quantum channels}\label{sec:2_qubits}
\cpnote{Cambiar el nombre al apendice cuando lo tengamos mas pulido}

% This action can be encoded using a vector index $\vec\alpha$, 
% as in \eqref{eq:pauli_strings}, hence each of the different $4^N$ 
% vector indices may be uniquely related to each of the PCE generators
% and thus denoted as $\mcG_{\vec\alpha}$.
% The proof is simplified if one uses the Kraus representation developed in 
% \sref{sec:kraus}, so we postpone the demonstration to appendix \ref{sec:2_qubits}.

% Notar que el hecho de que cada generador tiene un indice alpha. 
% Decir que ese alpha es la accion del operador 


% 1. La acción sobre el estado está determinada por la alpha. Eso ya se sabe. 

% Para generadores la accion fisica del elemento alpha_k es como actua sobre 
% el qubit k (o quizá el j). 


\begin{figure} % {{{
\centering
\includegraphics[width=\columnwidth]{twoQ_generator_diagrams-crop}
\caption{PCE generators of 2 qubits. \cpnote{Mejorar el caption}}
\label{fig:appex:twoQ_PCEGs}
\end{figure} % }}}



The action of a generator $\mcG_\valpha$ depends only, as its notations suggests  on 
the multi index $\valpha$. To interpret the meaning of $\valpha$, we consider the action
of the generator on a reduced density matrix. In fact, 
\janote{me confunde este enunciado. Yo sugiero el siguiente cambio}
\janote{To interpret the meaning of $\valpha$, we consider the reduced 
density matrix of the $k$-th qubit after the action of the generator $\mcG_\valpha$
on the whole system,}
\begin{align}
\tr_{\not k} \mcG_\valpha [\rho] &= 
\frac12 \tr_{\not k} (\rho + \sigma_\valpha \rho \sigma_\valpha)\nonumber \\
&= \frac{\rho_k}{2} + \frac{\sigma_{\alpha_k} \rho_k \sigma_{\alpha_k}}{2}\nonumber \\
&= \mcG_{\alpha_k} [\rho_k ],
\label{eq:generator:individual:qubit}
\end{align}
where the partial trace over everything except qubit $k$ is
denoted by $\tr_{\not k}$\ddnote{esta curiosa esta notacion, sugiero algo mas discreto, como $\tr_{\overline{k}}$}. The last line denotes a single qubit PCE generator
corresponding to the index $\alpha_k$, which is the $k$-th component 
of $\valpha$. 
\janote{Ya no entiendo para qué sirve todo lo que va después. 
Si el objetivo era relacionar las $\vec\alpha$ con la acción local de cada generador
para mí hasta aquí es suficiente.}

We now connect particular elements of a PCE map with the action 
over particular qubits. To do so, we define basis vectors $\vec e_k$ via its components as
$[\vec e_k]_j = \delta_{jk}$. This means that 
$\valpha = \sum_k \alpha_k \vec e_k$.
If we denote 
\begin{equation}
\mcE_{\{ \tau_\valpha\}}[\rho] = \sum_\valpha \tau_\valpha r_\valpha \sigma_\valpha,
\end{equation}
we can verify using direct calculation that 
\begin{equation}
\tr_{\not k} \mcE_{\{ \tau_\valpha\}}[\rho]  = 
\mcE_{\{ \tau_{\alpha_k} \}}[\rho_k].
\label{eq:trnottrnot}
\end{equation}
% = \sum_\valpha \tau_\valpha r_\valpha \sigma_\valpha.

\janote{El enunciado a continuación no lo entendí, incluso meditándolo un rato.
Tal vez ya estoy cansado, pero preferiría que me expliques.}
\Eref{eq:trnottrnot} means that the action of a channel over the $k$-th qubit is determined
by the four numbers $\tau_{\alpha_k \vec e_k}$ which, for the two qubit case correspond 
to the first column (if $k=1$) or the first row (if $k=2$). For larger $N$ the generalization 
is immediate. Connecting \eref{eq:trnottrnot} 
and \eref{eq:generator:individual:qubit} allows us to interpret  easily 
the multi-index in each generator. In \fref{fig:appex:twoQ_PCEGs} we present 
all two qubit generators and their corresponding $\valpha$. 



\cpnote{Quiza una figura con la accion de la identidad, la X Y y Z?}

\janote{Mi contrapropuesta}

The local-action of a generator $\mcG_\valpha$ on every qubit in the 
system depends only, as its notations suggests, on the multi index $\valpha$. 
For this matter, it is necessary to start discussing the PCE
single qubit generators $\mcG_\alpha$. They are shown in 
\fref{fig:A_and_PCEGs_connection}, $\mcG_0$ corresponds to the identity map, 
and the remaining $\mcG_{1,2,3}$  are the completely bit, phase and bit-phase
flip channels;
\janote{
recall that in a PCE diagram every square's color indicates the value 
attained by the corresponding $\tau_{\alpha}$.
That is to say, without regarding $\tau_0$, $\alpha=j$ when only one $\tau_{j}=1$,
and $\alpha=0$ when all $\tau_{j}=1$.
}
Now, \janote{to generalize this characterization rule for a $N$-qubit 
PCE generator $\mcG_\valpha$} we consider the reduced 
density matrix of the $k$-th qubit after generator $\mcG_\valpha$ acts
on the entire system,
\begin{align}
\tr_{\not k} \mcG_\valpha [\rho] &= 
\frac12 \tr_{\not k} (\rho + \sigma_\valpha \rho \sigma_\valpha)\nonumber \\
&= \frac{\rho_k}{2} + \frac{\sigma_{\alpha_k} \rho_k \sigma_{\alpha_k}}{2}\nonumber \\
&= \mcG_{\alpha_k} [\rho_k ],
\label{eq:generator:individual:qubit}
\end{align}
where $\not k$ means all qubits except for the $k$-th one are traced out.
\janote{
Note in the second line of \eqref{eq:generator:individual:qubit} that 
$\alpha_k$ not only characterizes $\mcG_{\alpha_k}$ but actually tells 
us which channel is locally acting on the $k$-th qubit, whether a 
completely bit, phase, bit-phase, or identity channel.
In the last line of \eqref{eq:generator:individual:qubit} 
$\mcG_{\alpha_k}$ is a single qubit PCE generator and the Pauli components
of $k$-th qubit's reduced density matrix $\rho_k$ are of the form
$r_{0,\ldots,j_k,\ldots,0}$, 
meaning that $\mcG_{\alpha_k}$ acts on $\rho_k$ as 
$r_{0,\ldots,j_k,\ldots,0}\mapsto 
\tau_{0,\ldots,j_k,\ldots,0}r_{0,\ldots,j_k,\ldots,0}$.
Thus, the general characterization rule of all PCE generators $\mcG_\valpha$ 
is clear now:
without considering $\tau_{0,\ldots,0}$,
if only one $\tau_{0,\ldots,j_k,\ldots,0}=1$, then 
$\alpha_k=j_k$, or if all $\tau_{0,\ldots,j_k,\ldots,0}=1$, 
then $\alpha_k=0$.
Let us give an example to illustrate the PCE generators characterization. 
For two qubits, the vector index $\valpha$ is of the form $(j_1,j_2)$,
where $j_1$ equals to 0, 1, 2, or 3 depending on the values $\tau_{j_1,0}$
and the aforementioned rule; the same follows for $j_2$.
In PCE diagrams language this means
that the vector index $\valpha$ is encoded in the first column and row of 
a diagram. See $\mcG_{(0,2)}$ in \fref{fig:appex:twoQ_PCEGs},
all $\tau_{j_1,0}=1$ and $\tau_{(0,2)}=1$, then $\valpha=(0,2)$. 
In \fref{fig:appex:twoQ_PCEGs} we show all two qubits PCE generators
and their corresponding notation $\mcG_\valpha$.
}  
%The last line denotes a single qubit PCE generator
%corresponding to the index $\alpha_k$, which is the $k$-th component 
%of $\valpha$. Hence, each component of the vector-index $\vec\alpha$ 
%provides what the local-action of $\mcG_\valpha$ is on each qubit in
%the system. For two qubits, the local-action of PCE quantum channels 
%in the diagrams is seen in the first row (qubit 1) and first column (qubit 2).
%We present in \fref{fig:appex:twoQ_PCEGs}
%all two qubit generators and their corresponding notation $\mcG_\valpha$. 
%\cpnote{Falta justo lo que va en la segunda parte de mi texto. Es decir asociar la 
%accion local sobre cada qubit con la primer fila o oclumna, etc del diagrama}

% Comentarios viejos {{{
% Esto es equivalente al canal k sobre un qubut, y se puede extrapolar de la Figura 1. 

% Así podemos leer la accion y relacionar las figuras con la notacion de las alphas.


%  Then, we can write the partial trace 
% of the generic state $\rho$, see \eref{eq:N_qubits_rho}, as
% $$
% \rho_k = 
% \tr_{\not k} \rho = 
% \sum_\alpha r_{\alpha \vec e_k} \sigma_\alpha.
% $$


% Conectar los elementos tau de la primera fila con la accion sobre el qubit individual. 

% We also introduce basis vectors for multi index 
% This means that 


% We shall now study the reduced action of a PCE map over a particular qubit. To do 
% so, we introduce the partial trace over everything except qubit $k$; this 
% operation  will be denoted by $\tr_{\not k}$. We also introduce basis vectors for multi index 
% $\valpha$. Define $\vec e_k$ via its components as
% $[\vec e_k]_j = \delta_{jk}$. Then, we can write the partial trace 
% of the generic state $\rho$, see \eref{eq:N_qubits_rho}, as
% $$
% \rho_k = 
% \tr_{\not k} \rho = 
% \sum_\alpha r_{\alpha \vec e_k} \sigma_\alpha.
% $$

% Consider now the multi qubit Pauli map defined by coefficients $\{ \tau_\valpha\}$, 
% whose action is 
% $$
% \mcE_{\{ \tau_\valpha\}}[\rho] = \sum_\valpha \tau_\valpha r_\valpha \sigma_\valpha.
% $$
% The action on qubit $k$ is determined solely by the four 
% coefficients $\{ \tau_{\alpha \vec e_k}\}$ in the sense that
% $$
% \tr_{\not k} \mcE_{\{ \tau_\valpha\}}[\rho]  = 
% \mcE_{\{ \tau_\alpha \vec e_k\}}[\rho_k].
% % = \sum_\valpha \tau_\valpha r_\valpha \sigma_\valpha.
% $$
% % where $\mcE_{\{ \tau_{\alpha } \}}


% E completo, luego trazado, es equivalente a E reducido aplicado a la traza del estado. 


% For a single qubit
% the PCE generators are the identity map plus the channels that map the Bloch 
% sphere to each Cartesian axis, as their associated vector subspaces are trivial
% to prove and their dimension $2(1)-1=1$ is met. 
% \janote{Here goes the bla bla for the discussion of the proof for the $N$-qubit
% \pce{} generators}
% Therefore, we can identify the index $\alpha_k$ in $\vec\alpha$ by the local action
% of the PCE generator on the $k$-th qubit. 
% 

% Próximamente... \Fref{fig:appex:twoQ_PCEGs}
% }}}

