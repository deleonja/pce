\section{Local-action and labeling of PCE generators}
\label{sec:2_qubits}

\begin{figure} % {{{
\centering
\includegraphics[width=\columnwidth]{twoQ_generator_diagrams-crop}
\caption{PCE generators for 2 qubits. Notice that all generators
are either symmetrical or anti-symmetrical with respect 
to horizontal and vertical reflections. \cpnote{Este  caption solo tiene una iterada} \afnote{para mi está OK.}}
\label{fig:appex:twoQ_PCEGs}
\end{figure} % }}}





The local-action of a generator $\mcG_\valpha$ on every qubit in the 
system depends only, as its notation suggests, on the multi-index $\valpha$. 
This index has a simple meaning that can be read from the graphical representation
of the channel. 
Recall the single qubit PCE generators, shown in 
\fref{fig:A_and_PCEGs_connection}, denoted by  $\mcG_0$ (corresponding to the identity map) 
and $\mcG_{1,2,3}$ (corresponding to the completely bit, phase and bit-phase
flip channels respectively).
% In such diagram, every square's color indicates the value attained by the
% corresponding $\tau_{\alpha}$. 
One can easily read the diagrams in the following 
manner: $\alpha =0$ corresponds to all squares black, whereas for $\alpha>0$ we have
only the 0-th and the $\alpha$-th squares black. 
Let us  generalize this characterization rule for $N$-qubit 
PCE generators. 
Consider the reduced 
density matrix of the $k$-th qubit after generator $\mcG_\valpha$ acts
on the entire system,
\begin{align}
\tr_{\not k} \mcG_\valpha [\rho] &= 
\frac12 \tr_{\not k} (\rho + \sigma_\valpha \rho \sigma_\valpha)\nonumber \\
&= \frac{\rho_k}{2} + \frac{\sigma_{\alpha_k} \rho_k \sigma_{\alpha_k}}{2}\nonumber \\
&= \mcG_{\alpha_k} [\rho_k ],
\label{eq:generator:individual:qubit}
\end{align}
where $\not k$ means all qubits except for the $k$-th one are traced out.
We can read from \eqref{eq:generator:individual:qubit} that 
$\alpha_k$ not only characterizes $\mcG_{\alpha_k}$ but actually tells 
us which single qubit channel is acting locally on the $k$-th qubit.  The
action of $\mcG_{\alpha_k}$ on the local components of the reduced density
matrix $\rho_k$ reads 
$r_{0,\ldots,j_k,\ldots,0}\mapsto 
\tau_{0,\ldots,j_k,\ldots,0}r_{0,\ldots,j_k,\ldots,0}$.
% 
% The Pauli components of the reduced density matrix $\rho_k$ are of the form
% $r_{0,\ldots,j_k,\ldots,0}$, and the action of 
% $\mcG_{\alpha_k}$ on them reads
% $r_{0,\ldots,j_k,\ldots,0}\mapsto 
% \tau_{0,\ldots,j_k,\ldots,0}r_{0,\ldots,j_k,\ldots,0}$.
The general characterization rule for all PCE generators $\mcG_\valpha$ is clear
now: if all $\tau_{0,\ldots,j_k,\ldots,0}=1$, then $\alpha_k=0$, 
otherwise, 
if $\tau_{0,\ldots,j_k,\ldots,0}=1$ (with $j_k>0$), then 
$\alpha_k=j_k$.
% 
% 
% without considering $\tau_{0,\ldots,0}$,
% if only one $\tau_{0,\ldots,j_k,\ldots,0}=1$, then 
% $\alpha_k=j_k$, or if all $\tau_{0,\ldots,j_k,\ldots,0}=1$, 
% then $\alpha_k=0$.
% Let us give an example to illustrate the PCE generators characterization. 
% For two qubits, the vector index $\valpha$ is of the form $(j_1,j_2)$,
% where $j_1$ equals to 0, 1, 2, or 3 depending on the values $\tau_{j_1,0}$
% and the aforementioned rule; the same follows for $j_2$.
For two qubit PCE diagrams this means
that the multi-index $\valpha$ is encoded in the first column and row of 
the diagrams. See $\mcG_{(0,2)}$ in \fref{fig:appex:twoQ_PCEGs},
all $\tau_{j_1,0}=1$ and $\tau_{(0,2)}=1$, then $\valpha=(0,2)$. 
In \fref{fig:appex:twoQ_PCEGs} we show all two qubits PCE generators
and their corresponding notation $\mcG_\valpha$.


% \janote{Mi contrapropuesta}
% That is to say, without regarding $\tau_0$, $\alpha=j$ when only one $\tau_{j}=1$,
% and $\alpha=0$ when all $\tau_{j}=1$.
% Let us start discussing the PCE
% single qubit generators $\mcG_\alpha$. They are shown in 
% \fref{fig:A_and_PCEGs_connection}, $\mcG_0$ corresponds to the identity map, 
% and the remaining $\mcG_{1,2,3}$  are the completely bit, phase and bit-phase
% flip channels;
% \janote{
% recall that in a PCE diagram every square's color indicates the value 
% attained by the corresponding $\tau_{\alpha}$.
% That is to say, without regarding $\tau_0$, $\alpha=j$ when only one $\tau_{j}=1$,
% and $\alpha=0$ when all $\tau_{j}=1$.
% }
% Now, \janote{to generalize this characterization rule for a $N$-qubit 
% PCE generator $\mcG_\valpha$} we consider the reduced 
% density matrix of the $k$-th qubit after generator $\mcG_\valpha$ acts
% on the entire system,
% \begin{align}
% \tr_{\not k} \mcG_\valpha [\rho] &= 
% \frac12 \tr_{\not k} (\rho + \sigma_\valpha \rho \sigma_\valpha)\nonumber \\
% &= \frac{\rho_k}{2} + \frac{\sigma_{\alpha_k} \rho_k \sigma_{\alpha_k}}{2}\nonumber \\
% &= \mcG_{\alpha_k} [\rho_k ],
% \label{eq:generator:individual:qubitx}
% \end{align}
% where $\not k$ means all qubits except for the $k$-th one are traced out.
% \janote{
% Note in the second line of \eqref{eq:generator:individual:qubit} that 
% $\alpha_k$ not only characterizes $\mcG_{\alpha_k}$ but actually tells 
% us which channel is locally acting on the $k$-th qubit, whether a 
% completely bit, phase, bit-phase, or identity channel.
% In the last line of \eqref{eq:generator:individual:qubit} 
% $\mcG_{\alpha_k}$ is a single qubit PCE generator and the Pauli components
% of $k$-th qubit's reduced density matrix $\rho_k$ are of the form
% $r_{0,\ldots,j_k,\ldots,0}$, 
% meaning that $\mcG_{\alpha_k}$ acts on $\rho_k$ as 
% $r_{0,\ldots,j_k,\ldots,0}\mapsto 
% \tau_{0,\ldots,j_k,\ldots,0}r_{0,\ldots,j_k,\ldots,0}$.
% Thus, the general characterization rule of all PCE generators $\mcG_\valpha$ 
% is clear now:
% without considering $\tau_{0,\ldots,0}$,
% if only one $\tau_{0,\ldots,j_k,\ldots,0}=1$, then 
% $\alpha_k=j_k$, or if all $\tau_{0,\ldots,j_k,\ldots,0}=1$, 
% then $\alpha_k=0$.
% Let us give an example to illustrate the PCE generators characterization. 
% For two qubits, the vector index $\valpha$ is of the form $(j_1,j_2)$,
% where $j_1$ equals to 0, 1, 2, or 3 depending on the values $\tau_{j_1,0}$
% and the aforementioned rule; the same follows for $j_2$.
% In PCE diagrams language this means
% that the vector index $\valpha$ is encoded in the first column and row of 
% a diagram. See $\mcG_{(0,2)}$ in \fref{fig:appex:twoQ_PCEGs},
% all $\tau_{j_1,0}=1$ and $\tau_{(0,2)}=1$, then $\valpha=(0,2)$. 
% In \fref{fig:appex:twoQ_PCEGs} we show all two qubits PCE generators
% and their corresponding notation $\mcG_\valpha$.
% }  
% %The last line denotes a single qubit PCE generator
% %corresponding to the index $\alpha_k$, which is the $k$-th component 
% %of $\valpha$. Hence, each component of the vector-index $\vec\alpha$ 
% %provides what the local-action of $\mcG_\valpha$ is on each qubit in
% %the system. For two qubits, the local-action of PCE quantum channels 
% %in the diagrams is seen in the first row (qubit 1) and first column (qubit 2).
% %We present in \fref{fig:appex:twoQ_PCEGs}
% %all two qubit generators and their corresponding notation $\mcG_\valpha$. 
% %\cpnote{Falta justo lo que va en la segunda parte de mi texto. Es decir asociar la 
% %accion local sobre cada qubit con la primer fila o oclumna, etc del diagrama}

% Comentarios viejos {{{
% Esto es equivalente al canal k sobre un qubut, y se puede extrapolar de la Figura 1. 

% Así podemos leer la accion y relacionar las figuras con la notacion de las alphas.


%  Then, we can write the partial trace 
% of the generic state $\rho$, see \eref{eq:N_qubits_rho}, as
% $$
% \rho_k = 
% \tr_{\not k} \rho = 
% \sum_\alpha r_{\alpha \vec e_k} \sigma_\alpha.
% $$


% Conectar los elementos tau de la primera fila con la accion sobre el qubit individual. 

% We also introduce basis vectors for multi index 
% This means that 


% We shall now study the reduced action of a PCE map over a particular qubit. To do 
% so, we introduce the partial trace over everything except qubit $k$; this 
% operation  will be denoted by $\tr_{\not k}$. We also introduce basis vectors for multi index 
% $\valpha$. Define $\vec e_k$ via its components as
% $[\vec e_k]_j = \delta_{jk}$. Then, we can write the partial trace 
% of the generic state $\rho$, see \eref{eq:N_qubits_rho}, as
% $$
% \rho_k = 
% \tr_{\not k} \rho = 
% \sum_\alpha r_{\alpha \vec e_k} \sigma_\alpha.
% $$

% Consider now the multi qubit Pauli map defined by coefficients $\{ \tau_\valpha\}$, 
% whose action is 
% $$
% \mcE_{\{ \tau_\valpha\}}[\rho] = \sum_\valpha \tau_\valpha r_\valpha \sigma_\valpha.
% $$
% The action on qubit $k$ is determined solely by the four 
% coefficients $\{ \tau_{\alpha \vec e_k}\}$ in the sense that
% $$
% \tr_{\not k} \mcE_{\{ \tau_\valpha\}}[\rho]  = 
% \mcE_{\{ \tau_\alpha \vec e_k\}}[\rho_k].
% % = \sum_\valpha \tau_\valpha r_\valpha \sigma_\valpha.
% $$
% % where $\mcE_{\{ \tau_{\alpha } \}}


% E completo, luego trazado, es equivalente a E reducido aplicado a la traza del estado. 


% For a single qubit
% the PCE generators are the identity map plus the channels that map the Bloch 
% sphere to each Cartesian axis, as their associated vector subspaces are trivial
% to prove and their dimension $2(1)-1=1$ is met. 
% \janote{Here goes the bla bla for the discussion of the proof for the $N$-qubit
% \pce{} generators}
% Therefore, we can identify the index $\alpha_k$ in $\vec\alpha$ by the local action
% of the PCE generator on the $k$-th qubit. 
% 

% Próximamente... \Fref{fig:appex:twoQ_PCEGs}
% }}}


An interesting relation of the generators and the $A$ matrix can be derived 
with the tools developed. 
Consider the generator 
$\mcG_\valpha$, and its Pauli components $\tau^{(\valpha)}_\vbeta$. We 
can calculate the former studying the action of the generator on the 
non-normalized state $\varrho =  \sum_\vgamma \sigma_\vgamma$:
Let us proceed with such calculation, using the Kraus decomposition \eref{eq:kraus:PCE}:
\begin{align}
\tau^{(\valpha)}_\vbeta &= \tr \sigma_\vbeta \mcG_\valpha[\varrho] \\
&= \frac12 \sum_\vgamma \tr \left[ \sigma_\vbeta \sigma_\vgamma
+ \sigma_\vbeta \sigma_\valpha \sigma_\vgamma \sigma_\valpha \right] \\
&= \frac12 \left(1+ A_{\valpha \vbeta} \right)
\end{align}
where we have used the orthogonality relations of Pauli matrices and \eref{eq:sigma_property}.
This means that one can read the $\alpha$-th generators directly from matrix $A$, see 
\fref{fig:A_and_PCEGs_connection} for the $n=1$ case. Alternatively 
one could construct the $A$ matrix for $n=2$, from \fref{fig:appex:twoQ_PCEGs}, 
where the first row of this matrix is read from $\mcG_{(0,0)}$, replacing
black (white) squares with 1's ($-1$s), the second row from $\mcG_{(0,1)}$, etc. 


