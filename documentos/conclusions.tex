\section{Conclusions}
\label{sec:conclusions}
\todo{al final}
\ddnote{Avienten ideas en cualquier idioma de lo que quieren que venga en las conclusiones}
   
% Frase de David {{{
\ddnote{Agrego esto:} A final remarks of the derivation as a tool. Concepts
might be complicated, but the tool gives some simple rules to the reader:
\textit{tell me what components you want to keep, and the (vector) summation
rules will give you the rest of components to make the target operation a CPTP
one.} \janote{Me gusta esto que puso David, pero lo movería a las conclusiones.}
\par % }}}

In this work we introduce and fully characterize 
a novel type of Pauli maps: Pauli component erasing (PCE) maps.
%PCE maps are operations that project a $N$-qubit density matrix 
%onto subsets of the Pauli strings basis. 
A PCE map is an operation that acts over the density matrix of a system 
of qubits and either preserves or completely erases the projections of 
the density matrix onto the elements of Pauli strings basis; 
we call this projections the Pauli components of the density matrix
of a system of qubits.
%By construction, PCE maps are trace-preserving.
We introduce in Figs. \ref{fig:one:qubit:examples} and 
\ref{fig:two:qubit:examples} diagrams to schematically represent PCE maps
of 1 and 2 qubits. 
%We study the conditions for a PCE map to represent a 
%quantum physical evolution, i.e. to be a quantum channel, thus we 
%investigated the conditions a PCE map needs to fulfill to 
%be completely positive. 
We study the complete positivity condition a PCE map has to fulfill to represent
a quantum physical evolution, i.e. to be a quantum channel, and we 
describe in detail the properties of PCE quantum channels.

We provide an exact diagonalization for Pauli diagonal maps' 
Choi matrix and evaluate the complete positivity condition, 
as established in the Choi-Jamiokolwsky isomorphism. 
Then, restricting ourselves to PCE maps we find an interesting algebraic structure:
if we number the Pauli matrices by pairs of binary indices and describe a Pauli string as the
corresponding vector of binary numbers, 
%the set on which a given PCE quantum channel is equal to 1,
the set of indices vectors of the Pauli strings to which 
the density matrix is projected by a given PCE quantum channel is
a vector space over the field of two elements $\{0,1\}$.
From this, we derive all properties of PCE quantum channels.
They are not only characterized by the number
of Pauli components preserved but for the particular sets of components preserved too.
If one wishes, say, to preserve at least a given set of strings, the vector space structure 
determines uniquely the minimal set of Pauli strings we must add to obtain  a quantum channel.
Furthermore, PCE quantum channels posses generators, understood
as maximal vector subspaces. PCE quantum channels have symmetries. From
the vector structure it follows the existence of PCE generators as 
maximal non trivial subspaces. This PCE generators have symmetries.

% ideas{{{
\textcolor{jacolor}{
\begin{enumerate}
\item We introduced a novel type of Pauli maps: Pauli component erasing (PCE) maps.
PCE maps are operators that totally project the density matrix of a system on $N$ 
qubits onto subsets of elements of Pauli strings basis. PCE maps either
preserve or completely erase the projections of the density matrix 
onto the elements of Pauli strings basis. By construction, PCE maps are
trace-preserving. In Figs \ref{fig:one:qubit:examples} and 
\ref{fig:two:qubit:examples} we introduced diagrams to represent PCE maps
of 1 and 2 qubits. We were interested in investigating PCE maps 
that represent quantum physical evolutions, i.e., PCE maps that are
quantum channels. The problem is reduced to finding the conditions for
a PCE map to be completely positive.
\item We provided exact diagonalization for Pauli diagonal maps in general.
To evaluate the CP condition we evaluated the positivity of Choi matrix of
a PCE map, as established in the Choi-Jamiokolwsky isomorphism. Exact 
diagonalization shows a structure where the CP conditions of general 
Pauli diagonal maps are encoded in a matrix of 1 qubit.
\item We then restricted to PCE maps and investigated the algebraic structure
of PCE quantum channels. We showed that PCE quantum channels have an associated
vector space structure over the field of two elements $\{0,1\}$. 
From there, we derived all properties of PCE 
quantum channels. PCE quantum channels are not only characterized by the number
of Pauli components preserved but for the particular sets of components preserved too.
The coefficients of Pauli components must form a vector space.
\item We showed that PCE quantum channels posses generators, understood
as maximal vector subspaces. PCE quantum channels have symmetries. From
the vector structure it follows the existence of PCE generators as 
maximal non trivial subspaces. This PCE generatos have symmetries.
\item Kraus operators of PCE quantum channels are easy. They are the 
identity plus...
\item Physical implementation of PCE maps.
\end{enumerate} }
% }}}


