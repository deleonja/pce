\clearpage
\section{Conclusions}
\label{sec:conclusions}
%\todo{al final}
%\ddnote{Avienten ideas en cualquier idioma de lo que quieren que venga en las conclusiones}
%   
%% Frase de David {{{
%\ddnote{Agrego esto:} A final remarks of the derivation as a tool. Concepts
%might be complicated, but the tool gives some simple rules to the reader:
%\textit{tell me what components you want to keep, and the (vector) summation
%rules will give you the rest of components to make the target operation a CPTP
%one.} \janote{Me gusta esto que puso David, pero lo movería a las conclusiones.}
%\par % }}}
\ddnote{Bullet points para contenido:}
{\blue \begin{itemize}
\item We generalize idempotent Pauli operations (\ie{} flip channels with flip probability of $1/2$), to the $N$ particle case.
\item To characterize them (evaluate CP conditions) a crucial step was to identify channels as finite vector spaces and use the powerful theory of vector spaces.
\item We found a semigroup structure where generators are in fact generalized flip operations, \ie{} channels that with probability $1/2$ apply a joint flip described with $\sigma_{\vec \alpha}$. \item As a note, partial tracing the effect of generators, give the same generators for smaller systems, in particular leaving only one qubit one recovers the known flip channels.
\item An interesting property of generators is that they can be implemented using always a single qubit as environment, therefore \pce{} channels can be viewed as the quantum system interacting stroboscopically (or colliding) with a qubit that is reset after each collision.
\item Decomposing \pce{} channels into their generators was also exploited to compute Markovian dynamics for which a target \pce{} channel is the fixed point.
\end{itemize}
}

In this work we introduce and characterize a set of multi-qubit  maps: Pauli
component erasing (PCE) maps. These maps 
remove some components of a system of qubits, 

in which the components of a mutiqubit system, expressed as a combination 
of Pauli strings, are preserved of completely erased. 




In this work we introduce and fully characterize 
a novel type of Pauli maps: Pauli component erasing (PCE) maps.
%PCE maps are operations that project a $N$-qubit density matrix 
%onto subsets of the Pauli strings basis. 
We call the Pauli components of the density matrix of a system of qubits 
to the projections of the density matrix onto the elements of Pauli string 
basis.
Then, a PCE map is defined as a \janote{positive and} linear operation acting over 
the density matrix of a system of qubits either preserving or 
completely erasing its Pauli components.
%By construction, PCE maps are trace-preserving.
In Figs. \ref{fig:one:qubit:examples} and \ref{fig:two:qubit:examples} 
we introduce diagrams that graphically represent PCE maps of 1 and 2 qubits. 
%We study the conditions for a PCE map to represent a 
%quantum physical evolution, i.e. to be a quantum channel, thus we 
%investigated the conditions a PCE map needs to fulfill to 
%be completely positive. 
PCE maps are by construction positive and trace-preserving but not 
completely positive in order to be a quantum physical evolution, i.e.
a quantum channel.
Hence, we study the conditions a PCE map needs to met to be 
a PCE quantum channel, and we describe in detail the properties 
of this subset of all PCE maps.

We provide an exact diagonalization for Pauli diagonal maps' 
Choi matrix and evaluate the complete positivity condition, 
as established in the Choi-Jamiokolwsky isomorphism. 
Then, restricting ourselves to PCE maps we find an interesting algebraic structure:
if we number the Pauli matrices by pairs of binary indices and describe a Pauli string as the
corresponding vector of binary numbers, 
the set of indices vectors of the Pauli strings to which 
the density matrix is projected by a given PCE quantum channel is
a vector space over the field of two elements $\{0,1\}$.
In other words, we find that PCE quantum channels may 
be understood as vector subspaces and from this 
derive all their properties.
PCE quantum channels are not only characterized by the number
of Pauli components preserved but in fact 
for the particular sets of components preserved 
[see \Fref{fig:two:qubit:examples}].
If one wishes, say, to preserve at least a given set of Pauli components, 
the vector space structure determines uniquely the minimal set of 
Pauli components we must add to obtain a quantum channel.

Furthermore, we find that PCE quantum channels posses generators
and these \textit{PCE generators} have symmetries.
In the vector space language of PCE quantum channels the PCE 
generators are described as maximal non trivial vector subspaces. 
For 2-qubits, the symmetries of PCE generators are easily described with 
the diagrams representing the corresponding maps 
[see \Fref{fig:appex:twoQ_PCEGs}], they are either symmetric or
antisymmetric with respect to a horizontal and vertical lines 
cutting the diagrams through the middle. Even more, we find an 
astonishing connection via this symmetry between PCE generators and the matrix 
that diagonalizes the Choi matrix of a Pauli diagonal channel: one can
take the rows or columns of this matrix, map all entries with -1 to 0, and
obtain all indices vectors of the PCE generators. 

Finally, we discuss two examples to see PCE quantum channels as
fixed points of a pure dissipative process and of a memory-less 
collision model. To derive these two implementations we exploit 
the fact that Kraus operators of PCE generators are the identity 
operator and one of the Pauli strings operators, up to a constant.
\janote{Falta un comentario para cerrar el párrafo}



\ddnote{Punchlines (sigo trabajando)}
{\blue 
\begin{itemize}
\item (la idea es conectar lo siguiente directo a los highlights, pensando que el ultimo highlight es el del espacio vectorial): the introduced vector structure 
\item Parrafo final: the study amounts to understand deeply decoherence for higher dimensions by generalizing the notion of qubit flip channels (with flip probability of $1/2$) to the case of $N$ qubits. Furthermore, we were able to prove that the generalized channels can be connected again 
\end{itemize}
}

This work may lead to future directions, being perhaps the most interesting
one the generalization of PCE maps for systems of qudits.
\janote{algo más para direcciones futuras de trabajo?}
