\section{Conclusions}
\todo{al final}
\ddnote{Avienten ideas en cualquier idioma de lo que quieren que venga en las conclusiones}
   
% Frase de David {{{
\ddnote{Agrego esto:} A final remarks of the derivation as a tool. Concepts
might be complicated, but the tool gives some simple rules to the reader:
\textit{tell me what components you want to keep, and the (vector) summation
rules will give you the rest of components to make the target operation a CPTP
one.} \janote{Me gusta esto que puso David, pero lo movería a las conclusiones.}
\par % }}}

% ideas{{{
\textcolor{jacolor}{
\begin{enumerate}
\item We introduced a novel type of Pauli maps: Pauli component erasing maps (PCE).
\item We provided exact diagonalization of all Pauli diagonal maps.
\item We then restricted ourseleves to PCE maps and understood all 
properties of PCE quantum channels from the vector structure of this
novel channels.
\item We showed that PCE quantum channels posses generators, understood
as maximal vector subspaces. PCE quantum channels have symmetries. 
\item Kraus operators of PCE quantum channels are easy.
\item Physical implementation of PCE maps.
\end{enumerate} }
% }}}


