% \clearpage

\section{Conclusions}
\label{sec:conclusions}
%\todo{al final}
%\ddnote{Avienten ideas en cualquier idioma de lo que quieren que venga en las conclusiones}
%   
%% Frase de David {{{
%\ddnote{Agrego esto:} A final remarks of the derivation as a tool. Concepts
%might be complicated, but the tool gives some simple rules to the reader:
%\textit{tell me what components you want to keep, and the (vector) summation
%rules will give you the rest of components to make the target operation a CPTP
%one.} \janote{Me gusta esto que puso David, pero lo movería a las conclusiones.}
%\par % }}}
% \ddnote{Bullet points para contenido:}

% \cpnote{We generalize idempotent Pauli operations (\ie{} flip channels with flip
% probability of $1/2$), to the $N$ particle case.}
% 
% In this work we introduce and fully characterize 
% a novel type of Pauli maps: Pauli component erasing (PCE) maps.
% %PCE maps are operations that project a $N$-qubit density matrix 
% %onto subsets of the Pauli strings basis. 
% We call the Pauli components of the density matrix of a system of qubits 
% to the projections of the density matrix onto the elements of Pauli string 
% basis.


In this work we introduce and characterize a set of quantum maps which either preserve
or completely erase the components of a multi-qubit density matrix, in the
basis of Pauli strings \afnote{Tal vez queda mejor hablar en pasado en la parte de conclusiones: In this work we have introduced and characterized...}; 
we call \afnote{called} those maps Pauli component erasing (PCE) maps.
For a single qubit these include the completely depolarizing
and dephasing channels.
To start the characterization, we note \afnote{noted} that 
not all PCE maps are quantum channels, as some are not completely positive.  In
fact, the most laborious task of this work was to evaluate complete positivity conditions
given by the Choi-\jami{} isomorphism, \janote{task} \cpnote{no} after which we showed that
the components of PCE quantum channels form a finite vector space.
% This leads immediately to a method to find
% constructively valid quantum maps starting with arbitrary \pce{} maps. 
This in turn allows \afnote{allowed} to unravel several properties, such as the possible number of PCE
channels, and the number of components preserved while also providing 
advantages to study numerically this set, for example by implying \afnote{employing (?)} an efficient method 
to construct all quantum channels for a given number of qubits.
\janote{No me gusta este enunciado, no me parece que resalte las
consecuencias más importantes de la estructura vectorial, pero tampoco tengo
una buena contra-propuesta}
%\janote{Propuesta para cambiar el enunciado anterior por:}
%\janote{
%This in turn allows to unravel several properties, such as the counting of PCE channels,
%the number of components preserved by a channel, and perhaps the most 
%important one, the fact that not all sets with equal number of preserved components
%are immediately a quantum channel, or said in other words, not all sets
%of preserved components form a vector space. Moreover, this algebraic structure also provides
%advantages for studying PCE channels numerically, for example, by implying an efficient method 
%to construct all quantum channels of this type for a given number of qubits.} 
%%
%\cpnote{
%This in turn allows to calculate several properties, such as the number of PCE channels,
%and the number of components preserved by a channel.  and perhaps the most 
%important one, the fact that not all sets with equal number of preserved components
%are immediately a quantum channel, or said in other words, not all sets
%of preserved components form a vector space\cpnote{No se 
%me hace importante}. Moreover, this algebraic structure also provides
%advantages for studying PCE channels numerically, for example, it allows to 
%efficiently construct all quantum channels of this type for a given number of qubits.} 
%
\afnote{Propongo que se inicie un nuevo párrafo desde aquí:} Similar to other objects in open quantum systems (for example, Lindblad
processes), PCE quantum channels form a semigroup, but finite in this case.
For
PCE channels, the generators  are generalized flip operations, \ie{} channels
that with probability $1/2$ apply a joint flip. 
This structure allows to link this channel with multi-qubit decoherence
processes which can be described, say by simple dissipative processes or
memory-less collision models, which in turn may pave a way to either implement
these channels or connect them with already existent decoherence families. 
% \ddnote{Lo anterior todavia lo siento como una lista sin mucha interconexcion. Contrapropuesta ya completa:}
This, together \janote{with} the discovered algebraic structure that translates complete
positivity into an explicit conditioned preservation of many-body correlations,
encompasses an advance in the knowledge of the mathematical structures
underlying general quantum channels.
\afnote{Propongo cambiar desde aqui hasta el final así (en un nuevo párrafo): We had provided a new family of quantum channels with a very special mathematical structure, thus extending our knowledge on the relation between quantum channels and many-body systems. In addition, the study of generalizations to high dimensional systems (qudits) and the understanding of the geometric role of PCE channels within the set of all quantum channels may be helpful to get a wider comprehension of open quantum systems.} Both generalizations (such as going from qubits to qudits) and
the geometric role of PCE channels
within the set of all quantum channels would be interesting to consider
and might further advance the understanding of open quantum systems. 
We had thus provided a family of quantum channels with a very special mathematical 
structure that allows to widen the understanding of quantum channels in the 
context of many-body systems. 


% In this work we introduced and characterized a family of channels that
% generalize Pauli idempotent channels to the multi-qubit case (\pce{} channels).
% The generalization was performed extending Pauli matrices to Pauli strings, the
% resulting maps (\pce{} maps) are diagonal in the Pauli string basis. Here,
% idempotence implies that either they nullify or leave invariant the components
% of the generalized Bloch vector, see~\eref{eq:N_qubits_rho}, which includes
% many-body correlations.
% %
% Not every \pce{} map is a quantum channel, so the main task of this work was to
% evaluate complete positivity conditions given by the Choi-\jami{} isomorphism.
% This was possible noticing that \pce{} channels have a very particular
% algebraic structure, they can be seen as finite vector subspaces of a larger
% finite vector space (representing the identity channel), see
% section~\ref{sec:vector_spaces}. This leads immediately to a method to find
% constructively valid quantum maps starting with arbitrary \pce{} maps. 
%
% This is, once choosing the generalized Bloch vector components that are going
% to be preserved, the finite vector space summation rule of the \pce{} map
% diagonal elements gives automatically the additional components that must be
% preserved in order to get a valid quantum channel.
% %
% The finite vector space structure in turn allowed to unravel several
% properties, such as the possible number of PCE
% channels, and the number of components preserved. 
% %
% Furthermore, \pce{} channels form a semigroup where generators are generalized
% flip operations, \ie{} channels that with probability $1/2$ apply a joint flip. 
% This structure allows us to link \pce{} channels with multi-qubit decoherence processes
% which can be described by simple dissipative processes or memory-less 
% collision models. 
% %
% This, together the discovered algebraic structure that translates complete positivity into an explicit conditioned preservation of many-body correlations, encompasses an advance in the knowledge of the mathematical structures underlying general quantum channels.

% 
% 
% 
% \cpnote{ We found a semigroup structure where generators are in fact generalized
% flip operations, \ie{} channels that with probability $1/2$ apply a joint flip
% described with $\sigma_{\vec \alpha}$. }
% 

% \cpnote{As a note, partial tracing the
% effect of generators, give the same generators for smaller systems, in
% particular leaving only one qubit one recovers the known flip channels.}


% \cpnote{ Falta mas de decoherencia}
% \cpnote{ An interesting property of generators is that they can be implemented
% using always a single qubit as environment, therefore \pce{} channels can be
% viewed as the quantum system interacting stroboscopically (or colliding) with a
% qubit that is reset after each collision.}
% 
% \cpnote{ Decomposing \pce{} channels into their generators was also exploited to
% compute Markovian dynamics for which a target \pce{} channel is the fixed
% point.  }
% 
% Finally, we discuss two examples to see PCE quantum channels as
% fixed points of a pure dissipative process and of a memory-less 
% collision model. To derive these two implementations we exploit 
% the fact that Kraus operators of PCE generators are the identity 
% operator and one of the Pauli strings operators, up to a constant.
% \janote{Falta un comentario para cerrar el párrafo}


% Intro de Francois y Jose Alfredo {{{
% In this work we introduce and fully characterize 
% a novel type of Pauli maps: Pauli component erasing (PCE) maps.
%PCE maps are operations that project a $N$-qubit density matrix 
%onto subsets of the Pauli strings basis. 
% We call the Pauli components of the density matrix of a system of qubits 
% to the projections of the density matrix onto the elements of Pauli string 
% basis.
% Then, a PCE map is defined as a \janote{positive and} linear operation acting over 
% the density matrix of a system of qubits either preserving or 
% completely erasing its Pauli components.
%By construction, PCE maps are trace-preserving.
% In Figs. \ref{fig:one:qubit:examples} and \ref{fig:two:qubit:examples} 
% we introduce diagrams that graphically represent PCE maps of 1 and 2 qubits. 
%We study the conditions for a PCE map to represent a 
%quantum physical evolution, i.e. to be a quantum channel, thus we 
%investigated the conditions a PCE map needs to fulfill to 
%be completely positive. 
% PCE maps are by construction positive and trace-preserving but not 
% completely positive in order to be a quantum physical evolution, i.e.
% a quantum channel.
% Hence, we study the conditions a PCE map needs to met to be 
% a PCE quantum channel, and we describe in detail the properties 
% of this subset of all PCE maps.

% We provide an exact diagonalization for Pauli diagonal maps' 
% Choi matrix and evaluate the complete positivity condition, 
% as established in the Choi-Jamiokolwsky isomorphism. 
% Then, restricting ourselves to PCE maps we find an interesting algebraic structure:
% if we number the Pauli matrices by pairs of binary indices and describe a Pauli string as the
% corresponding vector of binary numbers, 
% the set of indices vectors of the Pauli strings to which 
% the density matrix is projected by a given PCE quantum channel is
% a vector space over the field of two elements $\{0,1\}$.
% In other words, we find that PCE quantum channels may 
% be understood as vector subspaces and from this 
% derive all their properties.
% PCE quantum channels are not only characterized by the number
% of Pauli components preserved but in fact 
% for the particular sets of components preserved 
% [see \Fref{fig:two:qubit:examples}].
% If one wishes, say, to preserve at least a given set of Pauli components, 
% the vector space structure determines uniquely the minimal set of 
% Pauli components we must add to obtain a quantum channel.
% Furthermore, we find that PCE quantum channels posses generators
% and these \textit{PCE generators} have symmetries.
% In the vector space language of PCE quantum channels the PCE 
% generators are described as maximal non trivial vector subspaces. 
% For 2-qubits, the symmetries of PCE generators are easily described with 
% the diagrams representing the corresponding maps 
% [see \Fref{fig:appex:twoQ_PCEGs}], they are either symmetric or
% antisymmetric with respect to a horizontal and vertical lines 
% cutting the diagrams through the middle. Even more, we find an 
% astonishing connection via this symmetry between PCE generators and the matrix 
% that diagonalizes the Choi matrix of a Pauli diagonal channel: one can
% take the rows or columns of this matrix, map all entries with -1 to 0, and
% obtain all indices vectors of the PCE generators. 
% }}}




% \ddnote{Punchlines (sigo trabajando)}
% {\blue 
% \begin{itemize}
% \item (la idea es conectar lo siguiente directo a los highlights, pensando que el ultimo highlight es el del espacio vectorial): the introduced vector structure 
% \item Parrafo final: the study amounts to understand deeply decoherence for higher dimensions by generalizing the notion of qubit flip channels (with flip probability of $1/2$) to the case of $N$ qubits. Furthermore, we were able to prove that the generalized channels can be connected again 
% \end{itemize}
% }
% 
% This work may lead to future directions, being perhaps the most interesting
% one the generalization of PCE maps for systems of qudits.
% \janote{algo más para direcciones futuras de trabajo?}
