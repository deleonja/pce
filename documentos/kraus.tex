\section{Kraus operators and decoherence} % {{{
% \todo{Carlos hace una version}


The Kraus operators of PCE channels provide an insight in the physical processes
that yield such channels. In this section we investigate and construct such 
operators and explore the corresponding processes. 


To study the Kraus operators corresponding to PCE, it is enough to study 
the ones corresponding to generators, as all PCEs commute and thus to get the 
Kraus operators of an arbitrary PCE, one can easily combine the Kraus
operators of its generators. 


% The Kraus operators corresponding to one single qubit PCEs are 
% $\{\openone/\sqrt2, \sigma_i/\sqrt2\}$,
% % \begin{equation}
% % \{\openone/\sqrt2, \sigma_i/\sqrt2\}
% % \label{eq:kraus:PCE}
% % \end{equation}
% corresponding to the operation that leaves the component $\sigma_i$
% invariant~\cite{nielsen_chuang_2011}.
Inspection of the Kraus operators of two qubit PCEG lead to
the ansatz that the Kraus operators for the generator of 
$\mcG_{\vec \alpha}$ is 
\begin{equation}
\left\{\frac{\openone}{\sqrt2}, \frac{\sigma_{\vec \alpha}}{\sqrt2}\right\}. 
% \left\{\openone/\sqrt2, \left(\bigotimes_{k=1}^N \sigma_{\alpha_k}\right)/\sqrt2\right\}. 
\label{eq:kraus:PCE}
\end{equation}
Indeed, the Kraus operators corresponding to one single qubit PCEs are 
$\{\openone/\sqrt2, \sigma_\alpha/\sqrt2\}$,
corresponding to the operation that leaves the component $\sigma_\alpha$
invariant~\cite{nielsen_chuang_2011}.

We shall first show that the aforementioned Kraus operators indeed produce a PCE. 
Notice that 
% \begin{equation}
$\sigma_\alpha \sigma_\beta \sigma_\alpha =a_{\alpha \beta} \sigma_\beta$,
% \label{}
% \end{equation}
which in turn imply that 
\begin{equation}
\sigma_{\vec \alpha} \sigma_{\vec \beta} \sigma_{\vec \alpha} 
=\san_{\vec \alpha \vec \beta} \sigma_{\vec \beta}.
\label{}
\end{equation}
\cpnote{This is the original $a$, not $H \otimes H$. See how to relate these
two matrices, or define a new $a$ and $A$}
% with $a$ the matrix that appears in \eqref{eq:eigvals-PCE}. 
Next, consider the action of a channel with Kraus representation 
\eqref{eq:kraus:PCE} on an $N$ qubit system:
\begin{equation}
% \begin{align}
\rho & \to 
 \frac{\rho}{2} 
 + \sum_{\vec \beta} \frac{\tau_{\vec \beta}}{2}
\sigma_{\vec \alpha} \sigma_{\vec \beta} \sigma_{\vec \alpha}  
&= 
  \sum_{\vec \beta} 
   \tau_{\vec \beta}\frac{1+\san_{\vec \alpha \vec \beta}}{2}\sigma_{\vec \beta}. 
% 
% 
% \frac{\otimes_k \sigma_{i_k}}{\sqrt{2}} \rho
% \frac{\otimes_k \sigma_{i_k}}{\sqrt{2}} \\
% &=\frac12 \left(\rho + \bigotimes_k \sigma_{i_k} \rho \bigotimes_k \sigma_{i_k} \right)\\
% &=\frac12 \rho
% + \frac12\sum_{j_1, \ldots,j_n} r_{j_1 \ldots j_n} \bigotimes_k  \sigma_{i_k} \sigma_{j_k}  \sigma_{i_k} \\
% &=
% \sum_{j_1, \ldots,j_n} r_{j_1 \ldots j_n} \frac{1+\prod_k a^{i_k}_{j_k}}{2} \bigotimes_k  \sigma_{j_k} . 
\end{equation}
However, since $\san_{\vec \alpha \vec \beta} = \pm 1$, the channel characterized
by Kraus operators  \eref{eq:kraus:PCE} is a PCE. Moreover, One can notice that, 
except for the first row, half of the matrix elements of each row is $+1$ and 
half $-1$, which implies that the aforementioned channel is a PCEG. 

% We still have to show that except for $i_1=\cdots=i_n=0$, half of the numbers 
% $(1+\prod_k a^{i_k}_{j_k})/2=1$, or equivalently, half of $\prod_k a^{i_k}_{j_k}=1$. 
% Consider a particular $l$ such that $i_l \ne 0$. This will make half of 
% the $\prod_k a^{i_k}_{j_k}$ 1 and half $-1$ (for fixed $l$). This also implies 
% that half of the correlations of the density matrix are erased and half preserved 
% so the channel is a PCEG. 

Notice also that a different choice of $\vec \alpha$ in \eref{eq:kraus:PCE}
lead to different channels. This can be seen noticing that if two channels were
the same, this would imply that the matrix representation of the corresponding
superoperator of $\sigma_{\vec \alpha} \cdot \sigma_{\vec \alpha}$
would have to be the same which is clearly false. Since there are $4^n$ different 
$\vec \alpha$, this implies that all PCEG (plus the identity map) are in one to 
one correspondence.  

\cpnote{We might have spoken about the notation of the generators and here we 
can add a sentence about it, but I will wait till JA writes his part.}
% Two remarks. (i) The notation that we develop with Jose Alfredo coincides with the 
% generator. This notation is based on the action on each individual qubit. (ii) Counting the
% number of generators here is trivial. It is simply $4^n-1$ as previously noticed 
% from other arguments. 

\begin{itemize}
\item Construct the Kraus operators, and relate to decoherence, for PCEG
\item Play with non PCEG PCEs
\item Dependiendo de si sale algo bonito, podemos poner algun modelo en
particular para todos los canales (sistema central + ambiente). Seria de buscar
dilataciones sencillas y razonables por lo menos a los canales generadores.
\todo{Davalos lidera la discusión de ese punto, pero no escribe nada}
\end{itemize} 
% }}}
