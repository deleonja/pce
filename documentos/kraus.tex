\section{Kraus operators and decoherence  propuesta: Physical implementations of PCE maps} % {{{
\label{sec:kraus}
% Intro {{{

\cpnote{Contrapropuesta: PCE channels and decoherence}

\cpnote{Crep qe hay que hacer una mini introduccion en donde se explique el porqué de
esta seccion. Básicamente la idea es conectar estos bichos con procesos fisicos.
Siento que la introduccion propuesta est´a un poco debil}

% \cpnote{HAcer esta seccion despues de que david ponga lo de decoherencia, 
% para servirla bien}
% \todo{Carlos hace una version}

% {\red [texto viejo introductorio de la seccion] Furthermore, quantum channels enjoy the well-known \textit{Kraus
% representation} (or operator-sum representation) which reads 
% \begin{equation}
% \mcE[\rho]=\sum_i K_i \rho
% K_i^\dagger{},
% \end{equation}
% with $\sum_i K_i^\dagger{}K_i=\one$ (the trace-preserving condition)~\cite{Kraus1983}.
% 
% The Kraus operators of PCE channels provide an insight in the physical processes
% that yield such channels. In this section we investigate and construct such 
% operators and explore the corresponding processes. To study the Kraus operators corresponding to PCE, it is enough to study 
% the ones corresponding to generators, as all PCEs commute and thus to get the 
% Kraus operators of an arbitrary PCE, one can easily combine the Kraus
% operators of its generators. }

Given that \pce{} channels are projections, we might ask whether they are fixed
points of some memory-less (Markovian) process or not. In this section we prove
that they are, and give two examples of Markovian implementations. The first
one consist in identifying each \pce{} channel as a fixed point of some pure
dissipative process. The second, in identifying each \pce{} channel as fixed
points of some memory-less collision model.
% }}}
\subsection{Kraus representation} % {{{
To derive the aforementioned implementations, we exploit the existence of the
PCEGs and their \textit{Kraus representation} (or operator-sum representation)
which, for an arbitrary channel $\mcE$,  reads 
\begin{equation}
\mcE[\rho]=\sum_i K_i \rho
K_i^\dagger{},
\end{equation}
with $\sum_i K_i^\dagger{}K_i=\one$ (the trace-preserving condition)~\cite{Kraus1983}.
% The Kraus operators corresponding to one single qubit PCEs are 
% $\{\openone/\sqrt2, \sigma_i/\sqrt2\}$,
% % \begin{equation}
% % \{\openone/\sqrt2, \sigma_i/\sqrt2\}
% % \label{eq:kraus:PCE}
% % \end{equation}
% corresponding to the operation that leaves the component $\sigma_i$
% invariant~\cite{nielsen_chuang_2011}.
Inspection of the Kraus operators of two qubit PCEG leads to
the ansatz that the Kraus operators for the generator of 
$\mcG_{\vec \alpha}$ are 
\begin{equation}
K_0=\frac{\openone}{\sqrt2}, \quad K_1=\frac{\sigma_{\vec \alpha}}{\sqrt2},
% \left\{\openone/\sqrt2, 
% \left(\bigotimes_{k=1}^N \sigma_{\alpha_k}\right)/\sqrt2\right\}. 
\label{eq:kraus:PCE}
\end{equation}
since the Kraus operators corresponding to one single qubit PCEs are 
$\{\openone/\sqrt2, \sigma_\alpha/\sqrt2\}$,
corresponding to the operation that leaves the component $\sigma_\alpha$
invariant~\cite{nielsen_chuang_2011}.

We shall first show that the aforementioned Kraus operators produce a PCE. 
Notice that 
% \begin{equation}
$\sigma_\alpha \sigma_\beta \sigma_\alpha =a_{\alpha \beta} \sigma_\beta$,
% \label{}
% \end{equation}
which in turn imply that 
\begin{equation}
\sigma_{\vec \alpha} \sigma_{\vec \beta} \sigma_{\vec \alpha} 
=\san_{\vec \alpha \vec \beta} \sigma_{\vec \beta}.
\label{eq:sigma_property}
\end{equation}
% with $a$ the matrix that appears in \eqref{eq:eigvals-PCE}. 
Next, consider the action of a channel with Kraus representation 
\eqref{eq:kraus:PCE} on an $N$ qubit system:
\begin{align}
\rho \mapsto &\frac{1}{2^N}\sum_{\vec \beta} r_{\vec \beta} \left( \frac{1}{2} \sigma_{\vec \beta} + \frac{1}{2} \sigma_{\vec \alpha} \sigma_{\vec \beta} \sigma_{\vec \alpha} \right)\nonumber \\
=&\frac{1}{2^N}\sum_{\vec \beta} r_{\vec \beta} \frac{1+A_{\vec \alpha \vec \beta}}{ 2}\sigma_{\vec \beta}.
\end{align}
However, since $\san_{\vec \alpha \vec \beta} = \pm 1$, the channel characterized
by Kraus operators  \eref{eq:kraus:PCE} is a PCE channel. Moreover, one can notice that, 
except for the first row, half of the matrix elements of each row is $+1$ and 
half $-1$, which implies that the channel is a PCEG. 

% We still have to show that except for $i_1=\cdots=i_n=0$, half of the numbers 
% $(1+\prod_k a^{i_k}_{j_k})/2=1$, or equivalently, half of $\prod_k a^{i_k}_{j_k}=1$. 
% Consider a particular $l$ such that $i_l \ne 0$. This will make half of 
% the $\prod_k a^{i_k}_{j_k}$ 1 and half $-1$ (for fixed $l$). This also implies 
% that half of the correlations of the density matrix are erased and half preserved 
% so the channel is a PCEG. 

Notice also that a different choice of $\vec \alpha$ in \eref{eq:kraus:PCE}
leads to different channels. This follows from the fact that if two channels were
the same, this would imply that the matrix representation of the corresponding
superoperator of $\rho \mapsto \sigma_{\vec \alpha}\rho \sigma_{\vec \alpha}$
% $\sigma_{\vec \alpha} \cdot \sigma_{\vec \alpha}$ 
% \ddnote{en este punto me confundio bastante esta notacion, contrapropongo $\rho
% \mapsto \sigma_{\vec \alpha}\rho \sigma_{\vec \alpha}$, que es una notación que
% hemos venido usando.}
would have to be the same which is clearly false. Since there are $4^n$ different 
$\vec \alpha$, this implies that all PCEG (plus the identity map) are in one to 
one correspondence.  

% \cpnote{We might have spoken about the notation of the generators and here we 
% can add a sentence about it, but I will wait till JA writes his part.}
% Two remarks. (i) The notation that we develop with Jose Alfredo coincides with the 
% generator. This notation is based on the action on each individual qubit. (ii) Counting the
% number of generators here is trivial. It is simply $4^n-1$ as previously noticed 
% from other arguments. 
% }}}
\subsection{Pure dissipative implementation} % {{{
Any PCE channel can be seen as the fixed point of some decoherence process\cpnote{esto suena obvio, pero no lo es. Es decir no todos los canales son resultado de
lindblad. Me gustaria modificar esta frase para que pueda ser apreciada mejor}. We
will show this first for PCEGs. Consider the following dynamical process that
implements the PCEG $\mcG_{\vec \alpha}$ when $t\to \infty$,
\begin{align}
\mcG_{\vec \alpha,t}[\rho]&=e^{- \gamma t} \rho + (1-e^{-\gamma t})\mcG_{\vec \alpha}[\rho]\nonumber \\
&=\frac{\left( 1+e^{- \gamma t} \right)}{2} \rho +\frac{\left( 1-e^{- \gamma t} \right)}{2} \sigma_{\vec \alpha} \rho \sigma_{\vec \alpha},
\end{align}
where $\gamma>0$. It is easy to show that the family of channels $\mcG_{\vec
\alpha, t}$ parametrized with $t \geq 0$ forms a one-parametric semigroup,
\ie{} $\mcG_{\vec \alpha, t_1}\mcG_{\vec \alpha, t_2}=\mcG_{\vec \alpha,
t_1+t_2}$, therefore $\mcG_{\vec \alpha, t}$ describes a dissipative
time-homogeneous Markovian process, those are always characterized by some
Lindblad generator~\cite{lindblad}. The Lindblad generator of $\mcG_{\vec
\alpha, t}$, denoted by $\mcL_{\vec \alpha}$, can be obtained with the standard
procedure,  
\begin{equation}
\mcL_{\vec \alpha}[\rho]=\left. \frac{d\mcG_{\vec \alpha,t}[\rho]}{dt}\right|_{t=0}=\frac{\gamma\left(\sigma_{\vec \alpha}\rho \sigma_{\vec \alpha} -\rho \right)}{2},
\end{equation}
where the unique Lindblad operator associated with the relaxation ratio
$\gamma/2$ is simply $\sigma_{\vec \alpha}$. Notice that $\sigma_{\vec \alpha}$
is trace-less, therefore the process is pure dissipative~\cite{cirac}.

Since PCEGs commute, we can describe easily any other PCE channel as a fixed
point of a decoherence process. For them, the Lindblad generators are the sum
of the Lindbladians of the corresponding generators. As an example, consider
the channel depicted in \fref{fig:two:qubit:examples} b), it is equal to
$\mcG_{(0,3)}\mcG_{(3,3)}$, therefore it is the fixed point of the dissipation
process described with the following Lindbladian,
\begin{equation}
\mcL[\rho]=\frac{\gamma_{(0,3)}\left(\sigma_{(0,3)}\rho\sigma_{(0,3)}-\rho \right)}{2}
   +\frac{\gamma_{(3,3)}\left(\sigma_{(3,3)}\rho\sigma_{(3,3)}-\rho
\right)}{2},
\end{equation}
where $\gamma_{(0,3)}$ and $\gamma_{(3,3)}$ are positive and correspond to the Lindblad operators $\sigma_{(0,3)}$ and $\sigma_{(3,3)}$. Notice that such election of Lindblad operators is not unique, as the PCE channel described here is also equal to $\mcG_{(0,3)}\mcG_{(3,0)}$.

% }}}
\subsection{Collision model implementation} % {{{
We show now that PCE channels can also be implemented with \textit{simple
collision models}~\cite{Ziman2010}. To do this, observe that using the Stinespring dilation theorem~\cite{Stinespring2006}, PCEGs can
be implemented using an unitary over the system and an
ancilla. Since PCEG
always have Kraus rank $2$, one can
always choose a qubit as the ancillary system. Concretely,
\begin{equation}
\mcG_{\vec \alpha}[\rho]=
    \tr_{\text{qubit}} \left( 
         U_{\vec \alpha} \left(\rho\otimes \projj{0}\right) U^\dagger{}_{\vec \alpha} 
     \right),
\end{equation}
where $\tr_{\text{qubit}}$ denotes partial trace over the ancillary qubit, with
the unitary defined as follows,
\begin{align}
U_{\vec\alpha}\ket{\psi} \ket{0}= \frac{1}{\sqrt{2}} 
   \left( \ket{\psi} \ket{0}+\sigma_{\vec\alpha} \ket{\psi}\ket{1} \right),\\
U_{\vec\alpha}\ket{\psi} \ket{1}=\frac{1}{\sqrt{2}} 
   \left( \ket{\psi} \ket{0}-\sigma_{\vec\alpha} \ket{\psi}\ket{1} \right).
\label{eq:Nqubit_stinespring}
\end{align}
Therefore, any concatenation of PCEGs can be described as a collision model
with as many collisions as generators needed. In fact, generators are described
with one collision. For the general case consider some PCE channel $\mcE$
generated with $\left\{ \mcG_{\vec \alpha_1}, \mcG_{\vec \alpha_2}, \dots,
\mcG_{\vec \alpha_M} \right\}$. For this we can define an environment
consisting of $M$ qubits initially in the state $\left( \projj{0}
\right)^{\otimes M}$, or equivalently one qubit with the additional assumption
that its state is reset to $\ket{0}$ after each collision (memory-less
collisions). The collision with the $k\text{th}$ particle is described by
$U_{\vec \alpha_k}$, which acts solely over the system and the $k\text{th}$
particle. Therefore $\mcE$ can be written as follows,
\begin{equation}
\mcE[\rho]=\tr_{\text{E}} \left[ \left(U_{\vec \alpha_1} \dots U_{\vec \alpha_M}\right) \rho \otimes \left(\projj{0}\right)^{\otimes M} \left(U_{\vec \alpha_1} \dots U_{\vec \alpha_M} \right)^\dagger{} \right],
\end{equation}
where $\tr_{\text{E}}$ is the partial trace over the total of the ancillary qubits. Notice that as PCEGs commute, the order of the collisions can be any.
% \begin{itemize}
% \item Relate to decoherence, for PCEG\ddnote{hecho}
% \item Play with non PCEG PCEs\ddnote{hecho}
% \item Dependiendo de si sale algo bonito, podemos poner algun modelo en
% particular para todos los canales (sistema central + ambiente). Seria de buscar
% dilataciones sencillas y razonables por lo menos a los canales generadores.\ddnote{hecho, es el asunto de collision models}
% \todo{Davalos lidera la discusión de ese punto, pero no escribe nada}
% \end{itemize} 
% 
% \begin{equation}
% \mcL_{\vec \alpha}[\Delta]:= \frac{\gamma}{2}\left( \sigma_{\vec\alpha} \Delta \sigma_{\vec\alpha} -\Delta \right).
% \label{eq:Nqubit_semigroup}
% \end{equation}


% }}}
% }}}
