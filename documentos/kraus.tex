\section{Kraus operators and decoherence} % {{{
\cpnote{HAcer esta seccion despues de que david ponga lo de decoherencia, 
para servirla bien}
% \todo{Carlos hace una version}

Furthermore, quantum channels enjoy the well-known \textit{Kraus
representation} (or operator-sum representation) which reads 
\begin{equation}
\mcE[\rho]=\sum_i K_i \rho
K_i^\dagger{},
\end{equation}
with $\sum_i K_i^\dagger{}K_i=\one$ (the trace-preserving condition)~\cite{Kraus1983}.

The Kraus operators of PCE channels provide an insight in the physical processes
that yield such channels. In this section we investigate and construct such 
operators and explore the corresponding processes. 


To study the Kraus operators corresponding to PCE, it is enough to study 
the ones corresponding to generators, as all PCEs commute and thus to get the 
Kraus operators of an arbitrary PCE, one can easily combine the Kraus
operators of its generators. 


% The Kraus operators corresponding to one single qubit PCEs are 
% $\{\openone/\sqrt2, \sigma_i/\sqrt2\}$,
% % \begin{equation}
% % \{\openone/\sqrt2, \sigma_i/\sqrt2\}
% % \label{eq:kraus:PCE}
% % \end{equation}
% corresponding to the operation that leaves the component $\sigma_i$
% invariant~\cite{nielsen_chuang_2011}.
Inspection of the Kraus operators of two qubit PCEG lead to
the ansatz that the Kraus operators for the generator of 
$\mcG_{\vec \alpha}$ is 
\begin{equation}
\left\{\frac{\openone}{\sqrt2}, \frac{\sigma_{\vec \alpha}}{\sqrt2}\right\}. 
% \left\{\openone/\sqrt2, \left(\bigotimes_{k=1}^N \sigma_{\alpha_k}\right)/\sqrt2\right\}. 
\label{eq:kraus:PCE}
\end{equation}
Indeed, the Kraus operators corresponding to one single qubit PCEs are 
$\{\openone/\sqrt2, \sigma_\alpha/\sqrt2\}$,
corresponding to the operation that leaves the component $\sigma_\alpha$
invariant~\cite{nielsen_chuang_2011}.

We shall first show that the aforementioned Kraus operators indeed produce a PCE. 
Notice that 
% \begin{equation}
$\sigma_\alpha \sigma_\beta \sigma_\alpha =a_{\alpha \beta} \sigma_\beta$,
% \label{}
% \end{equation}
which in turn imply that 
\begin{equation}
\sigma_{\vec \alpha} \sigma_{\vec \beta} \sigma_{\vec \alpha} 
=\san_{\vec \alpha \vec \beta} \sigma_{\vec \beta}.
\label{eq:sigma_property}
\end{equation}
\cpnote{This is the original $a$, not $H \otimes H$. See how to relate these
two matrices, or define a new $a$ and $A$} \ddnote{tengo entendido que nunca se define o redefine $a$ como $H \otimes H$, ni en el apendice, solo se dice que se intercambian para ciertos calculos. Creo que la notacion es segura.}
% with $a$ the matrix that appears in \eqref{eq:eigvals-PCE}. 
Next, consider the action of a channel with Kraus representation 
\eqref{eq:kraus:PCE} on an $N$ qubit system:
\begin{align}
\rho \mapsto &\frac{1}{2^N}\sum_{\vec \beta} r_{\vec \beta} \left( \frac{1}{2} \sigma_{\vec \beta} + \frac{1}{2} \sigma_{\vec \alpha} \sigma_{\vec \beta} \sigma_{\vec \alpha} \right)\nonumber \\
=&\frac{1}{2^N}\sum_{\vec \beta} r_{\vec \beta} \frac{1+A_{\vec \alpha \vec \beta}}{ 2}\sigma_{\vec \beta}
\end{align}
However, since $\san_{\vec \alpha \vec \beta} = \pm 1$, the channel characterized
by Kraus operators  \eref{eq:kraus:PCE} is a PCE channel\ddnote{agrego 'channel', aqui el PCE map si es por construccion un canal}. Moreover, One can notice that, 
except for the first row, half of the matrix elements of each row is $+1$ and 
half $-1$, which implies that the aforementioned channel is a PCEG. 

% We still have to show that except for $i_1=\cdots=i_n=0$, half of the numbers 
% $(1+\prod_k a^{i_k}_{j_k})/2=1$, or equivalently, half of $\prod_k a^{i_k}_{j_k}=1$. 
% Consider a particular $l$ such that $i_l \ne 0$. This will make half of 
% the $\prod_k a^{i_k}_{j_k}$ 1 and half $-1$ (for fixed $l$). This also implies 
% that half of the correlations of the density matrix are erased and half preserved 
% so the channel is a PCEG. 

Notice also that a different choice of $\vec \alpha$ in \eref{eq:kraus:PCE}
leads to different channels. This can be seen noticing that if two channels were
the same, this would imply that the matrix representation of the corresponding
superoperator of $\sigma_{\vec \alpha} \cdot \sigma_{\vec \alpha}$ \ddnote{en este punto me confundio bastante esta notacion, contrapropongo $\rho \mapsto \sigma_{\vec \alpha}\rho \sigma_{\vec \alpha}$, que es una notación que hemos venido usando.}
would have to be the same which is clearly false. Since there are $4^n$ different 
$\vec \alpha$, this implies that all PCEG (plus the identity map) are in one to 
one correspondence.  

\cpnote{We might have spoken about the notation of the generators and here we 
can add a sentence about it, but I will wait till JA writes his part.}
% Two remarks. (i) The notation that we develop with Jose Alfredo coincides with the 
% generator. This notation is based on the action on each individual qubit. (ii) Counting the
% number of generators here is trivial. It is simply $4^n-1$ as previously noticed 
% from other arguments. 

Any PCE channel can be seen as the fixed point of some decoherence process. We will show this first for PCEGs. Consider the following dynamical process that \textit{implements} the PCEG $\mcG_{\vec \alpha}$ when $t\to \infty$,
\begin{align}
\mcG_{\vec \alpha,t}[\rho]&=e^{- \gamma t} \rho + (1-e^{-\gamma t})\mcG_{\vec \alpha}[\rho]\nonumber \\
&=\frac{\left( 1+e^{- \gamma t} \right)}{2} \rho +\frac{\left( 1-e^{- \gamma t} \right)}{2} \sigma_{\vec \alpha} \rho \sigma_{\vec \alpha},
\end{align}
where $\gamma>0$. It is easy to prove that the family of channels $\mcG_{\vec \alpha, t}$ parametrized with $t \geq 0$ forms a one-parametric semigroup, \ie{} $\mcG_{\vec \alpha, t_1}\mcG_{\vec \alpha, t_2}=\mcG_{\vec \alpha, t_1+t_2}$, therefore $\mcG_{\vec \alpha, t}$ describes a dissipative time-homogeneous Markovian process, those are always characterized by some Lindblad generator~\cite{lindblad}. The Lindblad generator of $\mcG_{\vec \alpha, t}$, $\mcL_{\vec \alpha}$, can be obtained with the standard procedure,  
\begin{equation}
\mcL_{\vec \alpha}[\rho]=\left. \frac{d\mcG_{\vec \alpha,t}[\rho]}{dt}\right|_{t=0}=\frac{\gamma\left(\sigma_{\vec \alpha}\rho \sigma_{\vec \alpha} -\rho \right)}{2},
\end{equation}
where the unique Lindblad operator associated with the relaxation ratio $\gamma/2$ is simply $\sigma_{\vec \alpha}$.

Since PCEGs commute, we can describe easily any other PCE channel as a fixed point of a decoherence process. For them, the Lindblad generators are simply the sum of the Lindbladians of the corresponding generators. As an example, consider the channel depicted in \fref{fig:two:qubit:examples} b), it is equal to $\mcG_{(0,3)}\mcG_{(3,3)}$, therefore it is the fixed point of the dissipation process described with the following Lindbladian,
\begin{equation}
\mcL=\frac{\gamma_{(0,3)}\left(\sigma_{(0,3)}\rho\sigma_{(0,3)}-\rho \right)}{2}+\frac{\gamma_{(3,3)}\left(\sigma_{(3,3)}\rho\sigma_{(3,3)}-\rho \right)}{2},
\end{equation}
where $\gamma_{(0,3)}$ and $\gamma_{(3,3)}$ are positive and correspond to the Lindblad operators $\sigma_{(0,3)}$ and $\sigma_{(3,3)}$. Notice that such election of Lindblad operators is not unique, as the PCE channel described here is also equal to $\mcG_{(0,3)}\mcG_{(3,0)}$.

In addition to the implementation via decoherence processes, PCE channels can also be described with collision models [cita]. To see this, observe that PCEGs can always be implemented using an unitary over the system and an ancilla [Stinespring citation]. For any PCEG, a qubit can always be chosen as the ancilla, concretely,
\begin{equation}
\mcG_{\vec \alpha}[\rho]:=\tr_{\text{qubit}} \left( U_{\vec \alpha} \left(\rho\otimes \projj{0}\right) U^\dagger{}_{\vec \alpha} \right),
\end{equation}
where $\tr_{\text{qubit}}$ denotes partial trace over the ancillary qubit, with
\begin{align}
U_{\vec\alpha}\ket{\psi} \ket{0}:=\frac{1}{\sqrt{2}} \left( \ket{\psi} \ket{0}+\sigma_{\vec\alpha} \ket{\psi}\ket{1} \right),\\
U_{\vec\alpha}\ket{\psi} \ket{1}:=\frac{1}{\sqrt{2}} \left( \ket{\psi} \ket{0}-\sigma_{\vec\alpha} \ket{\psi}\ket{1} \right).
\label{eq:Nqubit_stinespring}
\end{align}
Therefore, any concatenation of PCEGs can be described as a collision model with as many collisions as generators needed. In fact, generators are described with one collision. For the general case, consider some PCE channel $\mcE$ generated with $\left\{ \mcG_{\vec \alpha_1}, \mcG_{\vec \alpha_2}, \dots, \mcG_{\vec \alpha_M} \right\}$. For this, we have an environment consisting of $M$ qubits initially in the state $\left( \projj{0} \right)^{\otimes M}$ (or equivalently, one qubit environment with the additional assumption that its state is erased and transformed again $\ket{0}$ after each collision). The collision with the $k\text{th}$ particle is described by $U_{\vec \alpha_k}$, which acts solely over the system and the $k\text{th}$ particle. Therefore $\mcE$ can be written as follows,
\begin{equation}
\mcE[\rho]=\tr_{\text{E}} \left( \left(U_{\vec \alpha_1} \dots U_{\vec \alpha_M}\right)\left( \rho \otimes \projj{0}^{\otimes M}\right) \left(U_{\vec \alpha_1} \dots U_{\vec \alpha_M} \right)^\dagger{} \right),
\end{equation}
where $\tr_{\text{E}}$ is the partial trace over the total of the ancillary qubits. Notice that as PCEGs commute, the order of the collisions can be changed, taking care also of the order of the partial traces.
$U_{\vec\alpha_M}U_{\vec\alpha_{M-1}}\dots U_{\vec\alpha_1}$
Described as the system \textit{colliding} with qubit particles
\begin{itemize}
\item Relate to decoherence, for PCEG
\item Play with non PCEG PCEs
\item Dependiendo de si sale algo bonito, podemos poner algun modelo en
particular para todos los canales (sistema central + ambiente). Seria de buscar
dilataciones sencillas y razonables por lo menos a los canales generadores.
\todo{Davalos lidera la discusión de ese punto, pero no escribe nada}
\end{itemize} 

\begin{equation}
\mcL_{\vec \alpha}[\Delta]:= \frac{\gamma}{2}\left( \sigma_{\vec\alpha} \Delta \sigma_{\vec\alpha} -\Delta \right).
\label{eq:Nqubit_semigroup}
\end{equation}


% }}}
