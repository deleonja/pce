\section{Mathematical considerations} 
% Intro {{{
This section is devoted to derive the conditions a Pauli diagonal map needs to satisfy 
the complete positivity condition i.e. that all the eigenvalues of the Choi matrix
associated to the channel are non-negative.  To do so, we calculate
and diagonalize the Choi matrix associated with general Pauli diagonal maps, first for a
single qubit and then for $N$ qubits. We then restrict to PCEs and, for quantum
channels, associate a vector subspace to the set of coefficients
$\{\tau_\valpha\}$ in \eref{eq:PCE_definition}. This allows us to
derive several important \janote{properties of this particular family of channels.}
\sout{conclusions. }
% }}}
\subsection{Diagonalization of Choi matrix} % {{{

We now construct the Choi matrix associated to the single qubit map
$\mathcal{E}$. As described above, $\mathcal{E}$ is a linear map from
$\mcB(\hilbert)$ to itself. We shall denote elements of
$\mcB(\hilbert)$ by the notation $\kket{\cdot}$. Thus, for instance,
$\kket{\sigma_\alpha}$ represents the Pauli matrix $\sigma_\alpha$ understood
as a vector belonging to $\mcB(\hilbert)$, for the present case, in which
$\hilbert={\mathbb{C}^2}$. Since the scalar product on $\mcB(\hilbert)$ is
given by  $\bbrakket{A_1}{A_2}=\tr A_1^\dagger{} A_2$,
% We now study the matrix form of the single qubit map $\mathcal{E}$ in
% order to construct the associated Choi matrix. 
% To do so, we shall denote operators in the superoperator space with the symbol
% $\kket{\cdot}$. For example, $\kket{\sigma_{\alpha}}$ correspond to the
% vectorized 
elements of the Pauli basis satisfy the relation 
$\bbrakket{{\sigma_{\alpha}} }{{\sigma_{\alpha'} }} = \tr
\left(\sigma_{\alpha}^{\dagger} \sigma_{\alpha'}\right) = 2
\delta_{\alpha\alpha'}$.
In this language, the state reads
$\kket{\rho}=2^{-1}\sum_{\alpha=0}^{3} r_{\alpha} \kket{\sigma_{\alpha}}$
and the matrix form of the map $\mathcal{E}$ is
% 
\begin{equation}
\hat{\mathcal{E}} = \frac{1}{2}\sum_{\alpha=0}^{3} \tau_{\alpha}
\ddyada{\sigma_{\alpha}}.
\end{equation}
After some steps, detailed from \eref{eq:inicio:canal:pauli} to
\eref{Choi_Pauli_1q}, it is possible to show that the associated Choi matrix
reads
\begin{equation}
 \mathcal{D} = \frac{1}{2}\sum_{\alpha=0}^{3} \tau_{\alpha} \sigma_{\alpha}
\otimes \sigma_{\alpha}^*.
%  \label{Choi_1}
\end{equation}
Notice that $\ddyada{\sigma_{\alpha}}$ 
and $\sigma_{\alpha} \otimes \sigma_{\alpha}^*$ are different operators.
Indeed, the former acts as a linear map upon the vector space $\mcB(\hilbert)$,
whereas the latter acts on the tensor product $\hilbert\otimes\hilbert$. Of
course, there is a basis dependent identification between these two
spaces, which is used 
in the construction of the Choi matrix.
% \cpnote{Aca poner algo de que no es trivial y que se detalla en el apendice.}
% 
% Moreover, observe that $\mathcal{D}$ has the same form as the matrix
% representation of a random unitary channel, with Kraus operators given by
% $K_{\alpha}=\sqrt{\tau_{\alpha}/2}\sigma_{\alpha}$ \cite{CHRUSCINSKI20131425}.
Surprisingly, one can in fact show that $\mathcal{D}$ is diagonal in the Pauli
basis (see Appendix \ref{sec:appendix:diagonalization}  for details).
% In this way the Choi matrix $\mathcal{D}$ is diagonal in the Pauli basis (see
% Appendix for details),
%
% \begin{equation}
%  \mathcal{D} \kket{\sigma_{\alpha}}= \lambda_{\alpha} \kket{\sigma_{\alpha}},
% \end{equation}
% 
The eigenvalues are
\begin{equation}
 \lambda_{\alpha} = \frac12 \sum_{\beta=0}^3\sa_{\alpha\beta}\tau_{\beta},
\end{equation}
% or more compactly $\vec{\lambda} = \sa \vec{\tau}$, 
where 
\begin{equation}
\sa=\left(
\begin{array}{cccc}
1 &1 &1   &1   \\
 1 & 1  & -1&-1   \\
1  & -1  & 1 & -1\\
1&-1&-1&1  
\end{array}
\right).
% =\frac12\sa^{-1}.
\label{eq:1}
\end{equation}
We wish to add that one can replace $a$ with $H \otimes H$, with $H$ the
Hadamard matrix, and still diagonalize the same Choi matrix. This is due to the
fact that $a$ corresponds to a permutation of rows of  $H \otimes H$. However,
we chose the aforementioned definition as some later considerations [see
\eref{eq:sigma_property}] cannot be written easily in terms of $H \otimes H$.

% $\sa=2^{-1}H\otimes H$ and $H$ is the Hadamard matrix,
% \cpnote{Decidir si se usa esa $a$ o la otra para ser consistente con lo de los 
% generadores}
% %
% \begin{equation}
%  H=\begin{pmatrix}
%         1 & 1\\
%         1 & -1\\
%     \end{pmatrix}.
%  \label{Hadd_Mat}
% \end{equation}


% \cpnote{Mencionar que con la otra $a$ tambien se puede diagonalizar y la
% dejo por aca por si se nos ofrece}
% 
%  }}}
% \subsubsection{$N$ qubits} % {{{
% 

The same program can be carried out for $N$ qubits. In this case, 
the vectorized Pauli operators are defined as follows, 
\begin{equation}
\kket{\sigma_{\vec{\alpha} }}=\kket{\sigma_{\alpha_1}}\otimes\cdots
\otimes\kket{\sigma_{\alpha_N}}.
\end{equation}
This vectorization must not be confused with the direct vectorization of $\sigma_{\vec \alpha}$, since the tensor product and the vectorization process generally do not commute~\cite{Gilchrist2009}.
% 
% $\janote{aquí hay algo extraño, pues los operadores 
% que utilizamos son la vectorización del producto tensorial de las paulis, no los
% productos tensoriales de las paulis vectorizadas, que es lo que entiendo con la 
% expresión $\kket{\sigma_{\vec{\alpha} }}=\kket{\sigma_{\alpha_1}}\otimes\cdots
% \otimes\kket{\sigma_{\alpha_N}}$}\janote{mi sugerencia es nada más reescribir
% $\kket{\sigma_{\vec{\alpha} }}=\kket{\sigma_{\alpha_1}\otimes\cdots
% \otimes\sigma_{\alpha_N}}$}
% kkkj
The vectors satisfy the orthogonality relation
$\bbrakket{{\sigma_{\vec{\alpha} } }}{{\sigma_{\vec{\alpha}'} }} =  2^N
\delta_{\vec{\alpha}\vec{\alpha}'}$. The matrix representation of the
map corresponding to a Pauli diagonal operation is 
% Analogously to the single qubit case, one can define the $N$-qubits Pauli map $\mathcal{E}_N$
% % 
% \begin{equation}
% \rho \to \rho'=\mathcal{E}_N[\rho ]=\frac{1}{2^N}\sum_{\vec{\alpha}} \tau_{\vec{\alpha}} r_{\vec{\alpha}} \sigma_{\vec{\alpha}},
% % \label{Pauli_Map}
% \end{equation}
% % 
% with $-1\leq \tau_{\vec{\alpha}} \leq 1$. 
% % 
% In this case the density matrix in vector form holds 
% % 
% \begin{equation}
%  \kket{\rho}=\frac{1}{2^N}\sum_{\vec{\alpha}} r_{\vec{\alpha}} \kket{\sigma_{\vec{\alpha} }},
% \end{equation}
% where $\kket{\sigma_{\vec{\alpha} }}=\kket{\sigma_{\alpha_1}}\otimes\dots
% \otimes\kket{\sigma_{\alpha_N}}$, and satisfy the usual orthogonality relation
% $\bbrakket{{\sigma_{\vec{\alpha} } }}{{\sigma_{\vec{\alpha}'} }} = \tr
% \left(\sigma_{\vec{\alpha}}^{\dagger} \sigma_{\vec{\alpha}'}\right) = 2^N
% ~\delta_{\vec{\alpha}\vec{\alpha}'}$. The matrix representation of this map is
% given by
\begin{equation}
\hat{\mathcal{E}}_N = \frac{1}{2^N}\sum_{\vec{\alpha}} \tau_{\vec{\alpha}} \ddyada{\sigma_{\vec{\alpha} }}.
\end{equation}
As in the previous case, the Choi matrix $\mathcal{D}_N$ may be written in
terms of tensor products of Pauli matrices
\begin{equation}
 \mathcal{D}_N 
  = \frac{1}{2^N}\sum_{\vec{\alpha}} \tau_{\vec{\alpha}}
       \bigotimes_{j=1}^N \sigma_{\alpha_j} \otimes \sigma_{\alpha_j}^*.
%  \label{Choi_1}
\end{equation}
This superoperator is again diagonal in the (multi-qubit) Pauli basis, with the 
eigenvalue corresponding to 
$ \kket{\sigma_{\vec{\alpha} }}$ given by 
% 
% After several steps, it is possible to prove that analogously to the single qubit case, $\mathcal{D}_N$ is diagonal in the $N$-qubits Pauli basis
% \begin{equation}
%  \mathcal{D}_N \kket{\sigma_{\vec{\alpha} }}
%    = \lambda_{\vec{\alpha}} \kket{\sigma_{\vec{\alpha} }},
% \end{equation}
% with eigenvalues $\lambda_{\vec{\alpha}}$ given by 
\begin{equation}
 \lambda_{\valpha} = \frac{1}{2^N} \sum_{\vbeta} \san_{\valpha\vbeta}\tau_{\vbeta},
\label{eq:lambda:is:A:tau}
\end{equation}
where 
\begin{equation}\label{eq:san}
 \san=\sa^{\otimes N}
% =\frac{1}{2^N}\left(H\otimes H\right)^{\otimes N}=\left(\frac{H}{\sqrt{2}}\right)^{\otimes 2N}.
\end{equation}
Again, the proofs are provided in appendix \ref{sec:appendix:diagonalization}. We wish 
to add that we could diagonalize $\mathcal{D}_N$ with $H^{\otimes 2N}$ instead of
$a^{\otimes N}$, which might be more convenient for other applications.
% \janote{en una 
% iteración posterior del artículo completo quizás es buena idea enmarcar más esto
% último de la otra a, ya que puede ser de interés en alguna aplicación}
% \cpnote{Mencionar que con la otra $a$ tambien se puede diagonalizar para $N$ 
% qubits.}
% (Proofs in an appendix using binary indices XD)
% }}}



