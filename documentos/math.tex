\section{Mathematical derivation} % {{{

{Vector space, mathematical derivation of properties}

\begin{itemize}
\item Diagonalización de la matriz de Choi de los canales de Pauli, es decir prueba da la $a$
\item Prueba de que la a diagonaliza \afnote{Breve descripción de detalles importantes. Cálculos en un Apéndice?}
\item Canales PCE como espacios vectoriales
\item Regla de $2^k$
\item Tamaño de las diferentes clases que borran un numero fijo de componentes
y tambien nota de qeu son igual entre complementarios: computo del numero de bases dada una dimension de subespacios vectoriales, aqui entra la formula del binomial-q.
\item Intento de identificar elementos, aka hermanistos PCE \cpnote{Esto dependerá de si sale algo interesante,}: De entrada aqui se tiene que mencionar minimo la identificacion del tamaño de las clases.
\item \ddnote{Agrego esto:} A final remarks of the derivation as a tool. Concepts might be complicated, but the tool gives some simple rules to the reader: \textit{tell me what components you want to keep, and the (vector) summation rules will give you the rest of components to make the target operation a CPTP one.}
\end{itemize}

% }}}