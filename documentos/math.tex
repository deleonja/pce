\section{Mathematical considerations} 
\subsection{Diagonalization of Choi matrix} 

% Intro {{{
Prueba de la $\sa$ \todo{Alejo escribe esta parte. Dejelo aca todo y luego si acaso 
pasamos cosas al apendice. Caso general}

This part is devoted to derive the set of conditions a Pauli map need to
satisfy to fulfill the CP condition i.e. all the eigenvalues of the Choi matrix
associated to the channel must be non-negative. First we obtain the set of
inequalities for a single qubit Pauli channel and then we extend the results
for the general case of $N$ qubits. 

% }}}
\subsubsection{Single qubit Pauli channels} % {{{

\afnote{Notación para matrices densidad vectorizadas? $\varrho$ ?}
\janote{Qué tal $\vec\rho\quad$?}
\afnote{Por ahora uso este $\kket{\rho}$}

We now turn to find the matrix form of the map $\mathcal{E}$ in order to construct the Choi matrix and the conditions the set of parameters $\{\tau_{\alpha}\}$ need to satisfy in order to $\mathcal{E}$ be considered a quantum channel. By writing the density matrix in vector form $\kket{\rho}=2^{-1}\sum_{\alpha=0}^{3} r_{\alpha} \kket{\sigma_{\alpha}}$, the matrix form of the map $\mathcal{E}$ may be written as
% 
\begin{equation}
\hat{\mathcal{E}} = \frac{1}{2}\sum_{\alpha=0}^{3} \tau_{\alpha} \ddyada{\sigma_{\alpha}},
% \label{Map_Pauli}
\end{equation}
%
where the kets $\kket{\sigma_{\alpha}}$ correspond to the vectorized elements of the Pauli basis and satisfy the usual relation $ \bbrakket{{\sigma_{\alpha} }}{{\sigma_{\alpha'} }} = \tr \left(\sigma_{\alpha}^{\dagger} \sigma_{\alpha'}\right) = 2 ~\delta_{\alpha\alpha'}$. After some steps it is possible to show that the associated Choi matrix reads
% 
\begin{equation}
 \mathcal{D} = \frac{1}{2}\sum_{\alpha=0}^{3} \tau_{\alpha} \sigma_{\alpha} \otimes \sigma_{\alpha}^*.
%  \label{Choi_1}
\end{equation}
% 
Moreover, observe that $\mathcal{D}$ has the same form as the matrix representation of a random unitary channel, with Kraus operators given by $K_{\alpha}=\sqrt{\tau_{\alpha}/2}\sigma_{\alpha}$ \cite{CHRUSCINSKI20131425}. In this way the Choi matrix $\mathcal{D}$ is diagonal in the Pauli basis (see Appendix for details),
%
\begin{equation}
 \mathcal{D} \kket{\sigma_{\alpha}}= \lambda_{\alpha} \kket{\sigma_{\alpha}},
\end{equation}
% 
with eigenvalues given by 
% 
\begin{equation}
 \lambda_{\alpha} = \sum_{\beta=0}^3\sa_{\alpha\beta}\tau_{\beta},
\end{equation}
%
or more compactly $\vec{\lambda} = \sa~\vec{\tau}$, where $\sa=2^{-1}H\otimes
H$ and $H$ is the Hadamard matrix,
\cpnote{Decidir si se usa esa $a$ o la otra para ser consistente con lo de los 
generadores}
%
\begin{equation}
 H=\begin{pmatrix}
        1 & 1\\
        1 & -1\\
    \end{pmatrix}.
 \label{Hadd_Mat}
\end{equation}


\cpnote{Mencionar que con la otra $a$ tambien se puede diagonalizar y la
dejo por aca por si se nos ofrece}
\begin{equation}
\sa=\left(
\begin{array}{cccc}
1 &1 &1   &1   \\
 1 & 1  & -1&-1   \\
1  & -1  & 1 & -1\\
1&-1&-1&1  
\end{array}
\right)=\frac12\sa^{-1}.
\label{eq:1}
\end{equation}
% 
%  }}}
\subsubsection{$N$ qubits} % {{{
% 
Analogously to the single qubit case, one can define the $N$-qubits Pauli map $\mathcal{E}_N$
% 
\begin{equation}
\rho \to \rho'=\mathcal{E}_N[\rho ]=\frac{1}{2^N}\sum_{\vec{\alpha}} \tau_{\vec{\alpha}} r_{\vec{\alpha}} \sigma_{\vec{\alpha}},
% \label{Pauli_Map}
\end{equation}
% 
with $-1\leq \tau_{\vec{\alpha}} \leq 1$. 
% 
In this case the density matrix in vector form holds 
% 
\begin{equation}
 \kket{\rho}=\frac{1}{2^N}\sum_{\vec{\alpha}} r_{\vec{\alpha}} \kket{\sigma_{\vec{\alpha} }},
\end{equation}
% 
where $\kket{\sigma_{\vec{\alpha} }}=\kket{\sigma_{\alpha_1}}\otimes\dots
\otimes\kket{\sigma_{\alpha_N}}$, and satisfy the usual orthogonality relation
$\bbrakket{{\sigma_{\vec{\alpha} } }}{{\sigma_{\vec{\alpha}'} }} = \tr
\left(\sigma_{\vec{\alpha}}^{\dagger} \sigma_{\vec{\alpha}'}\right) = 2^N
~\delta_{\vec{\alpha}\vec{\alpha}'}$. The matrix representation of this map is
given by
% 
\begin{equation}
\hat{\mathcal{E}}_N = \frac{1}{2^N}\sum_{\vec{\alpha}} \tau_{\vec{\alpha}} \ddyada{\sigma_{\vec{\alpha} }}.
% \label{Map_Pauli}
\end{equation}
%
As in the previous case, the Choi matrix $\mathcal{D}_N$ may be written in terms of tensor products of Pauli matrices
% 
\begin{equation}
 \mathcal{D}_N = \frac{1}{2^N}\sum_{\vec{\alpha}} \tau_{\vec{\alpha}}  \bigotimes_{j=1}^N \sigma_{\alpha_j} \otimes \sigma_{\alpha_j}^*.
%  \label{Choi_1}
\end{equation}
% 
After several steps, it is possible to prove that analogously to the single qubit case, $\mathcal{D}_N$ is diagonal in the $N$-qubits Pauli basis
%
\begin{equation}
 \mathcal{D}_N \kket{\sigma_{\vec{\alpha} }}
   = \lambda_{\vec{\alpha}} \kket{\sigma_{\vec{\alpha} }},
\end{equation}
% 
with eigenvalues $\lambda_{\vec{\alpha}}$ given by 
% 
\begin{equation}
 \lambda_{\vec{\alpha}} = \sum_{\vec{\beta}} \san_{\vec{\alpha}\vec{\beta}}\tau_{\vec{\beta}},
\end{equation}
%
where 
% 
\begin{equation}
 \san=\sa^{\otimes N}
% =\frac{1}{2^N}\left(H\otimes H\right)^{\otimes N}=\left(\frac{H}{\sqrt{2}}\right)^{\otimes 2N}.
\end{equation}
Again, the proofs are provided in appendix \ref{sec:appendix:diagonalization}
\cpnote{Mencionar que con la otra $a$ tambien se puede diagonalizar para $N$ 
qubits.}
% (Proofs in an appendix using binary indices XD)
% }}}



