\documentclass[11pt,dvipsnames]{article} % {{{

\usepackage{geometry}
\geometry{total={170mm,240mm}, left=20mm, top=20mm}

\usepackage[utf8]{inputenc}
\usepackage[english]{babel}
\usepackage{enumerate} 
\usepackage{pgfplots}

% To add graphics
% \usepackage{graphicx}
% Path to figures
\graphicspath{ {../images/} }

\usepackage{subcaption}


\newcommand{\pce}{PCE}



\usepackage{multirow}

\newcommand{\Mod}[1]{\ (\mathrm{mod}\ #1)}

\newtheorem{conj}{Hypothesis}

\newcommand{\fref}[1]{fig.~\ref{#1}}  
\newcommand{\tref}[1]{table~\ref{#1}}
\newcommand{\Fref}[1]{Fig.~\ref{#1}}  
\newcommand{\Tref}[1]{Table~\ref{#1}}

\newcommand{\R}{\mathcal{R}}
\newcommand{\psii}{\psi_i}
\newcommand{\Pk}[1]{\ket{\psi_{#1} }}
\newcommand{\Pb}[1]{\bra{\psi_{#1} }}
\newcommand{\pk}{\ket{\psi}}
\newcommand{\M}{\mathcal{M}^{(N)}}
\newcommand{\E}{\mathcal{E}}
\newcommand{\Ej}[1]{\E_{j_{#1}} }
\newcommand{\id}[1]{\qty(\1-\E_{j_{#1}})}
\newcommand{\Erho}{\mathcal{E}(\rho)}
\newcommand{\1}{\mathds{1}}
\newcommand{\ten}{\otimes}
\newcommand{\h}[1]{\colorbox{Yellow}{#1}}
\newcommand{\hi}{\mathcal{H}}
\newcommand{\txt}[1]{\text{#1}}
\newcommand{\here}{\h{\hspace{15cm}} }
\newcommand{\rhoi}{\dyad{\psii}{\psii}}
\newcommand{\ind}[2]{ {{}^{#1}_{#2}} }
\newcommand{\rc}[1]{r_{#1}}
\newcommand{\pauli}[2]{\sigma_{#1}\otimes\sigma_{#2}}
\newcommand{\Er}{\E_{\mathcal{R}} }
\newcommand{\PCE}[1]{\mathsf{PCE}_\mathsf{#1}}
\newcommand{\ot}{\otimes}

\usepackage{imakeidx}
\makeindex[columns=3, title=Alphabetical Index, intoc]


%\usepackage[]{lineno}  \linenumbers
%\setlength\linenumbersep{3pt}

\usepackage{fancybox}
\usepackage{colortbl}
\usepackage{amsbsy}
\usepackage{physics}
\usepackage{amssymb}
\usepackage{dsfont}

\usepackage[draft,inline,nomargin]{fixme} \fxsetup{theme=color}
\newcommand\dsone{\mathds{1}}
\FXRegisterAuthor{cp}{acp}{\color{blue}CP}
\definecolor{mycolor}{RGB}{200,40,0} \FXRegisterAuthor{ja}{aja}{\color{mycolor}JA}
\FXRegisterAuthor{dd}{ddg}{\color{red}DD}
\FXRegisterAuthor{af}{aaf}{\color{ForestGreen}AF}
\title{Pauli component erasing operations} 

% }}}
\begin{document}
% Titulo y otros {{{
\maketitle
% \tableofcontents
% }}}

\section{Introduccion} % {{{
Several PRA articles go straight into matter. I guess we will write what we have to write
% }}}
\section{Single qubit proyective maps} % {{{
and display single qubit information relevant. 



The single qubit case, describe some of the features

Also, maybe the Krauss operators. 

Also some relation with decoherence. 

% }}}
\section{PCE Maps} % {{{
\begin{itemize}
\item Give a formal definition of the PCE, maybe give some two qubit examples
to ilustrate non-trivial points. 
\item This will serve to bring the derivation a bit more down to earth. 
\item Visualizacion y Ejemplos?
\item Notar lo de los generadores y como sale la estructura jerárquica 
\item Son positivas todas las matrices de densidad que resultan de aplicar un
mapeo no válido? \janote{Creo que sí.  Para 1 qubit aplicar PCEs no CP siempre
te deja dentro de la esfera de Bloch.  Así que estoy casi seguro que lo mismo
sucede para $n$ qubits.}
\item Canales de Ruskai, que la interseccion es no trivial (no estan contenidos en ninguna direccion, pero la interseccion es no vacia)
\item \afnote{Relación con canales citados en \cite{Siudzinska2020}?}
\end{itemize}
% }}}
\section{Mathematical derivation} % {{{

{Vector space, mathematical derivation of properties}

\begin{itemize}
\item Diagonalización de la matriz de Choi de los canales de Pauli, es decir prueba da la $a$
\item Prueba de que la a diagonaliza \afnote{Breve descripción de detalles importantes. Cálculos en un Apéndice?}
\end{itemize}
% }}}
\section{Some results with the vector space thing} % {{{
\begin{itemize}
\item Canales PCE como espacios vectoriales
\item Regla de $2^k$
\item Tamaño de las diferentes clases que borran un numero fijo de componentes
y tambien nota de qeu son igual entre complementarios
\item Intento de identificar elementos, aka hermanistos PCE \cpnote{Esto dependerá de si sale algo interesante,}
\end{itemize}

% }}}
\section{Generators} % {{{
\begin{itemize}
\item Clasificacion de los generadores, es decir que para dos qubits, lo podemos indicar con la acción sobre cada qubit individual
\end{itemize}
% }}}
\section{Krauss operators and decoherence} % {{{

\begin{itemize}
\item Construct the Krauss operators, and relate to decoherence, for PCEG
\item Play with non PCEG PCEs
\end{itemize}

% }}}
\section{The set of decoherence channels} % {{{
\begin{itemize}
\item Aca creo qe pueden ser puntos extremos de ese conjunto de canales
\item El conjunto dentro del conjunto de todos los canales (puntos extremos)
\end{itemize}
% }}}
\section{Qudits} % {{{

\begin{itemize}
\item Discuss what happens with general $d$ level systems and why it does not work
\item 
Linealidad de los eigenvalores de la matriz de Choi en función de las
$\tau_{j_1\ldots,j_n}$ de canales `\textit{component erasing}' de qudits. 
\end{itemize}

Discuss what happens with general $(2^k)^{\otimes n}$ level systems and why it does work
% }}}
\section{Ideas sueltas} % {{{
\subsection{Preguntas abiertas}
\begin{itemize}
\item \afnote{Clasificación de las clases de canales PCE: Entanglement breaking, divisible...}
\end{itemize}
% }}}
\bibliographystyle{unsrt}
\bibliography{references}
\end{document}
