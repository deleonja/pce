\documentclass[letterpaper,aps,pra,nolongbibliography,twocolumn,showpacs,floatfix,10pt]{revtex4-2} % {{{
%\usepackage[notref,notci2te]{showkeys} I
%\usepackage[notref]{showkeys}
%\usepackage{iopams}
%\usepackage{setstack}
\usepackage{url}
\usepackage{amsmath}
\usepackage{amssymb}
\usepackage[utf8]{inputenc}
\usepackage[english]{babel}
\usepackage[T1]{fontenc}
\usepackage{amsmath}
\usepackage{hyperref}
%\usepackage{tikz}
\usepackage{lipsum}
\usepackage{graphicx,helvet}
\usepackage{color}
%\usepackage{bm}                      % added by TG

\usepackage{mathtools}

% \usepackage[inline]{showlabels}

% \usepackage[inner]{showlabels}
%\usepackage{bbm,bm}
\usepackage{soul}
\usepackage{amsfonts}
 \usepackage{amsmath}
\usepackage{lipsum}% http://ctan.org/pkg/lipsum
\definecolor{mygray}{gray}{0.4}
\definecolor{light-blue}{rgb}{0.8,0.85,1}
\graphicspath{{images/}}
\usepackage{amsthm}
%\usepackage{MnSymbol}%
%\usepackage{wasysym}%
%\usepackage{bbm}% bold math
%\usetikzlibrary{decorations.shapes}
\usepackage[draft,inline,nomargin]{fixme} \fxsetup{theme=color}
\newcommand\dsone{\mathds{1}}
\FXRegisterAuthor{cp}{acp}{\color{blue}CP}
\definecolor{mycolor}{RGB}{200,40,0} \FXRegisterAuthor{ja}{aja}{\color{mycolor}JA}
\FXRegisterAuthor{dd}{ddg}{\color{red}DD}
\FXRegisterAuthor{af}{aaf}{\color{magenta}AF}
%\def\>{\rangle} \def\<{\langle}
\renewcommand{\>}{\rangle}
\mathchardef\Re="023C
\mathchardef\Im="023D
\usepackage[mathlines]{lineno}  
%\linenumbers
\setlength\linenumbersep{3pt}%\modulolinenumbers[5]
% \newcommand{\red}{\color{red}}
\newcommand{\<}{\langle}
\newcommand{\mele}[2]{\ensuremath{| #1 \rangle \langle #2 |}}
\newcommand{\prj}[1]{\ensuremath{| #1 \rangle \langle #1 |}}
\newcommand{\mcU}{\mathcal{U}}
\newcommand{\mcO}{\mathcal{O}}
\newcommand{\mcI}{\mathcal{I}}
\newcommand{\mcL}{\mathcal{L}}
\newcommand{\mcB}{\mathcal{B}}
\newcommand{\mcH}{\mathcal{H}}
\newcommand{\mcT}{\mathcal{T}}
\newcommand{\mcE}{\mathcal{E}}
\newcommand{\mcM}{\mathcal{M}}
\newcommand{\mcN}{\mathcal{N}}
\newcommand{\mcA}{\mathcal{A}}
\newcommand{\mcP}{\mathcal{P}}
\newcommand{\id}{\text{id}}
\newtheorem{theorem}{Theorem}
\newtheorem{definition}{Definition}
\newtheorem{proposition}{Proposition}
\newtheorem{corollary}{Corollary}
\newtheorem{lemma}{Lemma}
\newcommand{\mcQ}{\mathcal{Q}}
\newcommand{\mcG}{\mathcal{G}}
\newcommand{\mcF}{\mathcal{F}}
\newcommand{\mcK}{\mathcal{K}}
\newcommand{\mcS}{\mathcal{S}}
\newcommand{\ie}{i.e.}
\newcommand{\fmlong}{fuzzy measurements}
\newcommand{\Fmlong}{Fuzzy measurements}
\newcommand{\ipr}{\mathrm{ipr}}
\newcommand{\rmH}{\mathrm{H}}
\newcommand{\rmq}{\mathrm{q}}
\newcommand{\cut}{\mathrm{cut}}
\newcommand{\rmd}{\mathrm{d}}
\newcommand{\rmi}{\mathrm{i}}
\newcommand{\blue}{\color{blue}}
\newcommand{\mcC}{\mathcal{C}}
\newcommand{\cg}{\mcC}
%%%%CG commands
\newcommand{\Loc}[1]{\ensuremath{\Lambda_#1^\text{L}} }
\newcommand{\NoLoc}[1]{\ensuremath{\Lambda_{#1}^\text{NL}} }
\newcommand{\NoLoct}[1]{\ensuremath{\tilde{\Lambda}_{#1}^\text{NL}} }
\newcommand{\fuzzytwo}[1]{\ensuremath{\Lambda_#1^\text{fuzzy}} }
\newcommand{\cgtwo}[1]{\ensuremath{\Lambda_#1^\text{cg}} }
\newcommand{\tp}{TP}
\newcommand{\hp}{HP}
\newcommand{\Imi}{\imath}
\newcommand{\rmf}{f}
%\newcommand{\eg}{\textit{e.g.} }
%\newcommand{\Eg}{\textit{E.g.} }
\newcommand{\eref}[1]{eq.~(\ref{#1})} 
\newcommand{\sref}[1]{sec.~\ref{#1}}
\newcommand{\fref}[1]{fig.~\ref{#1}}
\newcommand{\tref}[1]{table~\ref{#1}}
\newcommand{\Eref}[1]{Eq.~(\ref{#1})} 
\newcommand{\Sref}[1]{Sec.~\ref{#1}}
\newcommand{\Fref}[1]{Fig.~\ref{#1}}  
\newcommand{\Tref}[1]{Table~\ref{#1}}
\newcommand{\Or}{\mathord{\mathrm{O}} }
\newcommand{\tr}{\mathop{\mathrm{Tr}}\nolimits}
% \newcommand{\comm}[1]{{\color{red} #1 }}
\newcommand{\qv}{QV}
\newcommand{\old}{\infty}
\newcommand{\Mmed}{\mcM^{\<\cdot\>}}
\newcommand{\MBLP}{\mcM^{\text{BLP}} }
\newcommand{\MRHP}{\mcM^{\text{RHP}} }
\newcommand{\gs}{GS}
\newcommand{\cptp}{CPTP}

\newcommand{\ket}[1]{{\vert #1 \rangle}}
\newcommand{\bra}[1]{{\langle #1 \vert}}
\newcommand{\proj}[2]{{\vert #1 \rangle \langle #2 \vert}}
% \newcommand{\cp}{\color{red}   }

%\newcommand{\mimath}{\rmi}
%\newcommand{\into}{\int_0^t \rmd \tau \{\blue [Verbo de la introducción]}int_0^t \rmd \tau'}
%\newcommand{\intoh}{\int_0^t \rmd \tau \int_0^\tau \rmd \tau'}

%\providecommand{\openone}{\leavevmode\hbox{\small1\kern-3.8pt\normalsize1}}

%\newcommand{\eps}{\varepsilon} \newcommand{\trc}{\tr_{\rm c}}
%\newcommand{\tre}{\tr_{\rm e}} \newcommand{\mcH}{\mathcal{H}}
%\newcommand{\mcO}{\mathcal{O}} \newcommand{\mcU}{\mathcal{U}}

\newcommand{\unam}{Universidad Nacional Aut\'onoma de M\'exico, Ciudad de M\'exico 01000, M\'exico}
\newcommand{\icf}{Instituto de Ciencias F\'{\i}sicas, \unam}
\newcommand{\ifunam}{Instituto de F\'{\i}sica, \unam}
\newcommand{\fisguadalajara}{Departamento de F\'isica, Universidad de Guadalajara, Guadalajara, Jal\'isco, M\'exico}
\newcommand{\red}{\color{red}}
\newcommand{\affjose}{afiliacion de JA}
\newcommand{\affalejandro}{afiliacion de Alejo}
% }}}
\begin{document}
% Title, authors, etc {{{
\title{Pauli component erasing operations} 
\author{Jose Alfredo} \affiliation{\affjose}
\author{Alejandro Fonseca} \affiliation{\affalejandro}
\author{François Leyvraz} \affiliation{\icf}
\author{David Davalos} \affiliation{\ifunam}
\author{Carlos Pineda} \email{carlospgmat03@gmail.com} \affiliation{\ifunam}
\begin{abstract} % {{{
We study a generalization of decoherence maps to maps erasing components of the generalized Bloch vector using the Pauli product basis 
\end{abstract} % }}}
%%
%\pacs{03.65.Yz, 03.65.Ta, 05.45.Mt}
 
\maketitle
% }}}
\newcommand{\sa}{a}
\newcommand{\san}{A}
\section{Propuesta de notacion} % {{{
\begin{itemize}
\item Single qubit matrix to diagonalize $\sa$: \verb!\sa!
\item Many qubit matrix to diagonalize $\san$: \verb!\san!
\end{itemize}
% }}}
\section{Introduction} % {{{
Several PRA articles go straight into matter. I guess we will write what we have to write
% }}}
\ddnote{Aqui vendria la breve introduccion a los canales cuanticos?}
\section{PCE Maps} % {{{
\subsection{Quantum channels}
\subsection{Single qubit projective maps} % {{{
\ddnote{Sugiero otro nombre para esta seccion, pues los PCE no son los unicos mapas proyectivos, hay mapas proyectivos no unitales}
\begin{itemize}
\item The single qubit case, describe some of the features.
\item Also, maybe the Kraus operators.
\item Also some relation with decoherence. 
\end{itemize}
% }}}
\subsection{Pauli component erasing maps} % {{{
\begin{itemize}
\item Give a formal definition of the PCE, maybe give some two qubit examples
to illustrate non-trivial points. 
\item Canales de Ruskai Pauli maps, que la interseccion es no trivial (no estan contenidos en ninguna direccion, pero la interseccion es no vacia)
\item \afnote{Relación con canales citados en \cite{Siudzinska2020}?}
\end{itemize}
% }}}
\begin{itemize}
\item Give a formal definition of the PCE, maybe give some two qubit examples
to illustrate non-trivial points. 
\item This will serve to bring the derivation a bit more down to earth. 
\item Visualizacion y ejemplos?
\item Notar lo de los generadores y como sale la estructura jerárquica 
\item Son positivas todas las matrices de densidad que resultan de aplicar un
mapeo no válido? \janote{Creo que sí.  Para 1 qubit aplicar PCEs no CP siempre
te deja dentro de la esfera de Bloch.  Así que estoy casi seguro que lo mismo
sucede para $n$ qubits.}
\end{itemize}
% }}}
\section{Mathematical derivation} % {{{

{Vector space, mathematical derivation of properties}

\begin{itemize}
\item Diagonalización de la matriz de Choi de los canales de Pauli, es decir prueba da la $a$
\item Prueba de que la a diagonaliza \afnote{Breve descripción de detalles importantes. Cálculos en un Apéndice?}
\item Canales PCE como espacios vectoriales
\item Regla de $2^k$
\item Tamaño de las diferentes clases que borran un numero fijo de componentes
y tambien nota de qeu son igual entre complementarios: computo del numero de bases dada una dimension de subespacios vectoriales, aqui entra la formula del binomial-q.
\item Intento de identificar elementos, aka hermanistos PCE \cpnote{Esto dependerá de si sale algo interesante,}: De entrada aqui se tiene que mencionar minimo la identificacion del tamaño de las clases.
\item \ddnote{Agrego esto:} A final remarks of the derivation as a tool. Concepts might be complicated, but the tool gives some simple rules to the reader: \textit{tell me what components you want to keep, and the (vector) summation rules will give you the rest of components to make the target operation a CPTP one.}
\end{itemize}

% }}}
\section{Generators} % {{{
\begin{itemize}
\item Clasificacion de los generadores, es decir que para dos qubits, lo podemos indicar con la acción sobre cada qubit individual
\item Quizá mostrar algunas propiedades de simetrías. 
\end{itemize}
% }}}
\section{Kraus operators and decoherence} % {{{

\begin{itemize}
\item Construct the Kraus operators, and relate to decoherence, for PCEG
\item Play with non PCEG PCEs
\item Dependiendo de si sale algo bonito, podemos poner algun modelo en particular para todos los canales (sistema central + ambiente). Seria de buscar dilataciones sencillas y razonables por lo menos a los canales generadores.
\end{itemize}

% }}}
\section{The set of decoherence channels} % {{{
\begin{itemize}
\item Aca creo qe pueden ser puntos extremos de ese conjunto de canales
\item El conjunto dentro del conjunto de todos los canales (puntos extremos)
\end{itemize}
% }}}
\section{Qudits} % {{{

\begin{itemize}
\item Discuss what happens with general $d$ level systems and why it does not work
\item 
Linealidad de los eigenvalores de la matriz de Choi en función de las
$\tau_{j_1\ldots,j_n}$ de canales `\textit{component erasing}' de qudits. 
\end{itemize}

Discuss what happens with general $(2^k)^{\otimes n}$ level systems and why it does work
% }}}
\section{Ideas sueltas} % {{{
\subsection{Preguntas abiertas}
\begin{itemize}
\item \afnote{Clasificación de las clases de canales PCE: Entanglement breaking, divisible...}
\end{itemize}
% }}}
\par
\section{Acknowledgments}
\bibliographystyle{unsrt}
\bibliography{references}
\end{document}
