\documentclass[11pt,dvipsnames]{article} % {{{

\usepackage{geometry}
\geometry{total={170mm,240mm}, left=20mm, top=20mm}

\usepackage[utf8]{inputenc}
\usepackage[english]{babel}
\usepackage{enumerate} 
\usepackage{pgfplots}

% To add graphics
% \usepackage{graphicx}
% Path to figures
\graphicspath{ {../images/} }

\usepackage{subcaption}


\newcommand{\pce}{PCE}



\usepackage{multirow}

\newcommand{\Mod}[1]{\ (\mathrm{mod}\ #1)}

\newtheorem{conj}{Hypothesis}

\newcommand{\fref}[1]{fig.~\ref{#1}}  
\newcommand{\tref}[1]{table~\ref{#1}}
\newcommand{\Fref}[1]{Fig.~\ref{#1}}  
\newcommand{\Tref}[1]{Table~\ref{#1}}

\newcommand{\R}{\mathcal{R}}
\newcommand{\psii}{\psi_i}
\newcommand{\Pk}[1]{\ket{\psi_{#1} }}
\newcommand{\Pb}[1]{\bra{\psi_{#1} }}
\newcommand{\pk}{\ket{\psi}}
\newcommand{\M}{\mathcal{M}^{(N)}}
\newcommand{\E}{\mathcal{E}}
\newcommand{\Ej}[1]{\E_{j_{#1}} }
\newcommand{\id}[1]{\qty(\1-\E_{j_{#1}})}
\newcommand{\Erho}{\mathcal{E}(\rho)}
\newcommand{\1}{\mathds{1}}
\newcommand{\ten}{\otimes}
\newcommand{\h}[1]{\colorbox{Yellow}{#1}}
\newcommand{\hi}{\mathcal{H}}
\newcommand{\txt}[1]{\text{#1}}
\newcommand{\here}{\h{\hspace{15cm}} }
\newcommand{\rhoi}{\dyad{\psii}{\psii}}
\newcommand{\ind}[2]{ {{}^{#1}_{#2}} }
\newcommand{\rc}[1]{r_{#1}}
\newcommand{\pauli}[2]{\sigma_{#1}\otimes\sigma_{#2}}
\newcommand{\Er}{\E_{\mathcal{R}} }
\newcommand{\PCE}[1]{\mathsf{PCE}_\mathsf{#1}}
\newcommand{\ot}{\otimes}

\usepackage{imakeidx}
\makeindex[columns=3, title=Alphabetical Index, intoc]


%\usepackage[]{lineno}  \linenumbers
%\setlength\linenumbersep{3pt}

\usepackage{fancybox}
\usepackage{colortbl}
\usepackage{amsbsy}
\usepackage{physics}
\usepackage{amssymb}
\usepackage{dsfont}

\usepackage[draft,inline,nomargin]{fixme} \fxsetup{theme=color}
\newcommand\dsone{\mathds{1}}
\FXRegisterAuthor{cp}{acp}{\color{blue}CP}
\definecolor{mycolor}{RGB}{200,40,0} \FXRegisterAuthor{ja}{aja}{\color{mycolor}JA}
\FXRegisterAuthor{dd}{ddg}{\color{red}DD}

\title{Pauli component erasing operations} 

% }}}
\begin{document}
% Titulo y otros {{{
\maketitle
% \tableofcontents
% }}}

% \section{Krauss}

In order to study the Krauss operators corresponding to PCE's, it is enough to study 
the ones corresponding to generators, that is, the ones that leave 
have of the components invariant. 

As a historical fact, the Krauss operators corresponding to one single qubit PCEs
are 
\begin{equation}
\{\openone/\sqrt2, \sigma_i/\sqrt2\}
\label{}
\end{equation}
(corresponding to the operation that leaves the component $\sigma_i$ invariant), 
see Eq. 8.95 of Chuang's book.  Visual inspection of the Krauss operators 
of two qubit PCEG (PCE generators) lead to the ansatz that 
the appropriate Krauss operators are 
\begin{equation}
\left\{\openone/\sqrt2, \left(\bigotimes_k \sigma^{(k)}_{i_k}\right)/\sqrt2\right\}. 
\label{}
\end{equation}

We shall first show that the aforementioned Kraus operators indeed produce a PCE. 
Notice that 
\begin{equation}
\sigma_i \sigma_j \sigma_i =a^i_{j} \sigma_j,
\label{}
\end{equation}
with $a$ the matrix that appears in 
\eref{eq:eigvals-PCE}. Next, consider the action of a channel with Krauss representation 
\eref{} on an $n$ qubit system with Bloch sphere representation \eref{eq:nqubit:bloch},



Counting, mostrar qe el numero de bichos que aparecen ahi coincide con el numero de generadores. 

Show that they are unique 

% \section{Eigenvalues of PCE operations' Choi matrix} % {{{
% Intro {{{
Choi matrix's eigenvalues of a 1-qubit Pauli Channel are
\begin{align}
\lambda_{\mu}=1+\sum_{j=1}^3a_j^{\mu}\tau_j
\hspace{35pt}
\mu=0,1,2,3;
\hspace{35pt}
a_j^{\mu}=\left\{ \begin{array}{rl}
             1, & \mu=j \lor \mu=0\\
             -1, & \mu\neq j
             \end{array}
   \right..
\end{align}
Therefore, the inequalities needed to be sattisfied for a 1-qubit 
unital Pauli channel in order to be CP may be expressed in the 
following compact expression:
\begin{equation}
\sum_{j=0}^3a_j^{\mu}\tau_j\geq -1,
\hspace{35pt}
\mu=1,2,3;
\hspace{35pt}
a_j^{\mu}=\left\{ \begin{array}{rl}
             1, & \mu=j\\
             -1, & \mu\neq j
             \end{array}
   \right..
\end{equation}

Now, Choi matrix's eigenvalues of an n-qubit Pauli Channel are
\begin{align}\label{eq:eigvals-PCE}
\lambda_{\mu_1,\ldots,\mu_n}=1+
\sum_{\underset{(j_1,\ldots,j_n\neq0)}{j_1,\ldots,j_n=0}}^3
a_{j_1}^{\mu_1}\ldots a_{j_n}^{\mu_n}\tau_{j_1,\ldots,j_n}
\hspace{35pt}
\mu_l&=0,1,2,3,\nonumber\\
a_{j_l}^{\mu_l}&=\left\{ \begin{array}{rl}
             1, & \mu_l=j_l \lor j_l=0\lor \mu_1,\ldots,\mu_n=0\\
             -1, & \mu_l\neq j_l
             \end{array}
   \right..
\end{align}

Therefore, the inequalities needed to be sattisfied for the CP are
\begin{equation}\label{eq:inequalities-PCE}
\sum_{\underset{(j_1,\ldots,j_n\neq0)}{j_1,\ldots,j_n=0}}^3
a_{j_1}^{\mu_1}\ldots a_{j_n}^{\mu_n}\tau_{j_1,\ldots,j_n}
\geq
-1,
\hspace{35pt}
\mu_l=0,1,2,3;
\hspace{35pt}
a_{j_l}^{\mu_l}=\left\{ \begin{array}{rl}
             1, & \mu_l=j_l \lor j_l=0\\
             -1, & \mu_l\neq j_l
             \end{array}
   \right.,
\end{equation}
where we used that $\lambda_{0,\ldots,0}$ is always non-negative.
% }}}
\subsection{Another observation from this} % {{{
Choi's matrix of 2 qubits is diagonal in Pauli product basis
(Kraus operators). Does this follow in the general case???? 
% }}}
% }}}
\section{A general characterization of PCE quantum channels} % {{{
\subsubsection*{To avoid confusion} % {{{
\cpnote{Igual el canal totalmente despolarizante y el estado totalmente mixto 
tendrían el mismo diagrama. Quiza el borde se puede ser de diferente color
}
\janote{Me gustaría discutir antes si igual vale la pena hacer la distinción.
Dado que al final la distinción sólo es una ayuda que se me ocurrió para 
explicar gráficamente el procedimiento de la concatenación.}
The figures we've been using (columns, boards, cubes) may represent
both PCE operations (dynamics) and density matrices (kinematics). We
will make a distinction between both representations with colors.
B$\&$W represent PCE operations, and colored figures represent 
density matrices in Pauli products basis.
\begin{figure}
\begin{subfigure}{.5\textwidth}
\centering
\includegraphics[width=3cm]{PCEoperation}
\caption{PCE operation that erases all Pauli components except 
for $r_{0,0}$, $r_{1,1}$, $r_{1,3}$, and $r_{0,3}$.}
\end{subfigure}
\begin{subfigure}{.5\textwidth}
\centering
\includegraphics[width=3cm]{state}
\caption{Density matrix in Pauli basis with all components equal
to zero except for $r_{0,0}$, $r_{1,1}$, $r_{1,3}$, and $r_{0,3}$.}
\end{subfigure}
\end{figure}
% }}}

\subsection*{Hypothesis} % {{{
Let us denote $\PCE{n}$ the set of all PCE quantum channels for 
$n$ qubits.
\begin{conj}{}
There exists a set $\Gamma_n\subset\PCE{n}$ that is sufficient (and 
necessary?) to generate the rest of elements in $\PCE{n}$ as 
concatenations of different elements in $\Gamma_n$.
\end{conj}

Let us consider $\E_j\in\Gamma_n$ $(j=1,\ldots,4^{n}-1)$,
and $\Lambda\in\PCE{n}$. All $\Lambda$ are a composition of 
$\E_j$,
\begin{equation}\label{eq:PCE-characterization}
\underbrace{\E_{j_{2n}}\ldots\E_{j_1}}_{k\text{ elements}}=\Lambda,
\hspace{20pt}
j_l\neq j_{l+1},
\hspace{20pt}
k=1,\ldots,2n.
\end{equation}
% }}}
\subsection*{A way to find $\Gamma$} % {{{
PCE quantum channels in $\Gamma_n$ may be found as the quantum 
channels defined by the the coefficients $a_j^{\mu}$ of the $\tau_i$
in the eigenvalues expression of Choi matrix, with $a^{k}_{k}=0$ 
(instead of -1). Let us show the 2-qubits case as an example to illustrate this. 
The eigenvalues of Choi matrix of 2-qubits Pauli Channel are
\begin{align}
\lambda_{1}&=
 \tau _{0,0}+\tau _{0,1}-\tau _{0,2}-\tau _{0,3}+\tau _{1,0}+\tau _{1,1}-\tau _{1,2}-\tau
   _{1,3}-\tau _{2,0}-\tau _{2,1}+\tau _{2,2}+\tau _{2,3}-\tau _{3,0}-\tau _{3,1}+\tau
   _{3,2}+\tau _{3,3} \nonumber \\
\lambda_{2}&=   
 \tau _{0,0}-\tau _{0,1}-\tau _{0,2}+\tau _{0,3}+\tau _{1,0}-\tau _{1,1}-\tau _{1,2}+\tau
   _{1,3}-\tau _{2,0}+\tau _{2,1}+\tau _{2,2}-\tau _{2,3}-\tau _{3,0}+\tau _{3,1}+\tau
   _{3,2}-\tau _{3,3} \nonumber \\
\lambda_{3}&=
 \tau _{0,0}-\tau _{0,1}+\tau _{0,2}-\tau _{0,3}+\tau _{1,0}-\tau _{1,1}+\tau _{1,2}-\tau
   _{1,3}-\tau _{2,0}+\tau _{2,1}-\tau _{2,2}+\tau _{2,3}-\tau _{3,0}+\tau _{3,1}-\tau
   _{3,2}+\tau _{3,3} \nonumber \\
\lambda_{4}&=
 \tau _{0,0}+\tau _{0,1}+\tau _{0,2}+\tau _{0,3}+\tau _{1,0}+\tau _{1,1}+\tau _{1,2}+\tau
   _{1,3}-\tau _{2,0}-\tau _{2,1}-\tau _{2,2}-\tau _{2,3}-\tau _{3,0}-\tau _{3,1}-\tau
   _{3,2}-\tau _{3,3}\nonumber \\
\lambda_{5}&=
 \tau _{0,0}+\tau _{0,1}-\tau _{0,2}-\tau _{0,3}-\tau _{1,0}-\tau _{1,1}+\tau _{1,2}+\tau
   _{1,3}-\tau _{2,0}-\tau _{2,1}+\tau _{2,2}+\tau _{2,3}+\tau _{3,0}+\tau _{3,1}-\tau
   _{3,2}-\tau _{3,3}\nonumber \\
\lambda_{6}&=
 \tau _{0,0}-\tau _{0,1}-\tau _{0,2}+\tau _{0,3}-\tau _{1,0}+\tau _{1,1}+\tau _{1,2}-\tau
   _{1,3}-\tau _{2,0}+\tau _{2,1}+\tau _{2,2}-\tau _{2,3}+\tau _{3,0}-\tau _{3,1}-\tau
   _{3,2}+\tau _{3,3}\nonumber \\
\lambda_{7}&=
 \tau _{0,0}-\tau _{0,1}+\tau _{0,2}-\tau _{0,3}-\tau _{1,0}+\tau _{1,1}-\tau _{1,2}+\tau
   _{1,3}-\tau _{2,0}+\tau _{2,1}-\tau _{2,2}+\tau _{2,3}+\tau _{3,0}-\tau _{3,1}+\tau
   _{3,2}-\tau _{3,3}\nonumber \\
\lambda_{8}&=   
 \tau _{0,0}+\tau _{0,1}+\tau _{0,2}+\tau _{0,3}-\tau _{1,0}-\tau _{1,1}-\tau _{1,2}-\tau
   _{1,3}-\tau _{2,0}-\tau _{2,1}-\tau _{2,2}-\tau _{2,3}+\tau _{3,0}+\tau _{3,1}+\tau
   _{3,2}+\tau _{3,3}\nonumber \\
\lambda_{9}&=   
 \tau _{0,0}+\tau _{0,1}-\tau _{0,2}-\tau _{0,3}-\tau _{1,0}-\tau _{1,1}+\tau _{1,2}+\tau
   _{1,3}+\tau _{2,0}+\tau _{2,1}-\tau _{2,2}-\tau _{2,3}-\tau _{3,0}-\tau _{3,1}+\tau
   _{3,2}+\tau _{3,3}\nonumber \\
\lambda_{10}&=   
 \tau _{0,0}-\tau _{0,1}-\tau _{0,2}+\tau _{0,3}-\tau _{1,0}+\tau _{1,1}+\tau _{1,2}-\tau
   _{1,3}+\tau _{2,0}-\tau _{2,1}-\tau _{2,2}+\tau _{2,3}-\tau _{3,0}+\tau _{3,1}+\tau
   _{3,2}-\tau _{3,3}\nonumber \\
\lambda_{11}&=   
 \tau _{0,0}-\tau _{0,1}+\tau _{0,2}-\tau _{0,3}-\tau _{1,0}+\tau _{1,1}-\tau _{1,2}+\tau
   _{1,3}+\tau _{2,0}-\tau _{2,1}+\tau _{2,2}-\tau _{2,3}-\tau _{3,0}+\tau _{3,1}-\tau
   _{3,2}+\tau _{3,3}\nonumber \\
\lambda_{12}&=   
 \tau _{0,0}+\tau _{0,1}+\tau _{0,2}+\tau _{0,3}-\tau _{1,0}-\tau _{1,1}-\tau _{1,2}-\tau
   _{1,3}+\tau _{2,0}+\tau _{2,1}+\tau _{2,2}+\tau _{2,3}-\tau _{3,0}-\tau _{3,1}-\tau
   _{3,2}-\tau _{3,3}\nonumber \\
\lambda_{13}&=   
 \tau _{0,0}+\tau _{0,1}-\tau _{0,2}-\tau _{0,3}+\tau _{1,0}+\tau _{1,1}-\tau _{1,2}-\tau
   _{1,3}+\tau _{2,0}+\tau _{2,1}-\tau _{2,2}-\tau _{2,3}+\tau _{3,0}+\tau _{3,1}-\tau
   _{3,2}-\tau _{3,3}\nonumber \\
\lambda_{14}&=   
 \tau _{0,0}-\tau _{0,1}-\tau _{0,2}+\tau _{0,3}+\tau _{1,0}-\tau _{1,1}-\tau _{1,2}+\tau
   _{1,3}+\tau _{2,0}-\tau _{2,1}-\tau _{2,2}+\tau _{2,3}+\tau _{3,0}-\tau _{3,1}-\tau
   _{3,2}+\tau _{3,3}\nonumber \\
\lambda_{15}&=   
 \tau _{0,0}-\tau _{0,1}+\tau _{0,2}-\tau _{0,3}+\tau _{1,0}-\tau _{1,1}+\tau _{1,2}-\tau
   _{1,3}+\tau _{2,0}-\tau _{2,1}+\tau _{2,2}-\tau _{2,3}+\tau _{3,0}-\tau _{3,1}+\tau
   _{3,2}-\tau _{3,3}\nonumber \\
\lambda_{16}&=  
 \tau _{0,0}+\tau _{0,1}+\tau _{0,2}+\tau _{0,3}+\tau _{1,0}+\tau _{1,1}+\tau _{1,2}+\tau
   _{1,3}+\tau _{2,0}+\tau _{2,1}+\tau _{2,2}+\tau _{2,3}+\tau _{3,0}+\tau _{3,1}+\tau
   _{3,2}+\tau _{3,3}. 
\end{align}

Now, taking all $\tau_{i,j}$ with coefficients $-1$ equal to zero 
one is led to 16 elements of $\PCE{2}$ (one for each eigenvalue): \newline

\includegraphics[width=\textwidth]{2Q-fundamentals}

The first 15 elements are the elements of $\Gamma_2$. 

For the sake
of completeness let us show how to construct all elements in $\PCE{2}$
from $\Gamma_2$. Recall that $\PCE{2}$ can be ordered in 
equivalence classes with PCE quantum channels that are 
connected via particle swaps and local permutations 
of basis.

\begin{itemize}
\item \textbf{8 components:} 
\begin{itemize}
\item C$_1^8$: $\E_{13}$\newline
\begin{tabular}{m{2cm} m{2cm} m{2cm}}
\includegraphics[width=2.2cm]{id}  
& \hspace{0.8cm}$\longrightarrow$ 
& \includegraphics[width=2.2cm]{C81} \\ 
 & \includegraphics[width=2.2cm]{ruleC81} & 
\end{tabular} 

\item C$_2^8$: $\E_1$\newline
\begin{tabular}{m{2cm} m{2cm} m{2cm}}
\includegraphics[width=2.2cm]{id}  
& \hspace{0.8cm}$\longrightarrow$ 
& \includegraphics[width=2.2cm]{8comp}\\ 
 & \includegraphics[width=2.2cm]{16To8} &  
\end{tabular} 
\end{itemize}

\item \textbf{4 components:} \newline
\begin{itemize}
\item C$_1^4$: $\E_{15}\E_{13}$\newline
\begin{tabular}{m{2cm} m{2cm} m{2cm} m{2cm} m{2cm}}
\includegraphics[width=2.2cm]{id}  
& \hspace{0.8cm}$\longrightarrow$ 
& \includegraphics[width=2.2cm]{C81} 
& \hspace{0.8cm}$\longrightarrow$ 
& \includegraphics[width=2.2cm]{C41}\\ 
 & \includegraphics[width=2.2cm]{ruleC81} &  
 & \includegraphics[width=2.2cm]{ruleC81_2} &\\ 
\end{tabular} 

\item C$_2^4$: $\E_{13}\E_1$\newline
\begin{tabular}{m{2cm} m{2cm} m{2cm} m{2cm} m{2cm}}
\includegraphics[width=2.2cm]{id}  
& \hspace{0.8cm}$\longrightarrow$ 
& \includegraphics[width=2.2cm]{8comp} 
& \hspace{0.8cm}$\longrightarrow$ 
& \includegraphics[width=2.2cm]{C42}\\ 
 & \includegraphics[width=2.2cm]{16To8} &  
 & \includegraphics[width=2.2cm]{ruleC81} &\\ 
\end{tabular} 

\pagebreak
\item C$_3^4$: $\E_{15}\E_{1}$\newline
\begin{tabular}{m{2cm} m{2cm} m{2cm} m{2cm} m{2cm}}
\includegraphics[width=2.2cm]{id}  
& \hspace{0.8cm}$\longrightarrow$ 
& \includegraphics[width=2.2cm]{8comp} 
& \hspace{0.8cm}$\longrightarrow$ 
& \includegraphics[width=2.2cm]{C43}\\ 
 & \includegraphics[width=2.2cm]{16To8} &  
 & \includegraphics[width=2.2cm]{ruleC81_2} &\\ 
\end{tabular} 

\item C$_4^4$: $\E_7\E_1$\newline
\begin{tabular}{m{2cm} m{2cm} m{2cm} m{2cm} m{2cm}}
\includegraphics[width=2.2cm]{id}  
& \hspace{0.8cm}$\longrightarrow$ 
& \includegraphics[width=2.2cm]{8comp} 
& \hspace{0.8cm}$\longrightarrow$ 
& \includegraphics[width=2.2cm]{4comp}\\ 
 & \includegraphics[width=2.2cm]{16To8} &  
 & \includegraphics[width=2.2cm]{8To4} &\\ 
\end{tabular} 
\end{itemize}


\item \textbf{2 components:}\newline
\begin{itemize}
\item C$_1^2$: $\E_{15}\E_{14}\E_1$\newline
\begin{tabular}{m{2cm} m{2cm} m{2cm} m{2cm} m{2cm} m{2cm} m{2cm}}
\includegraphics[width=2.2cm]{id}  
& \hspace{0.8cm}$\longrightarrow$ 
& \includegraphics[width=2.2cm]{8comp} 
& \hspace{0.8cm}$\longrightarrow$ 
& \includegraphics[width=2.2cm]{C43}
& \hspace{0.8cm}$\longrightarrow$ 
& \includegraphics[width=2.2cm]{C21}\\ 
 & \includegraphics[width=2.2cm]{16To8} &  
 & \includegraphics[width=2.2cm]{semefue} &
 & \includegraphics[width=2.2cm]{ruleC81_2} &\\ 
\end{tabular} 

\item C$_2^2$: $\E_6\E_7\E_1$\newline
\begin{tabular}{m{2cm} m{2cm} m{2cm} m{2cm} m{2cm} m{2cm} m{2cm}}
\includegraphics[width=2.2cm]{id}  
& \hspace{0.8cm}$\longrightarrow$ 
& \includegraphics[width=2.2cm]{8comp} 
& \hspace{0.8cm}$\longrightarrow$ 
& \includegraphics[width=2.2cm]{4comp}
& \hspace{0.8cm}$\longrightarrow$ 
& \includegraphics[width=2.2cm]{C22}\\ 
 & \includegraphics[width=2.2cm]{16To8} &  
 & \includegraphics[width=2.2cm]{8To4} &
 & \includegraphics[width=2.2cm]{otroTambien}\\ 
\end{tabular}
\end{itemize}


\pagebreak
\item \textbf{1 component:} $\E_9\E_6\E_7\E_1$\newline
\begin{tabular}{m{2cm} m{2cm} m{2cm}}
\includegraphics[width=2.2cm]{C22}
& \hspace{0.8cm}$\longrightarrow$ 
& \includegraphics[width=2.2cm]{depolarizing} \\ 
 & \includegraphics[width=2.2cm]{otro} &  \\ 
\end{tabular}
\end{itemize}

\begin{conj}
Only $n$ elements in $\Gamma_n$ are sufficient to generate the 
remaining elements.
\end{conj}
% }}}
\subsection{Numerical Results} % {{{
Hypothesis 1 has strong numerical support. The search for PCE quantum
channels has been done in two different ways:
\begin{enumerate}
\item Using the inequalities in \eqref{eq:inequalities-PCE} to test CP of 
all PCE operations (even the ones that do not follow the power-of-2 rule).
\item Using hypoteshis 1.
\end{enumerate}
Both methods have analyzed the cases of 1, 2, 3 and partially 4 qubits. 
Both of them get the same results. 

\begin{figure}
  \centering
  \includegraphics[width=.6\textwidth]{cuadro}
\end{figure}
% }}}
% }}}
\section{Things to do and questions to answer} % {{{
\begin{enumerate}
\item This characterization may explain the power-of-2 rule. 
It is quite obvious (in the figures representation) that the concatenation
always erases half of the number of non-zero components. A proof 
for that is missing. \janote{This is well explained with the vector spaces.}
\item Analytic proof for \eqref{eq:eigvals-PCE}. \janote{Alejo's proof.} 
\item Analytic proof that \eqref{eq:PCE-characterization} is necessary 
and sufficient to find all elements in $\PCE{n}$. \janote{Francois' vector spaces.}
\item Analytic proof that taking $-1\to 0$ for $a_k^k$ leads to CP. \janote{Francois'
general procedure to find PCE channels.}
I suggest using particle swaps and local basis permutation in order 
to reduce the problem to $n$ inequalities.
\item Find a way to count PCE channels. \janote{Also from Francois' vector spaces.}
\end{enumerate}
% }}}

\section{Number of PCE channels by the ratio of components 
left invariant}
%\begin{itemize}
%\item 1$/$2 of total components invariant:
%\begin{align}
%4^n-1
%\end{align}
%
%\item 1$/$4 of total components invariant:
%\begin{align}
%\frac{\binom{4^n-1}{2}}{3},
%\end{align}
%
%\item 1$/$8 of total components invariant:
%\begin{align}
%\frac{\binom{4^n-1}{3}-\binom{4^n-1}{2}/3}{28}
%\end{align}
%
%\item 1$/$16 of total components invariant:
%\begin{align}
%\frac{\binom{4^n-1}{4}-35\qty(\frac{\binom{4^n-1}{3}-\binom{4^n-1}{2}/3}{28})}{840}
%\end{align}
%\end{itemize}
%
%\begin{table}[h!]
%\tiny
%\begin{tabular}{llllccccccclll}
%{\color[HTML]{FF0000} \textbf{1}} &                       &                           &                                & \multicolumn{1}{l}{} & \multicolumn{1}{l}{}               & \cellcolor[HTML]{FCE5CD}\textbf{1}   & \textbf{3}                             & \textbf{1}                              & \multicolumn{1}{l}{}                    & \multicolumn{1}{l}{}                &                                                       &                                    &                                \\
%{\color[HTML]{FF0000} \textbf{2}} &                       &                           &                                & \multicolumn{1}{l}{} & \cellcolor[HTML]{FCE5CD}\textbf{1} & \cellcolor[HTML]{CFE2F3}\textbf{15}  & \cellcolor[HTML]{FCE5CD}\textbf{35}    & \textbf{15}                             & \textbf{1}                              & \multicolumn{1}{l}{}                &                                                       &                                    &                                \\
%{\color[HTML]{FF0000} \textbf{3}} &                       &                           &                                & \textbf{1}           & \textbf{63}                        & \cellcolor[HTML]{FCE5CD}\textbf{651} & \cellcolor[HTML]{CFE2F3}\textbf{1,395} & \cellcolor[HTML]{FCE5CD}\textbf{651}    & \textbf{63}                             & \textbf{1}                          &                                                       &                                    &                                \\
%{\color[HTML]{FF0000} \textbf{4}} &                       &                           & \multicolumn{1}{c}{\textbf{1}} & \textbf{255}         & \textbf{10,795}                    & 97,155                               & \cellcolor[HTML]{FCE5CD}200,787        & \cellcolor[HTML]{CFE2F3}\textbf{97,155} & \cellcolor[HTML]{FCE5CD}\textbf{10,795} & \textbf{255}                        & \multicolumn{1}{c}{\textbf{1}}                        &                                    &                                \\
%{\color[HTML]{FF0000} \textbf{5}} &                       & \multicolumn{1}{c}{1}     & \multicolumn{1}{c}{1,023}      & 174,251              & 6,347,715                          & 53,743,987                           & \multicolumn{1}{l}{}                   & \cellcolor[HTML]{FCE5CD}53,743,987      & \cellcolor[HTML]{CFE2F3}6,347,715       & \cellcolor[HTML]{FCE5CD}174,251     & \multicolumn{1}{c}{\textbf{1,023}}                    & \multicolumn{1}{c}{\textbf{1}}     &                                \\
%{\color[HTML]{FF0000} \textbf{6}} & \multicolumn{1}{c}{1} & \multicolumn{1}{c}{4,095} & \multicolumn{1}{c}{2,794,155}  & 408,345,795          & 13,910,980,083                     & \multicolumn{1}{l}{}                 & \multicolumn{1}{l}{}                   & \multicolumn{1}{l}{}                    & \cellcolor[HTML]{FCE5CD}13,910,980,083  & \cellcolor[HTML]{CFE2F3}408,345,795 & \multicolumn{1}{c}{\cellcolor[HTML]{FCE5CD}2,794,155} & \multicolumn{1}{c}{\textbf{4,095}} & \multicolumn{1}{c}{\textbf{1}}
%\end{tabular}
%\end{table}

From Francois' manuscript on the PCE's he states that it can be 
proved from the vector spaces' idea that the number of PCE channels 
of $N$ qubits $\mathcal{S}_{N,K}$ that leave $K$ Pauli components invariant is:
\begin{equation}
\mathcal{S}_{N,K}=\prod_{m=0}^{K-1}\frac{2^{2N-m}-1}{2^{K-m}-1}.
\end{equation}

\section{Rainbow correspondence}
\begin{figure}
  \centering
  \includegraphics[width=\textwidth]{hermanitos}
  \caption{Los hermanitos de 2 qubits. La fila de arriba muestra al hermano 
  mayor del canal que tiene exactamente abajo.}
  \label{•}
\end{figure}

Estos hermanitos tienen la siguiente forma
\begin{align}
\Phi=\E_i\ot \E_j+\qty(\1-\E_i)\ot \qty(\1-\E_j),
\end{align}
donde $\E_i$ son el canal identidad y los 3 canales PCE que mapean 
la esfera de Bloch hacia alguno de los 3 ejes cartesianos.

En general, un canal cuántico PCE $\phi$ que deja invariantes únicamente 
dos componentes, $\tau_{0,\ldots,0}$ y otra $\tau_{j_1,\ldots,j_n}$,
tiene un `hermano mayor PCE' $\Phi$ que deja invariantes $4^n-1$
componentes que es de la forma
\begin{align}
\Phi=\sum_{i=0}^{i\leq n/2}
\sum_{j\in \mathcal{S}_{2i}^n}
\bigotimes_{j_i=1}^n\Omega{j_i},
\end{align}
donde $\mathcal{S}_{2i}^n$ es el conjunto que contiene 
a todos los subconjuntos con $2i$ elementos del 
conjunto \newline $\{ 1,2,3,\ldots,n \}$ y
\begin{align}
\Omega_{j_i}= \left\{ \begin{array}{rc}
             \E_{j_i}, & j_i \notin j \\
             \1 -\E_{j_i}, & j_i \in j.
             \end{array}
   \right.
\end{align}

For example, let us elaborate the example for the 3 and 4 qubits 
generators. First, the PCE generators for 3 qubits are
\begin{align}
\Phi_{j_1,j_2,j_3}=& \sum_{i=0}^{i\leq 3/2}
\sum_{j\in \mathcal{S}_{2i}^3}
\bigotimes_{j_i=1}^3\Omega{j_i} \nonumber \\
=&\ \Ej{1}\ot\Ej{2}\ot\Ej{3} + \id{1}\ot\id{2}\ot\Ej{3} \\
&+ \id{1}\ot\Ej{2}\ot\id{3} + \Ej{1}\ot\id{2}\ot\id{3}. \nonumber
\end{align}
The 4 qubits PCE generators are
\begin{align}
\Phi_{j_1,j_2,j_3,j_4}=& \sum_{i=0}^{i\leq 2}
\sum_{j\in \mathcal{S}_{2i}^4}
\bigotimes_{j_i=1}^4\Omega{j_i} \nonumber \\
=&\ \Ej{1}\ot\Ej{2}\ot\Ej{3}\ot\Ej{4} + 
\qty(\1-\Ej{1})\ot\qty(\1-\Ej{2})\ot\Ej{3}\ot\Ej{4} \nonumber \\
&+ \qty(\1-\Ej{1})\ot\Ej{2}\ot\qty(\1-\Ej{3})\ot\Ej{4}
+ \qty(\1-\Ej{1})\ot\Ej{2}\ot\Ej{3}\ot\qty(\1-\Ej{4}) \\
&+ \Ej{1}\ot\qty(\1-\Ej{2})\ot\qty(\1-\Ej{3})\ot\Ej{4}
+ \Ej{1}\ot\qty(\1-\Ej{2})\ot\Ej{3}\ot\qty(\1-\Ej{4}) \nonumber \\
&+ \Ej{1}\ot\Ej{2}\ot\qty(\1-\Ej{3})\ot\qty(\1-\Ej{4}) \nonumber \\
&+ \id{1}\ot\id{2}\ot\id{3}\ot\id{4}. \nonumber
\end{align}

% \input{notas_david.tex}
% \input{notas_alejo.tex}

\section{Resultados}
\begin{itemize}
\item Clases relacionadas por localidad. Quizá ver si las podemos juntar con control not
\item Regla de $2^k$
\item Reglas de simetría \janote{hermanitos PCE?}
\item Clasificacion de los generadores
\item El conjunto dentro del conjunto de todos los canales (puntos extremos)
\item Canales de Ruskai, que la interseccion es no trivial (no estan contenidos en ninguna direccion, pero la interseccion es no vacia)
\item Diagonalización de la matriz de Choi de los canales de Pauli
\item Canales PCE como espacios vectoriales
\end{itemize}

\section{Preguntas abiertas}
\begin{itemize}
\item El conjunto dentro del conjunto de todos los canales
\item Relaciones de localidad con las reglas de Francois. Creo que no hay. 
\item Son positivas todas las matrices de densidad que resultan de aplicar un mapeo no válido? \janote{Creo que sí. 
Para 1 qubit aplicar PCEs no CP siempre te deja dentro de la esfera de Bloch. 
Así que estoy casi seguro que lo mismo sucede para $n$ qubits.}
\item Linealidad de los eigenvalores de la matriz de Choi en función de las $\tau_{j_1\ldots,j_n}$ de canales `\textit{component erasing}' de qudits. 
\end{itemize}

\section{Secciones}
\begin{itemize}
\item Exposicion/motivacion del problema
\item Prueba de que la a diagonaliza
\item Los espacios vectoriales de francois
\item Derivaciones y resultados
\item Visualizacion y Ejemplos?
\end{itemize}





\bibliographystyle{unsrt}
\bibliography{references}
\end{document}
