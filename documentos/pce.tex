\section{PCE Maps} % {{{
\subsection{Quantum channels}
\subsection{Single qubit projective maps} % {{{
\ddnote{Sugiero otro nombre para esta seccion, pues los PCE no son los unicos mapas proyectivos, hay mapas proyectivos no unitales}
\begin{itemize}
\item The single qubit case, describe some of the features.
\item Also, maybe the Kraus operators.
\item Also some relation with decoherence. 
\end{itemize}
% }}}
\subsection{Pauli component erasing maps} % {{{
\begin{itemize}
\item Give a formal definition of the PCE, maybe give some two qubit examples
to illustrate non-trivial points. 
\item Canales de Ruskai Pauli maps, que la interseccion es no trivial (no estan contenidos en ninguna direccion, pero la interseccion es no vacia)
\item \afnote{Relación con canales citados en \cite{Siudzinska2020}?}
\end{itemize}
% }}}
\begin{itemize}
\item Give a formal definition of the PCE, maybe give some two qubit examples
to illustrate non-trivial points. 
\item This will serve to bring the derivation a bit more down to earth. 
\item Visualizacion y ejemplos?
\item Notar lo de los generadores y como sale la estructura jerárquica 
\item Son positivas todas las matrices de densidad que resultan de aplicar un
mapeo no válido? \janote{Creo que sí.  Para 1 qubit aplicar PCEs no CP siempre
te deja dentro de la esfera de Bloch.  Así que estoy casi seguro que lo mismo
sucede para $n$ qubits.}
\end{itemize}
% }}}