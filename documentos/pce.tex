\section{PCE Maps} % {{{
\subsection{Quantum channels}
\subsection{Single qubit case} % {{{
% \ddnote{Sugiero otro nombre para esta seccion, pues los PCE no son los unicos
% mapas proyectivos, hay mapas proyectivos no unitales}
\todo{David escribe una primera version y luego Carlos itera}
\begin{itemize}
\item The single qubit case, describe some of the features.
\item Single qubit Kraus operators.
\item Single qubit relation with decoherence. 
\end{itemize}
% }}}
\subsection{Pauli component erasing maps} % {{{
\begin{itemize}
\item Give a formal definition of the PCE \todo{Jose Alfredo escribe}
\item Give some two qubit examples to illustrate non-trivial points with examples
(isualizacion y ejemplos)
\todo{Jose Alfredo escribe}
\item Canales de Ruskai Pauli maps, que la interseccion es no trivial (no estan contenidos en ninguna direccion, pero la interseccion es no vacia)
y Relación con canales citados en \cite{Siudzinska2020}? \cpnote{Esto quiza lo tengamos
que mover si es que necesitamos herramientas que discutiremos mas adelante para
hacer notar la diferencia}
\end{itemize}
% }}}
% \begin{itemize}
% \item Son positivas todas las matrices de densidad que resultan de aplicar un
% mapeo no válido? \janote{Creo que sí.  Para 1 qubit aplicar PCEs no CP siempre
% te deja dentro de la esfera de Bloch.  Así que estoy casi seguro que lo mismo
% sucede para $n$ qubits.}
% \end{itemize}
% }}}
