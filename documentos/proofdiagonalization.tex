\section{Diagonalization of Choi Matrix $\mathcal{D}_N$} % {{{
\label{sec:appendix:diagonalization}
% 
\subsection{Single qubit}
In order to simplify the derivation of the relations, let us employ pairs of binary indices instead of a single quaternary, i.e. $\alpha\to\mu+2\nu$. Hence we can write the elements of the Pauli basis
$(\sigma_{0},\sigma_{1},i\sigma_{2},\sigma_{3})$, compactly as
$\sigma_{\mu\nu}=\sum_{j=0}^{1} (-1)^{j \mu}\dyad{j}{j\oplus \nu}$. In vectorized form
\begin{equation}
\kket{\sigma_{\mu\nu}}=\sum_{j=0}^{1} (-1)^{j \mu}\kket{j,j\oplus \nu},
\end{equation}
in the same way,
\begin{equation}
\kket{l,l\oplus n}=\frac{1}{2}\sum_{m=0}^{1} (-1)^{ml}\kket{\sigma_{mn}}.
\label{PauToComp}
\end{equation}
On the other hand, the matrix form of an arbitrary Pauli map $\mathcal{E}$ may be written as
\begin{align}
\hat{\mathcal{E}} & = \frac{1}{2}\sum_{\mu\nu=0}^{1} \tau_{\mu\nu} \ddyada{\sigma_{\mu\nu}},\\
& = \frac{1}{2}\sum_{jk\mu\nu} \tau_{\mu\nu} (-1)^{\mu(j+k)} \ddyad{j,j\oplus\nu}{k,k\oplus\nu}.
% \label{Map_Pauli}
\end{align}
After applying the reshuffling operation on $\hat{\mathcal{E}}$, we obtain the Choi matrix associated to the map. It reads
\begin{equation}
 \mathcal{D} = 
  \frac{1}{2}\sum_{jk\mu\nu} \tau_{\mu\nu} (-1)^{\mu(j+k)} 
      \ddyad{j,k}{j\oplus\nu,k\oplus\nu}.
\label{Choi_a1}
\end{equation}
Note furthermore that the expression above may also be written as a combination of tensor products of Pauli matrices
% 
\begin{equation}
 \mathcal{D} = \frac{1}{2}\sum_{\mu\nu} \tau_{\mu\nu} \sigma_{\mu\nu} \otimes \sigma_{\mu\nu}^*.
 \label{Choi_Pauli_1q}
\end{equation}
% 
Back to the Eq. \ref{Choi_a1}), let us apply the index relabeling $k\to j\oplus m$, then the Choi matrix holds
\begin{equation}
 \mathcal{D} =
    \frac{1}{2}\sum_{jm\mu\nu} \tau_{\mu\nu} (-1)^{\mu m} 
        \ddyad{j,j\oplus m}{j\oplus\nu,j\oplus\nu\oplus m}.
% \label{Choi_a1}
\end{equation}
Due to the fact that we are dealing with binary indices, thus we can safely write the modular sums in the exponents as standard sums. By employing the relation between computational and Pauli elements (Eq. \ref{PauToComp}), by noticing that $\sum_j(-1)^{j(m\pm n)}=2\delta_{mn}$, and after some steps we have
\begin{align*}
 \mathcal{D} & = 
%  \frac{1}{2}\sum_{jm\mu\nu} \tau_{\mu\nu}(-1)^{\mu m}  \left( \frac{1}{2}\sum_{k=0}^{1} (-1)^{kj}\kket{\sigma_{km}}\right)\left(\frac{1}{2}\sum_{n=0}^{1} (-1)^{n(j\oplus\nu)}\bbra{\sigma_{nm}}\right),\\
\frac{1}{8}\sum_{jkmn\mu\nu} \tau_{\mu\nu}(-1)^{\mu m+kj+n(j\oplus\nu)}  \ddyad{\sigma_{km}}{\sigma_{nm}},\\
%  & = \frac{1}{8}\sum_{kmn\mu\nu} \tau_{\mu\nu} (-1)^{\mu m+n\nu} \left(\sum_{j} (-1)^{j(k+n)} \right) \ddyad{\sigma_{km}}{\sigma_{nm}},\\
%  & = \frac{1}{8}\sum_{kmn\mu\nu} \tau_{\mu\nu} (-1)^{\mu m+n\nu} (2\delta_{kn}) \ddyad{\sigma_{km}}{\sigma_{nm}},\\
 & = \frac{1}{4}\sum_{mn\mu\nu} \tau_{\mu\nu} (-1)^{\mu m+n\nu} \ddyad{\sigma_{nm}}{\sigma_{nm}},\\
 & = \frac{1}{2}\sum_{mn} \left(\frac{1}{2}\sum_{\mu\nu} \tau_{\mu\nu} (-1)^{\mu m+n\nu}\right) \ddyad{\sigma_{nm}}{\sigma_{nm}},
% \label{Choi_a1}
\end{align*}
Then $\mathcal{D}$ is diagonal in the Pauli basis, with eigenvalues given by
\begin{equation}
\lambda_{mn}=\frac{1}{2}\sum_{\mu\nu} \tau_{\mu\nu} (-1)^{\mu m+n\nu},
\end{equation}
or more compactly $\vec{\lambda} = \sa~\vec{\tau}$, where $\sa=2^{-1}H\otimes
H$ and $H$ is the Hadamard matrix.

\subsection{$N$ qubits}

Analogously to the previous case, we can define an arbitrary Pauli map acting on $N$ qubits density matrices in terms of the tensor products of vectorized Pauli matrices
\begin{equation}
\hat{\mathcal{E}}_N = \frac{1}{2^N}\sum_{\vec{\mu}\vec{\nu}} \tau_{\vec{\mu}\vec{\nu}} \ddyada{\sigma_{\vec{\mu}\vec{\nu}}}, 
\end{equation}
where
\begin{widetext}
\begin{align}
\kket{\sigma_{\vec{\mu}\vec{\nu}}} & = \kket{\sigma_{\mu_1\nu_1}}\otimes \cdots \kket{\sigma_{\mu_N\nu_N}},\\
& = \left(\sum_{j_1=0}^{1} (-1)^{j_1 \mu_1}\kket{j_1,j_1\oplus \nu_1}\right)\otimes\cdots \otimes \left(\sum_{j_N=0}^{1} (-1)^{j_N \mu_N}\kket{j_N,j_N\oplus \nu_N}\right),\\
& = \sum_{\vec{j}} (-1)^{\vec{j}\cdot\vec{\mu}} \kket{j_1,j_1\oplus \nu_1}\otimes\cdots \otimes \kket{j_N,j_N\oplus \nu_N},\\
& = \sum_{\vec{j}} (-1)^{\vec{j}\cdot\vec{\mu}} \kket{\vec{j},\vec{j}\oplus \vec{\nu}},
\end{align}
\end{widetext}
and
\begin{equation}
\bbrakket{\sigma_{\vec{\mu}\vec{\nu}}}{\sigma_{\vec{\mu}'\vec{\nu}'}} = 2^N \delta_{\vec{\mu}\vec{\mu}'}\delta_{\vec{\nu}\vec{\nu}'}.
\end{equation}

Recall the relation between the computational and Pauli basis elements (Eq. \ref{PauToComp}), hence for the $N$-qubits case we have
% 
\begin{widetext}
\begin{align}
 \kket{\vec{l},\vec{l}\oplus \vec{n}} & = \kket{l_1,l_1\oplus n_1} \otimes\cdots \otimes \kket{l_N,l_N\oplus n_N} \\
  & = \left( \frac{1}{2}\sum_{m_1=0}^{1} (-1)^{m_1l_1}\kket{\sigma_{m_1n_1}}\right)\otimes\cdots \otimes\left(\frac{1}{2}\sum_{m_N=0}^{1} (-1)^{m_Nl_N}\kket{\sigma_{m_Nn_N}}\right),\\
 & = \frac{1}{2^N}\left(\sum_{\vec{m}} (-1)^{\vec{m}\cdot\vec{l}}\kket{\sigma_{m_1n_1}}\otimes\cdots \otimes\kket{\sigma_{m_Nn_N}}\right),\\
 & = \frac{1}{2^N}\sum_{\vec{m}} (-1)^{\vec{m}\cdot\vec{l}}\kket{\sigma_{\vec{m}\vec{n}}}.
\label{PauToCompN}
\end{align}
\end{widetext}
By employing the previous relations, we can write the matrix representation of the map, $\hat{\mathcal{E}}_N $ in the $N$-qubits computational basis as
\begin{align}
\hat{\mathcal{E}}_N = & \frac{1}{2^N}\sum_{\vec{\mu}\vec{\nu}} \tau_{\vec{\mu}\vec{\nu}} \ddyada{\sigma_{\vec{\mu}\vec{\nu}}},\\
= & \frac{1}{2^N}\sum_{\vec{j}\vec{k}\vec{\mu}\vec{\nu}} \tau_{\vec{\mu}\vec{\nu}} (-1)^{\vec{\mu}\cdot(\vec{j}+\vec{k})} \ddyad{\vec{j},\vec{j}\oplus \vec{\nu}}{\vec{k},\vec{k}\oplus \vec{\nu}}.
\end{align}
In this way it is straightforward to apply the reshuffling operation on $\hat{\mathcal{E}}_N$ to obtain the associated Choi matrix,
\begin{equation}
\mathcal{D}_N =\frac{1}{2^N}\sum_{\vec{j}\vec{k}\vec{\mu}\vec{\nu}} \tau_{\vec{\mu}\vec{\nu}} (-1)^{\vec{\mu}\cdot(\vec{j}+\vec{k})} \ddyad{\vec{j},\vec{k}}{\vec{j}\oplus \vec{\nu},\vec{k}\oplus \vec{\nu}}.
\end{equation}
After the change of vector indices $\vec{k}=\vec{j}\oplus\vec{l}$, the Choi matrix holds
\begin{equation}
\mathcal{D}_N =\frac{1}{2^N}\sum_{\vec{j}\vec{l}\vec{\mu}\vec{\nu}} \tau_{\vec{\mu}\vec{\nu}} (-1)^{\vec{\mu}\cdot(\vec{j}+\vec{j}\oplus\vec{l})} \ddyad{\vec{j},\vec{j}\oplus\vec{l}}{\vec{j}\oplus \vec{\nu},\vec{j}\oplus \vec{\nu}\oplus \vec{l}}.
\end{equation}
Back to the vectorized Pauli basis (Eq. \ref{PauToCompN}) we have
\begin{equation}
\mathcal{D}_N = \frac{1}{2^{3N}}\sum_{\substack{\vec{j}\vec{l}\vec{m} \\ \vec{n}\vec{\mu}\vec{\nu}}} \tau_{\vec{\mu}\vec{\nu}} (-1)^{\vec{\mu}\cdot\vec{l}+\vec{j}\vec{m}+(\vec{j}\oplus\vec{\nu})\cdot\vec{n}} \ddyad{\sigma_{\vec{m}\vec{l}}}{\sigma_{\vec{n}\vec{l}}}.
\end{equation}
On the other hand, recall the relation $\sum_{\vec{j}}(-1)^{\vec{j}\cdot(\vec{m}+\vec{n})}=2^N\delta_{\vec{m}\vec{n}}$. After some steps $\mathcal{D}_N$ reduces to 
\begin{align}
\mathcal{D}_N & = \frac{1}{2^{2N}}\sum_{\vec{m}\vec{n}\vec{\mu}\vec{\nu}} \tau_{\vec{\mu}\vec{\nu}} (-1)^{\vec{\mu}\cdot\vec{n}+\vec{\nu}\cdot\vec{m}} \ddyad{\sigma_{\vec{m}\vec{n}}}{\sigma_{\vec{m}\vec{n}}},\\
& = \frac{1}{2^{N}}\sum_{\vec{m}\vec{n}} \left( \frac{1}{2^{N}}\sum_{\vec{\mu}\vec{\nu}} \tau_{\vec{\mu}\vec{\nu}} (-1)^{\vec{\mu}\cdot\vec{n}+\vec{\nu}\cdot\vec{m}} \right)\ddyad{\sigma_{\vec{m}\vec{n}}}{\sigma_{\vec{m}\vec{n}}}.
\end{align}
Thus the eigenvalues of the Choi matrix $\mathcal{D}_N$ may be written as
\begin{equation}
\lambda_{\vec{m}\vec{n}}= \frac{1}{2^{N}}\sum_{\vec{\mu}\vec{\nu}} \tau_{\vec{\mu}\vec{\nu}} (-1)^{\vec{\mu}\cdot\vec{n}+\vec{\nu}\cdot\vec{m}}
\end{equation}
or more compactly $\vec{\lambda} = \san~\vec{\tau}$, where $\san=a^{\otimes N}$, $\sa=2^{-1}H\otimes H$ and $H$ is the Hadamard matrix.
% }}}


