\section{Diagonalization of Choi Matrix $\mathcal{D}_N$} % {{{
\label{sec:appendix:diagonalization}
% 
In order to simplify the derivation of the relations, let us employ pairs of
binary indices instead of a single quaternary, i.e. $\alpha\to\mu+2\nu$. In
this way we can write the elements of the Pauli basis
$(\sigma_{0},\sigma_{1},i\sigma_{2},\sigma_{3})$ in the standard and vectorized
forms, compactly as
\begin{equation}
\sigma_{\mu\nu}=\sum_{j=0}^{1} (-1)^{j \mu}\dyad{j}{j\oplus \nu},
\end{equation}
and
\begin{equation}
\kket{\sigma_{\mu\nu}}=\sum_{j=0}^{1} (-1)^{j \mu}\ket{j,j\oplus \nu}.
\end{equation}
In addition, recall the inverse relation between the computational and Pauli elements
\begin{equation}
\ket{l,l\oplus n}=\frac{1}{2}\sum_{m=0}^{1} (-1)^{ml}\kket{\sigma_{mn}}.
\label{PauToComp}
\end{equation}
On the other hand, the matrix form of an arbitrary Pauli map $\mathcal{E}$ may
be written as
\begin{align}
\hat{\mathcal{E}} & = \frac{1}{2}\sum_{\mu\nu=0}^{1} \tau_{\mu\nu} \ddyada{\sigma_{\mu\nu}},\\
& = \frac{1}{2}\sum_{jk\mu\nu} \tau_{\mu\nu} (-1)^{\mu(j+k)} \dyad{j,j\oplus\nu}{k,k\oplus\nu}.
% \label{Map_Pauli}
\end{align}
After applying the reshuffling operation on $\hat{\mathcal{E}}$, we obtain the
Choi matrix associated to the map. It reads
\begin{equation}
 \mathcal{D} = 
  \frac{1}{2}\sum_{jk\mu\nu} \tau_{\mu\nu} (-1)^{\mu(j+k)} 
      \dyad{j,k}{j\oplus\nu,k\oplus\nu}.
\label{Choi_a1}
\end{equation}

Let us apply the index relabeling $k\to j\oplus m$, then the Choi matrix holds
\begin{equation}
 \mathcal{D} =
    \frac{1}{2}\sum_{jm\mu\nu} \tau_{\mu\nu} (-1)^{\mu m} 
        \dyad{j,j\oplus m}{j\oplus\nu,j\oplus\nu\oplus m}.
\label{Choi_a1}
\end{equation}
By using the relation between computational and Pauli elements (Eq.
\ref{PauToComp}), we have
\begin{align*}
 \mathcal{D} & = \frac{1}{2}\sum_{jm\mu\nu} \tau_{\mu\nu}(-1)^{\mu m}  \left( \frac{1}{2}\sum_{k=0}^{1} (-1)^{kj}\kket{\sigma_{km}}\right)\left(\frac{1}{2}\sum_{n=0}^{1} (-1)^{n(j\oplus\nu)}\bbra{\sigma_{nm}}\right),\\
 & = \frac{1}{8}\sum_{jkmn\mu\nu} \tau_{\mu\nu}(-1)^{\mu m+kj+n(j\oplus\nu)}  \ddyad{\sigma_{km}}{\sigma_{nm}},\\
 & = \frac{1}{8}\sum_{kmn\mu\nu} \tau_{\mu\nu} (-1)^{\mu m+n\nu} \left(\sum_{j} (-1)^{j(k+n)} \right) \ddyad{\sigma_{km}}{\sigma_{nm}},\\
 & = \frac{1}{8}\sum_{kmn\mu\nu} \tau_{\mu\nu} (-1)^{\mu m+n\nu} (2\delta_{kn}) \ddyad{\sigma_{km}}{\sigma_{nm}},\\
 & = \frac{1}{4}\sum_{mn\mu\nu} \tau_{\mu\nu} (-1)^{\mu m+n\nu} \ddyad{\sigma_{nm}}{\sigma_{nm}},\\
 & = \frac{1}{2}\sum_{mn} \left(\frac{1}{2}\sum_{\mu\nu} \tau_{\mu\nu} (-1)^{\mu m+n\nu}\right) \ddyad{\sigma_{nm}}{\sigma_{nm}},
% \label{Choi_a1}
\end{align*}
Then $\mathcal{D}$ is diagonal in the Pauli basis, with eigenvalues given by
\begin{equation}
\lambda_{mn}=\frac{1}{2}\sum_{\mu\nu} \tau_{\mu\nu} (-1)^{\mu m+n\nu},
\end{equation}
or more compactly $\vec{\lambda} = \sa~\vec{\tau}$, where $\sa=2^{-1}H\otimes
H$ and $H$ is the Hadamard matrix.

\afnote{Queda faltando pasar a limpio la generalizacion para $N$ qubits.}

% }}}


