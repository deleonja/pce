\section{Krauss}

In order to study the Krauss operators corresponding to PCE's, it is enough to study 
the ones corresponding to generators, that is, the ones that leave 
have of the components invariant. 

As a historical fact, the Krauss operators corresponding to one single qubit PCEs
are 
\begin{equation}
\{\openone/\sqrt2, \sigma_i/\sqrt2\}
\label{}
\end{equation}
(corresponding to the operation that leaves the component $\sigma_i$ invariant), 
see Eq. 8.95 of Chuang's book.  Visual inspection of the Krauss operators 
of two qubit PCEG (PCE generators) lead to the ansatz that 
the appropriate Krauss operators are 
\begin{equation}
\left\{\openone/\sqrt2, \left(\bigotimes_k \sigma^{(k)}_{i_k}\right)/\sqrt2\right\}. 
\label{}
\end{equation}

We shall first show that the aforementioned Kraus operators indeed produce a PCE. 
Notice that 
\begin{equation}
\sigma_i \sigma_j \sigma_i =a^i_{j} \sigma_j,
\label{}
\end{equation}
with $a$ the matrix that appears in 
\eref{eq:eigvals-PCE}. Next, consider the action of a channel with Krauss representation 
\eref{} on an $n$ qubit system with Bloch sphere representation \eref{eq:nqubit:bloch},



Counting, mostrar qe el numero de bichos que aparecen ahi coincide con el numero de generadores. 

Show that they are unique 
