\documentclass[11pt,dvipsnames]{article} % {{{

\usepackage{geometry}
\geometry{total={170mm,240mm}, left=20mm, top=20mm}

\usepackage[utf8]{inputenc}
\usepackage[english]{babel}
\usepackage{enumerate} 
\usepackage{pgfplots}
\usepackage{float}

% To add graphics
\usepackage{graphicx}
% Path to figures
\graphicspath{ {../images/} }

\usepackage[utf8]{inputenc}
\usepackage[english]{babel}
\usepackage{amsmath}
\usepackage{amsfonts}
\usepackage{amssymb}

\author{José Alfredo de León}
\title{Numerical progress on Weyl component erasing operations}
\begin{document}
\maketitle
\section{3-level system}
The results for Weyl component erasing (WCE) quantum channels claimed by Alejo
have been reproduced by an independent party (me). Since Alejo has been 
using two indices for the components $\tau_{\mu\nu}$ to characterize a WCE operation
we can represent the 3-level system WCE quantum channels as in Fig. 
\ref{fig:weylCE_1particle_d3}. Note the existence of the additional quantum 
channel that leaves invariant $\alpha_{00}$, $\alpha_{1,2}$ and $\alpha_{2,1}$
(fifth figure from left to right). Therefore, the so-called \textit{rainbow rule}
follows as 1 4 1 WCE channels that leave invariant 1 3 9 ``Weyl'' components
of the density matrix. 
\begin{figure}[h]
	\centering
	\includegraphics[width=12cm]{weylCE_1particle_d3}
	\caption{Representation of a 3-level system WCE quantum channels.}
	\label{fig:weylCE_1particle_d3}
\end{figure}

\section{4-level system}
In Fig. \ref{fig:weylCE_1particle_d4} I show the numerical results for 
4-level system WCE quantum channels. It is worth noting that the 
\textit{rainbow rule} is weirdly followed: 1 3 7 3 1 number of 
quantum channels that leave invariant 1 2 4 8 16 ``Weyl components'' 
of the density matrix.
\begin{figure}[h]
	\centering
	\includegraphics[width=10cm]{weylCE_1particle_d4}
	\caption{Representation of a 4-level system WCE quantum channels.}
	\label{fig:weylCE_1particle_d4}
\end{figure}

\end{document}