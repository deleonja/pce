\documentclass[11pt,dvipsnames]{article} % {{{

\usepackage{geometry}
\geometry{total={170mm,240mm}, left=20mm, top=20mm}

\usepackage[utf8]{inputenc}
\usepackage[english]{babel}
\usepackage{enumerate} 
\usepackage{pgfplots}
\usepackage{float}

% To add graphics
\usepackage{graphicx}
% Path to figures
\graphicspath{ {../images/} }

\usepackage[utf8]{inputenc}
\usepackage[english]{babel}
\usepackage{amsmath}
\usepackage{amsfonts}
\usepackage{amssymb}
\usepackage{physics}

%% commands
\newcommand{\alphas}{\vec\alpha}

\author{José Alfredo de León}
\title{Numerical progress on Weyl component erasing operations}
\begin{document}
\maketitle
\section{3-level system}
The results for Weyl component erasing (WCE) quantum channels claimed by Alejo
have been reproduced by an independent party (me). Since Alejo has been 
using two indices for the components $\tau_{\mu\nu}$ to characterize a WCE operation
we can represent the 3-level system WCE quantum channels as in Fig. 
\ref{fig:weylCE_1particle_d3}. Note the existence of the additional quantum 
channel that leaves invariant $\alpha_{00}$, $\alpha_{1,2}$ and $\alpha_{2,1}$
(fifth figure from left to right). Therefore, the so-called \textit{rainbow rule}
follows as 1 4 1 WCE channels that leave invariant 1 3 9 ``Weyl'' components
of the density matrix. 
\begin{figure}[h]
	\centering
	\includegraphics[width=12cm]{weylCE_1particle_d3}
	\caption{Representation of a 3-level system WCE quantum channels.}
	\label{fig:weylCE_1particle_d3}
\end{figure}

\section{4-level system}
In Fig. \ref{fig:weylCE_1particle_d4} I show the numerical results for 
4-level system WCE quantum channels. It is worth noting that the 
\textit{rainbow rule} is weirdly followed: 1 3 7 3 1 number of 
quantum channels that leave invariant 1 2 4 8 16 ``Weyl components'' 
of the density matrix.
\begin{figure}[h]
	\centering
	\includegraphics[width=10cm]{weylCE_1particle_d4}
	\caption{Representation of a 4-level system WCE quantum channels.}
	\label{fig:weylCE_1particle_d4}
\end{figure}

\section{Generalization \boldmath{$\oplus$}}
\subsection{New notation for \boldmath{$\alphas$} and definition of \boldmath{$\oplus$}}
We will rewrite the multi-indices $\alphas$ using base-$d$ numeral system 
and will re-define the operation $\oplus$, previously associated by François with the 
Klein group.
Every index in the multi-indices $\alphas$ of a WCE quantum channel 
can be written as a length-2 list with its base-$d$ numeral system digits. 
Therefore, every index $\alpha_i$ in $\alphas$ can be written as
\begin{align}
	\alpha_i=\qty( \alpha_{i_1},\alpha_{i_2} ).
\end{align}
The multi-indices $\alphas$ then reads
\begin{align}\label{eq:alphas_base-d}
	\alphas=\qty{ \qty( \alpha_{1_1},\alpha_{1_2} ),\ldots,\qty( \alpha_{N_1},\alpha_{N_2} ) }.
\end{align}
We define the sum of two multi-indices $\vec\alpha$ and $\vec\beta$ as 
\begin{align}\label{eq:oplus_generalized}
	\vec\alpha\oplus\vec\beta=
	\qty{ \qty( \alpha_{1_1}\oplus\beta_{1_1},\alpha_{1_2}\oplus\beta_{1_2} ),
	\ldots,
	\qty( \alpha_{N_1}\oplus\beta_{N_1},\alpha_{N_2}\oplus\beta_{N_2} ) 
	 },
\end{align}
where operation $\oplus$ is the sum modulo $d$.

\subsection{Examples}
Let us look at some examples. First, let us discuss a familiar example with a PCE 
quantum channel.
The set of multi-indices associated with the PCE quantum channel of 
Fig. \ref{fig:two_qubits_PCEChann_01}
are, in decimal basis, $\{(0,0),(1,1),(2,1),(3,0) \}$. The same multi-indices
written in binary basis are 
$$\qty{ \{(0,0),(0,0)\}, \{(0,1),(0,1)\}, \{(1,0),(0,1)\}, \{(1,1),(0,0)\} }.$$
Let us sum $(2,1)\oplus(1,1)$ as previously defined by François
\begin{align}
	(2,1)\oplus(1,1)=(2\oplus 1,1\oplus 1)=(3,0).
\end{align}
Now, let us sum the same two multi-indices, written in binary basis, using
the definition \eqref{eq:oplus_generalized}
\begin{align}
	 \{(1,0),(0,1)\}\oplus\{(0,1),(0,1)\}=\{(1\oplus0,0\oplus1),(0\oplus0,1\oplus1)\}
	 =\qty{ (1,1),(0,0) },
\end{align}
which is equal to $(3,0)$ in decimal basis. This is an example that shows how 
the operation $\oplus$ defined with the Klein group is equivalent with the new notation 
and definition \eqref{eq:oplus_generalized}.

\begin{figure}[h] 
	\centering
	\includegraphics[width=2cm]{two_qubits_c43}
	\caption{A random two-qubits PCE quantum channel.}
	\label{fig:two_qubits_PCEChann_01}
\end{figure}
Now, let us look at the set of multi-indices of the 
fifth 3-level-system WCE channel in Fig. \ref{fig:weylCE_1particle_d3}:
$\qty{ \qty{(0,0)}, \qty{(1,2)}, \qty{(2,1)} }$.
The only three different possible sums modulo 3 are
\begin{subequations}
\begin{align}
	(0,0)\oplus(1,2)&=(0\oplus1,0\oplus2)=(1,2)\\
	(0,0)\oplus(2,1)&=(0\oplus2,0\oplus1)=(2,1)\\
	(1,2)\oplus(2,1)&=(1\oplus2,2\oplus1)=(0,0)
\end{align}
\end{subequations}
Then, the set of multi-indices $\alphas$ is closed. 
As the last example, let us check the closure of the set of multi-indices of the 
tenth 4-level-system WCE channel in Fig \ref{fig:weylCE_1particle_d4}:
$\qty{ \qty{(0,0)}, \qty{(1,2)}, \qty{(2,0)}, \qty{(3,2)} }$
The only six different possible sums modulo 4 are
\begin{subequations}
\begin{align}
	(0,0)\oplus(1,2)&=(1,2)\\
	(0,0)\oplus(2,0)&=(2,0)\\
	(0,0)\oplus(3,2)&=(3,2)\\
	(1,2)\oplus(2,0)&=(3,2)\\
	(1,2)\oplus(3,2)&=(0,0)\\
	(2,0)\oplus(3,2)&=(1,2),
\end{align}
\end{subequations}
then the set of multi-indices is closed. 

\subsection{Hypothesis}
The set of multi-indices of a $N$-qudits WCE quantum channel, 
written in base-$d$ numeral system as in \eqref{eq:alphas_base-d},
is closed under sum modulo $d$ as defined in \eqref{eq:oplus_generalized}.
There are numerical results that support this hypothesis. The method of brute
force has been implemented to check if all sets of a
multi-indices of 3, 4 and partially 5-level system WCE quantum channels 
are closed under $\oplus$ as defined in \eqref{eq:oplus_generalized}.
The converse has been numerically analyzed with the method of brute force too.
All sets of multi-indices that are closed under the operation $\oplus$ are
associated with a WCE quantum channel. All of this shows clues that a vector 
structure exists as in the case of PCE quantum channels.


\end{document}