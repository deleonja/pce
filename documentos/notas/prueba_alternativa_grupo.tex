\documentclass[11pt,a4paper]{article}
\usepackage[utf8]{inputenc}
\usepackage[english]{babel}
\usepackage{amsmath}
\usepackage{amsfonts}
\usepackage{amssymb}
\usepackage{graphicx}
\usepackage[left=2cm,right=2cm,top=2cm,bottom=2cm]{geometry}
\usepackage{hyperref}
\usepackage{subfig}
\usepackage{placeins}

\usepackage[mathlines]{lineno}  \linenumbers
\setlength\linenumbersep{3pt}
\usepackage[draft,inline,nomargin]{fixme} \fxsetup{theme=color}
\FXRegisterAuthor{cp}{acp}{\color{blue}CP}
\definecolor{jacolor}{RGB}{200,40,0} \FXRegisterAuthor{ja}{aja}{\color{jacolor}JA}

\usepackage{physics}
\usepackage{graphicx}
% \usepackage{mathrsfs}
\usepackage{dutchcal}
\usepackage{multirow}
%


%% Notas
\usepackage[dvipsnames]{xcolor}
\newcommand\todo[1]{\textcolor{Green}{#1}} 

\usepackage{titlesec}

\setcounter{secnumdepth}{4}

\titleformat{\paragraph}
{\normalfont\normalsize\bfseries}{\theparagraph}{1em}{}
\titlespacing*{\paragraph}
{0pt}{3.25ex plus 1ex minus .2ex}{1.5ex plus .2ex}


\usepackage{imakeidx}
\makeindex[columns=3, title=Alphabetical Index, intoc]


\def\bbra#1{\mathinner{\langle \! \langle{#1}|}}
\def\kket#1{\mathinner{|{#1}\rangle \! \rangle}}
\def\ddyada#1{\mathinner{|{#1}\rangle\!\rangle\!\langle\!\langle{#1}|}}
\def\ddyad#1#2{\mathinner{|{#1}\rangle\!\rangle\!\langle\!\langle{#2}|}}
\def\bbrakket#1#2{\mathinner{\langle\!\langle {#1}|{#2}\rangle\!\rangle}}
\newcommand{\sa}{\ensuremath{\textit{a}}}
\newcommand{\san}{\ensuremath{\textit{A}}}
\newcommand{\Mod}[1]{\ (\mathrm{mod}\ #1)}
% 
\title{Group structure of PCEs}
\author{José Alfredo de León}
\date{}
% 
\begin{document}
\maketitle
%\tableofcontents


For a PCE quantum channel If $\tau_{pq}=\tau_{p'q'}=1$, with $p,q\ne p',q'$, 
then $\tau_{p+p',q+q'}=1$. Our proof goes as follows. With the expressions
for $\tau_{pq}$ in terms of $\lambda_{\mu\nu}$ and the sum of all 
$\lambda_{\mu\nu}$ we derive another expression that must hold for a 
particular set of indices $\mu,\nu$. With this, the algebraic property
is computed straightforwardly. This very important 
property reveals the semigroup structure of 
PCE quantum channels. 

We will begin from the expression for the Choi's eigenvalues $\lambda_{\mu \nu}$
with binary indices, that is $\mu,\nu\in\qty{0,1}$. This form of the 
eigenvalues read 
\begin{align}
\lambda_{\mu\nu}=\frac{1}{2}\sum_{m,n=0}^1\qty(-1)^{\mu m+\nu n}\tau_{mn}
\Mod{2},
\end{align}
where sums are modulo $2$. From here on we will use implicitly the
modulo 2 arithmetic and drop the $\Mod{2}$ notation.
To invert this expression let us multiply each side by $\qty(-1)^{-\mu p-\nu q}$
and sum over $\mu$ and $\nu$,
\begin{align}\label{eq:inverting}
\sum_{\mu,\nu=0}^1\qty(-1)^{-\mu p-\nu q}\lambda_{\mu\nu}=
\frac{1}{2}\sum_{m,n,\mu,\nu=0}^1\qty(-1)^{\mu(m-p) m+\nu (n-q)}\tau_{mn}.
\end{align}
We notice that 
\begin{align}
\sum_{j=0}^1(-1)^{j(k-l)}=1+(-1)^{k-l}=2\delta_{kl}
\end{align}
because $k-l=0$ when $k=l$ and $k-l=1$ when $k\ne l$. There is no other possible
value because we are using modulo 2 arithmetic. Using this property, we resume
to eq. \eqref{eq:inverting},
\begin{subequations}
\begin{align}
\sum_{\mu,\nu=0}^1\qty(-1)^{-\mu p-\nu q}\lambda_{\mu\nu}&=
\frac{1}{2}\sum_{m,n,\mu,\nu=0}^1(2\delta_{mp})(2\delta_{nq})\tau_{mn} \\
\tau_{pq}&=\frac{1}{2}\sum_{\mu,\nu=0}^1\qty(-1)^{-\mu p-\nu q}\lambda_{\mu\nu}.
\label{eq:tau:lambdas}
\end{align}
\end{subequations}
Now, we subtract $\sum_{\mu,\nu}\lambda_{\mu\nu}=2$
from $2=\sum_{\mu,\nu}\qty(-1)^{-\mu p-\nu q}\lambda_{\mu\nu}$ (setting
$\tau_{pq}=1$ in \eqref{eq:tau:lambdas}),
\begin{align}\label{eq:paso}
\sum_{\mu,\nu=0}^1\lambda_{\mu\nu}\qty[(-1)^{-\mu p-\nu q}-1]=0.
\end{align}
Recall that $\lambda_{\mu \nu}\geq 0$ for the corresponding map to be 
completely positive, and $(-1)^{-\mu p-\nu q}-1\leq 0$ because we are
summing modulo 2, thus $-\mu p-\nu q=0,1$. These being said, now it must
be evident that every term in left side of \eqref{eq:paso} has to be
zero for the expression to hold. Therefore, if $\lambda_{\mu\nu}\ne 0$,
then
\begin{align}\label{eq:property}
(-1)^{-\mu p}=(-1)^{\nu q},\qquad \forall\,\mu,\nu \vert \lambda_{\mu\nu}\ne 0.
\end{align}
The same expression holds for indices $p',q'$. Now we use this expression to
evaluate 
\begin{subequations}
\begin{align}
\tau_{m+m',n+n'}&=\frac{1}{2}\sum_{\mu,\nu=0}^1
(-1)^{-\mu(p+p')-\nu(q+q')}\lambda_{\mu\nu} \\
&=\frac{1}{2}\sum_{\mu,\nu|\lambda_{\mu\nu}\ne 0}
(-1)^{\nu(q+q')-\nu(q+q')}\lambda_{\mu\nu}\label{eq:paso:7b}\\
&=\frac{1}{2}\sum_{\mu,\nu|\lambda_{\mu\nu}\ne 0}\lambda_{\mu\nu}\label{eq:paso:7c}\\
&=1,
\end{align}
\end{subequations}
where \eqref{eq:property} has been used in \eqref{eq:paso:7b} and the fact
that Choi's eigenvalues sum $2$ in \eqref{eq:paso:7c}. $\square$
% 
\bibliography{References}{}
\bibliographystyle{unsrt}
% 
\end{document}
