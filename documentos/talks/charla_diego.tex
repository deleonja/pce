% {{{Preamble
\documentclass{beamer}
% packages and style {{{
\usepackage[utf8]{inputenc}
% \usepackage{marvosym}
\usepackage{extarrows}
% \usepackage{hieroglf}
% \usepackage{cancel}
\usepackage{pgf,pgfnodes,pgfarrows}
\graphicspath{{images/},{figures/},{../},{/home/carlosp/investigacion/varios/images/},{/home/carlosp/fun/Dropbox/Stoch/JPA/},{/home/carlosp/investigacion/antiguos_proyectos/cuda/images/},{/home/carlosp/docencia/images/},{/home/carlosp/investigacion/antiguos_proyectos/nacho/images/},{/home/carlosp/investigacion/antiguos_proyectos/satya/images/},{/home/carlosp/investigacion/antiguos_proyectos/cuda/program_description/images/},{/home/carlosp/investigacion/antiguos_proyectos/mera2d/images/},{/home/carlosp/investigacion/antiguos_proyectos/newspectra/eps/},{/home/carlosp/administrativo/academico/old/2013/definitividad/images/},{/home/carlosp/investigacion/antiguos_proyectos/tesis/texto/eps/},{/home/carlosp/fun/Dropbox/quantum_vault/images/},{/home/carlosp/investigacion/antiguos_proyectos/quantum_vault/charla/images/},{/home/carlosp/fun/Dropbox/quantum_vault/charla/},{/home/carlosp/investigacion/antiguos_proyectos/gorin/images/},{/home/carlosp/investigacion/antiguos_proyectos/cgg/talks/images/},{/home/carlosp/investigacion/antiguos_proyectos/cgg/talks/},{/home/carlosp/investigacion/antiguos_proyectos/cgg/images/},{/home/carlosp/investigacion/antiguos_proyectos/diversity/images/},{/home/carlosp/administrativo/academico/2013/definitividad/images/},{/home/carlosp/administrativo/academico/2013/titular_a/images/},{/home/carlosp/investigacion/antiguos_proyectos/quantum_vault/charla/},{/home/carlosp/investigacion/antiguos_proyectos/quantum_vault/},{/home/carlosp/investigacion/antiguos_proyectos/nacho/},{/home/carlosp/docencia/cursos/images/}}
% \usepackage{psfrag}
% \usepackage{pst-node}
\usepackage{tikz}

\definecolor{colora}{rgb}{0.898438, 0.621094, 0.}
\definecolor{colorb}{rgb}{0.335938, 0.703125, 0.910156}
\definecolor{colorc}{rgb}{0., 0.617188, 0.449219}
\definecolor{colord}{rgb}{0.9375, 0.890625, 0.257813}
\definecolor{colore}{rgb}{0., 0.445313, 0.695313}
\definecolor{colorf}{rgb}{0.832031, 0.367188, 0.}

\newcommand{\GHZ}{\text{GHZ}}
\newcommand{\ket}[1]{{| #1 \rangle}}
\newcommand{\tabitem}{~~\llap{\textbullet}~~}
\newcommand{\fuzzytwo}[1]{\ensuremath{\Lambda_#1^\text{fuzzy}} }
\newcommand{\mcS}{\mathcal{S}}
\newcommand{\cgtwo}[1]{\ensuremath{\Lambda_#1^\text{cg}} }

\usepackage{dsfont}
\usepackage{pifont}% http://ctan.org/pkg/pifont
% \usepackage{soul}
% 
%     \setul{0.5ex}{0.3ex}
%     \definecolor{Blue}{rgb}{0,0.0,1}
%     \setulcolor{Blue}

\newcommand{\cmark}{\ding{51}}%
\newcommand{\xmark}{\ding{55}}%

\usepackage{cfp}
\usepackage{array}
\newcolumntype{L}[1]{>{\raggedright\let\newline\\\arraybackslash\hspace{0pt}}m{#1}}
\newcolumntype{C}[1]{>{\centering\let\newline\\\arraybackslash\hspace{0pt}}m{#1}}
\newcolumntype{R}[1]{>{\raggedleft\let\newline\\\arraybackslash\hspace{0pt}}m{#1}}
\usetikzlibrary{arrows,positioning} 
\usetikzlibrary{shapes,snakes}
% The following 2 to include movies
% \usepackage{movie15}
\usepackage{hyperref}
% 
\usetikzlibrary{arrows,backgrounds}
\usepackage{amsmath}
\mode<presentation>
{ 
% \usetheme{Malmoe} 
% \usetheme{Pittsburgh}
\usetheme{default}
\useoutertheme{infolines}
\usecolortheme{default}
% \usecolortheme{lily}
% \usecolortheme{albatross}
}
% \usepackage{mathptmx} \usepackage{helvet}

%Diferente, bien
%\usepackage{arev}

% Mas diferente y bien. Un poco no profesional
%\usepackage{avant}

% Diferente y bien
% \usepackage{bookman}

% PRofesional y diferente.
%\usepackage{chancery}

% Font mathematico bonito.
% \usepackage{euler}

% Font interesante para mathematicas
% \usepackage{mathptm}
% \usepackage{mathptmx}

% Tambien font interensante. no se\dots 
% \usepackage{newcent}


% Profesional e interesante 
%\usepackage{palatino}

%\usepackage{pifont}
% \usepackage{utopia}

% \usefonttheme[onlymath]{serif}
% \usepackage{fontspec} 
% \defaultfontfeatures{Mapping=tex-text} 
% % \setsansfont[Ligatures={Common}]{Futura}
% \setmonofont[Scale=0.8]{Monaco} 




\makeatletter
\AtBeginPart{%
    \addtocontents{parttoc}{\protect\beamer@partintoc{\the\c@part}{\beamer@partnameshort}{\the\c@page}}%
    \frame{\partpage \titlepage}%
}
\newcommand{\parttableofcontents}{\@starttoc{parttoc}}
\newcommand{\beamer@partintoc}[3]{#2\par}
  \def\X@no@opt@arg#1{$#1$ &\ttfamily\string#1}
  \def\X{\@ifnextchar[{\X@opt@arg}{\X@no@opt@arg}}
\DefineNamedColor{named}{BrickRed}      {cmyk}{0,0.89,0.94,0.28}
\DefineNamedColor{named}{Red}           {cmyk}{0,1,1,0}
\DeclareMathOperator*{\proptol}{\propto}
\input{commondefinitions}
% \usepackage{stmaryrd}
% \usepackage[activeacute]{babel} 
% \spanishdecimal{.}
\renewcommand{\div}{\text{divi}}
\def\q{{\textrm{q} }}
\newcommand{\mcJ}{\mathcal{J}}	
\newcommand{\imag}{\rmi}
\newcommand{\KI}{\text{KI}}
\newcommand{\mcI}{\mathcal{I}}	
\newcommand{\mcB}{\mathcal{B}}	
\newcommand{\mcP}{\mathcal{P}}	
\newcommand{\mcM}{\mathcal{M}}	
\newcommand{\mcN}{\mathcal{N}}	
\newcommand{\mcE}{\mathcal{E}}	
\newcommand{\mcL}{\mathcal{L}}	
\newcommand{\crit}{\text{crit}}	
\newcommand{\env}{\text{env}}	
% \newcommand{\env}{\text{rest}}	
\newcommand{\cen}{\text{cen}}	
\newcommand{\GUE}{\text{GUE}}	
\newcommand{\eff}{\text{eff}}	
\renewcommand{\tr}{\text{tr}}	
\newcommand{\diag}{\mathop{\mathrm{diag}}\nolimits}
\newcommand{\rmf}{f}
\newcommand{\mcK}{\mathcal{K}}
\newcommand{\Mmed}{\mcM^{\<\cdot\>}}
\newcommand{\old}{\infty}
\newcommand{\mcQ}{\mathcal{Q}}
\newcommand{\mcC}{\mathcal{C}}
%\newcommand{\mathbbm}{\openone}
\renewcommand{\>}{\rangle}
\def\1{{\mathchoice{\rm 1\mskip-4mu l}{\rm 1\mskip-4mu l}{\rm 1\mskip-4.5mu l}{\rm 1\mskip-5mu l} }} 
\newcommand{\jamiol}{Jamio{\l}kowski }
\newcommand{\id}{\mathbbm{1}}
\newcommand{\C}{\mathbb{C}}
\newcommand{\R}{\mathbb{R}}
\newcommand{\affiliationone}{$^\flat$}
\newcommand{\affiliationtwo}{$^\sharp$}
\newcommand{\affiliationthree}{$^\natural$}
% }}}
% \englishtrue
% \englishtrue \espanolfalse
\englishfalse \espanoltrue
% \espanoltrue
% \title[]{2-d Ising Quantum Models in CUDA:\\Phase Transitions in Spectral Densities}
\title{
% A quantum framework for \\c
Pauli component erasing operations
% Coarse graining and fuzzy measurements in quantum many-body-systems
}
\author[]{Carlos Pineda$^1$\\
\footnotesize
José Alfredo de León$^2$
Alejandro Fonseca$^3$
François Leyvraz$^1$
David Davalos$^4$
}
\institute[c. pineda]{
$^1$Universidad Nacional Autónoma de México\\
$^2$Universidad de San Carlos de Guatemala\\
$^3$Universidade Federal de Pernambuco (Brazil)\\
$^4$Slovak Academy of Sciences
}





\date[]{
\centering
% \includegraphics[bb = 93 648 515 770,clip, width=.8\textwidth]{articulo_2012}
% \small Phys. Rev. Lett. {\bf 107}, 080404 (2011)
\small Not published yet
}
% \date{November 16, 2010}
% \date {Quantum Optics V\\Cozumel, November 16, 2010 \\
% \small arXiv:1006.4651}
% {{{ Avisos al principio de las secciones y subsecciones
% http://tex.stackexchange.com/questions/11100/beamer-atbeginsection-and-insertsubsectionnavigation
\AtBeginSection[] 
{
\begin{frame}[plain]
% \begin{frame}<beamer>
%     \frametitle{Outline}
    \tableofcontents[sectionstyle=show/hide,subsectionstyle=hide/show/hide]
  \end{frame}
  \addtocounter{framenumber}{-1}% If you don't want them to affect the slide number

}
% \AtBeginSubsection[] % Do nothing for \subsection*
% {
% \begin{frame}[plain]%
%   \begin{center}%
%     \usebeamerfont{section title}\insertsubsection%
%   \end{center}%
% \end{frame}
% \begin{frame}<beamer>
% \frametitle{Outline}
% \tableofcontents[currentsection,currentsubsection]
% \end{frame}
% }
% }}}
% }}}
% {{{ Colors % \setbeamercolor{section in toc}{fg=yellow,bg=white}

% Aca el texto normal y el background global
% \setbeamercolor*{normal text}{fg=yellow!20!white,bg=blue!10!black}

% Los titulos
% \setbeamercolor*{structure}{fg=blue!10!white}
\setbeamertemplate{navigation symbols}{}%remove navigation symbols
% }}}
\begin{document}
\frame[plain]{% {{{ title page
\titlepage{}
}% }}}
\begin{frame}{\enges{General goal}{El objetivo general}} %{{{ 
Los canales cuánticos de un qubit puede proyectar a algunas componentes, pero 
no arbitrariamente. 


% \begin{minipage}{.48\textwidth}
\begin{center}
% Sistemas naturales, estables\\[.2cm]
Acá un dibujo en dos partes, con la esfera de Bloch en la primera indicando 
algun punto que ava a viajar al disco central, luego una segunda con solo el 
disco. Luego una segunda serie con eso, pero proyectado al eje z. 
\end{center}
\Large\pause

\begin{center}
\begin{itemize}
\item Entender un poco mejor porqué.
\item ¿Qué sucede para más qubits?
\item Salen matemáticas {\it muy} bonitas.
\end{itemize}
\end{center}

% \begin{tikzpicture}
 %nodes
%  \node[punkt] (market) {Market (b)};
 % We make a dummy figure to make everything look nice.
%  \node (t)  at (0, 0) {\includegraphics[scale=0.3]{tree.jpeg}};
%  \node (t)  at (2.3, 0) {\includegraphics[scale=0.3]{arrow3}};
%  \node (t)  at (4, 0) {\includegraphics[scale=0.1]{measure2}};
%  \node (t)  at (6, 0) {\includegraphics[scale=0.3]{arrow3}};
%  \node (t)  at (8.5, 0) {\includegraphics[scale=2.3, viewport=0 12 36 25, clip]{emc6}};

%  \node (t) {\includegraphics[width=.3\textwidth, viewport=65 90 310 330, clip]{cubo_o.pdf}};
%  \node (o) at (-1.3, -.8) {\tiny $\arrow$};
%  \node (o) at (-1.3, .9) {\tiny $\rho + \epsilon e_2$};
%  \node (o) at (.5, -1.7)  {\tiny $\rho+ \epsilon e_1$};
%  \node (o) at (0, 0)  {\tiny $\rho+ \epsilon e_3$};
% \end{tikzpicture}
% \end{center}
% Naturaleza -> Mediciónes -> Fisica. 



% Las observaciones en física con frecuencia son mediciones. 
\end{frame}% }}}
\begin{frame}{\enges{Outline}{Estructura de la charla}} %{{{ 
\hspace{1cm}\begin{minipage}{.85\textwidth}\large
\tableofcontents
\end{minipage}
\end{frame}% }}}
\section{\enges{Some tools}{Algunas herramientas matemáticas}} %{{{  
\begin{frame}{\enges{Quantum channels}{Canales cuánticos}} %{{{ 

\large 
\enges{
A quantum channel ($\Lambda$) generalizes conveniently unitary 
evolution, in a similar way as a density matrix generalizes a ket.}
{Un  canal cuántico ($\Lambda$) generaliza convenientemente la evolución
unitaria, de manera similar a como una matriz de densidad generaliza un ket.}
\\[.3cm]

\enges{Quantum channels can describe:}{Los canales pueden describir:}
\begin{itemize}
\item \enges{Unitary evolution}{Evolución unitaria} $\Lambda(\rho) = U \rho  U^\dagger  $
\item \enges{Reduced dynamics}{Dinámica reducida}
     $\Lambda(\rho) =\tr_\env \left[ U \rho \otimes \rho_\env U^\dagger \right] $
\item \enges{Decoherence}{Decoherencia}
\item \enges{Measurements}{Mediciones (sin postselección)}\\[.3cm]
\end{itemize}


\pause
% \Large 
\enges{Quantum channels are linear operators acting on the spaces of operators, i.e.
superoperators.}{Son operadores lineales sobre el espacio de operadores
(superoperadores).}
\pause
\enges{This implies that they have spectrum, eigenvalues, symmetries, etc.}{Esto
implica que tienen espectro, eigenoperadores, simetrías, etc.}

\end{frame} % }}}

\begin{frame}{\enges{Quantum channels}{Canales cuánticos}} %{{{ 
Algunas propiedades matemáticas como que mapean matriz de densidad en matrices
que preservan traza, que son lineales y que son completamente positivos. 

Decir que la completa positividad se traduce en un reordenamiento de los indices. 
\end{frame} % }}}

\begin{frame}{\enges{Quantum channels}{Canales cuánticos de un qubit}} %{{{ 
Ejemplos de como el canal de un qubit da algo bueno en un caso y algo malo en otro. 
\end{frame}

% }}}
% }}}
\section{Mapeos y canales PCE} %{{{  
\begin{frame}{PCE de un qubit y sus diagramas} %{{{ 
Definicion de un mapeo PCE para un qubit

Mostrar que los mapeos anteriores son mapeos y que algunos son y que no son 

Mostrar los diagramas para 1 qubit, decir cuales son y cuales no son. 

Definicion para n qubits. 

Diagramas para representar a los qubuts
Ejemplos para un qubit. 
\end{frame}
% }}}
\begin{frame}{PCE de n qubits } %{{{ 

Definicion para n qubits. 

Diagramas para representar a los qubuts de dos qubits
Ejemplos para un qubit. 
\end{frame}
% }}}
\begin{frame}{Ejemplos y reglas} %{{{ 

Mostrar los generadores, mostrar que solo se admiten cosas de $2^k$ componentes
regla de espejo y otras cosas mas.
\end{frame}
% }}}
% }}}
\section{Solución del problema} %{{{  
\begin{frame}{Planteamiento del problema} %{{{ 
Escribir acá el problema, de forma general
\end{frame}
% }}}
\begin{frame}{Diagonalizacion de la matriz de Choi} %{{{ 
\end{frame} % }}}
\begin{frame}{Un espacio vectorial bonito} %{{{ 
Planteamiento de coso como un espaico vectorial. Inversion de la matriz A, 
\end{frame} % }}}
\begin{frame}{Mas del espacio vectorial} %{{{ 
\end{frame} % }}}
\begin{frame}{Algunas consecuencias} %{{{ 
Conescuencias, como el numero de elementos del bicho ese 
cuantos canales hay
regla del arco iris.
\end{frame} % }}}
\begin{frame}{} %{{{ 
\end{frame} % }}}
% }}}
\section{Los generadores y decoherencia} %{{{  
\begin{frame}{Explicación intuitiva} %{{{ 
Aca podemos mostrar algunos ejemplos, pero mostrar que no son unicas
\end{frame} % }}}
\begin{frame}{Perspectiva desde los espacios vectoriales} %{{{ 
\end{frame} % }}}
\begin{frame}{Simetrías presentes} %{{{ 
Las regals que encontramos de los generadores
\end{frame} % }}}
\begin{frame}{Clasificación de los generadores} %{{{ 
Cada generador lo podemos poner de una forma única.
\end{frame} % }}}
\begin{frame}{Decoherencia generalizada} %{{{ 
Operadores de Kraus para decoherencia
\end{frame} % }}}
% }}}
\section{Conclusiones} %{{{  
\begin{frame}{Conclusiones} %{{{ 

\end{frame} % }}}
% }}}
\section{Adendum} %{{{  
\begin{frame}{Sistemas de $d$ niveles} %{{{ 
Mencionar las matrices encuestion que ayudan, mencionar que ya no forman un campo
y las matemáticas son mas complicadas

\end{frame} % }}}
% }}}
\end{document}
\begin{frame}{} %{{{ 
\end{frame} % }}}
\begin{frame}{} %{{{ 
\end{frame} % }}}
\begin{frame}{} %{{{ 
\end{frame} % }}}
\begin{frame}{} %{{{ 
\end{frame} % }}}


Quizá unos 30 slides. 

\section*{\enges{Introduction}{Matriz de densidad}} %{{{  
\begin{frame}{\enges{Motivation}{Motivación}} % {{{
\begin{itemize}
\item 
\enges{Consider a many-body system and an imperfect detector.}{
Sistema de muchos cuerpos con mediciones imperfectas. }
\item 
\enges{The measurement apparatus sometimes doesn't measure the desired particles.}
{Los aparatos de medida equivocan la partícula.}
\end{itemize}

\begin{center}
\includegraphics[width=4cm, viewport= 560 80 860 270, clip=true]{IonDiagram.png}\\
% \includegraphics[width=4cm, viewport= 210 80 560 270, clip=true]{IonDiagram.png}\\
% https://iontrap.physics.indiana.edu/research.html
\tiny Schematic overview of a 2D ion trap quantum simulation. (Richerme's group in IU)
\end{center}


\begin{itemize}
\item It might even not be able to resolve a single particle.
\item \pause
\enges{Provide a general description not considering the measuring devices, but 
rather the states.}{
Descripción general, no de instrumentos sino de estados, al realizar dichas mediciones.
}
\item \pause
% Ver si esto genera un canal válido y proveer la teoría apropiada para ello. 
\enges{Provide a framework in terms of quantum channels.}{
Enmarcar en una teoría general de canales cuánticos. }
% \item \pause
% \enges{
% Quantum tomography requires an exponential amount of measurements}{
% Tomografía toma una cantidad exponencial de medidas.}
% \item \pause
% \enges{The state will be effectevly 
% Ver que estado podemos medir al realizar dicha medición.
\end{itemize}

% The number of independent measurements that can be done on quantum systems scale rapidly as one grows the dimension of the systems . This introduces technical problems and inviability to do tomography of large quantum systems, as well lacking of fine control of the observables 

\end{frame} % }}}
% \begin{frame}{Technical reminders} % {{{
% 
% \begin{itemize}
%   \setlength\itemsep{1em}
% \item
% The density matrix $\rho$ is an operator that represents a state in which
% \alert{all} measurable information is encoded (similar to the state vector).
% \pause
% \item
% It is used often in quantum information and statistical mechanics. 
% \pause
% \item
% The measurable quantities, expected values, are calculated as $\tr \hat M \rho$
% \pause
% \item
% Quantum channels \alert{$\Lambda$} are linear operators transforming states
% $\rho$ to other states $\Lambda(\rho)$. 
% \pause
% \item
% They are used to describe open quantum systems or uncertain quantum systems. 
% \end{itemize}
% \end{frame} % }}}
% }}}
\section{\enges{Fuzzy measurements}{Mediciones borrosas}} %{{{  
\begin{frame}{\enges{Two-qubit fuzzy measurements}{Medición borrosa para dos qubits}} % {{{
\enges{For two qubits one wishes to measure}{
Para dos qubits, queremos medir}
$$\alert{\sigma_{ij}} = \sigma_i \otimes \sigma_j,\qquad i,j=0,1,2,3,$$
or linear combinations of it, 
e.g. $\sigma_x^{(1)}$, $ \sigma_z \otimes \sigma_z$ or Bell measurements. \\[.3cm]
% [covers {\it all} measurements].\\[.3cm]
\pause

Assume our apparatus is imperfect; \pause it measures 
\begin{itemize}
\item $\sigma_{ij}$, \enges{with probability}{con probabilidad} $p$, \enges{and}{y} 
\item $\sigma_{ji}$ \enges{with probability}{con probabilidad} $1-p$.
\pause
\end{itemize}
% \enges{but our apparatus measures }{
% pero nuestro aparato mide }
% $\sigma_{ij}$,
% \enges{with probability}{
% con probabilidad }
% $p$, 
% \enges{and}{y} $\sigma_{ji}$
% \enges{with probability}{
% con probabilidad }
% $1-p$.
\begin{align*}
\<\text{\enges{Expected value}{Valor esperado}}\> 
 &= \onslide<2->{p \tr \sigma_{ij} \rho + (1-p) \tr \sigma_{\alert{ji}} \rho} \\ 
 \onslide<3->{&= p \tr \sigma_{ij} \rho + (1-p) \tr S \sigma_{ij} S \rho \qquad \text{$S$ is swap}} \\ 
%  \onslide<4->{&= p \tr \sigma_{ij} \rho + (1-p) \tr \sigma_{ij} S \rho S }\\ 
 \onslide<4->{&= \tr \sigma_{ij} [p \rho + (1-p)  S \rho S ] .}
\end{align*}
\\[.2cm]
\pause  % \pause 
%\pause \pause

\enges{The effect of the imperfection of the device can be translated to $\rho$ via a quantum map:}{
Esto nos induce el mapeo }
$$
\<\text{\enges{Expected value}{Valor esperado}}\>  = \tr \left( \sigma_{ij} \Lambda_p[\rho] \right),\qquad
\Lambda_p[\rho]   = p \rho + (1-p)  S \rho S.
$$
% \enges{which takes into account exactly what we will measure.}{
% Que toma en cuenta exactamente lo que podemos medir.} 
% \alert{$p$} 
% \enges{quantifies how precise is our measuring device.}{cuantifica 
% que tan preciso es nuestro aparato de medición.}

% Consider a system of two qubits and measurements of variables with the form
% \begin{equation}
% \fuzzytwo{p}[\rho]=p \rho +(1-p) S_{01}\rho S_{01},
% \end{equation}
% where $\mcS_{01}[\rho]$ is the swap map between the two particles.


% Decir que uno a veces al medir, no mas puede medir una sola particula.
% Entonces, cuantos parametetros hay que medir? La idea de reconstruir un estado
% de una serie de observaciones. 


\end{frame} % }}}
\begin{frame}{\enges{Generalizations of fuzzy measurements}{Generalizaciones de mediciones borrosas}} % {{{
\enges{This idea can be generalize to several physical situations:}{
Esto lo podemos generalizar a varias situaciones físicas:}\\[.3cm]
\pause
\begin{tabular}{lc}
$\bullet$ 
\enges{Spin chain}{Cadena de espines }
&  \\
\raisebox{0.99cm}
{{\color{white} $\bullet$} $ \Lambda_p[\rho] = p \rho + \frac{1-p}{n} \sum_i S_{i,i+1}[\rho] $} &  
% \raisebox{1cm}
{\includegraphics[width=0.2\textwidth, clip, viewport=70 80 340 250]{colorchain}}
\\[.0cm]
\pause
$\bullet$ 
\enges{Two dimensional systems}{Sistema bidimensional}
&  \\
\raisebox{0.99cm}
{{\color{white} $\bullet$} $ \Lambda_p[\rho] = p \rho + \frac{1-p}{n'} \sum_{i,j}
(S_{i,j;i,j+1}[\rho] + S_{i,j;i+1,j}[\rho]) $} &  
% \raisebox{1.5cm}
{\includegraphics[width=0.25\textwidth]{SpinGrid}}
\\[.0cm]
\pause
$\bullet$
\enges{Generic imperfect detector}{Todos contra todos}
&  \\
\raisebox{0.99cm}
{{\color{white} $\bullet$} 
$ \Lambda_p[\rho] = p \rho + \sum_{i} p_i P_{i}[\rho] $,
$P_i$ a $n$-permutation} &  
% \raisebox{1.5cm}
{\includegraphics[width=0.17\textwidth]{Nucleus}}
% {\includegraphics[width=0.20\textwidth, clip, viewport=85 10 150 64]{CABFinal}}
%  \includegraphics[scale=1.3]{CABFinal}
% \raisebox{1cm}{\includegraphics[width=0.2\textwidth, clip, viewport=70 80 340 250]{colorchain}}
\\
 &  
\end{tabular}
% \begin{itemize}
% \item 
% 
% 
% \item 
% Generalización a un sistema bidimensional, de fuzzy
% 
% 
% Generalizacion a todos con todos, de Fuzzy
% \end{itemize}
\end{frame} % }}}
% }}}
\section{\enges{Characterization of the channels}{Caracterización de los canales}} %{{{  
\begin{frame}{\enges{Fidelity distribution}{Distribución de fidelidad,} $f$} % {{{
\enges{The fidelity measures distance between states}{La fidelidad mide distancias }
$f = |\<\psi|\phi\>|^2 = \<\psi|\phi\>\<\phi|\psi\>$. %\\[.5cm]\pause
\enges{If one of the states is pure, and the other is mixed,}{Si un estado es puro, y el otro es mixto, }
$$
f = \<\psi|\rho|\psi \> = \<\psi| \Lambda_p[|\psi\>\<\psi|]|\psi\>.
$$
\pause


\begin{center}
\includegraphics[width=6cm]{distro}
% \includegraphics[width=4cm, viewport= 210 80 560 270, clip=true]{IonDiagram.png}
\end{center}
\enges{For the map related to the spin chain, we can calculate the fidelity distribution}{
Para el mapeo de la cadena, }
$
f \sim G\left(p+\frac{1-p}{4}, \frac{1}{n} \left[ 4 p^2+\frac{(1-p)^2}{4}  \right]
\right).
$

% \begin{itemize}
% \item El mapa no rota los estados (eigenvalores reales del mapeo)
% \item Sólo hay una contracción no uniforme. 
% \item Sampleamos de manera uniforme  (medida de Haar) el conjunto de estados.
% \end{itemize}
\end{frame} % }}}
\begin{frame}{Symmetries} % {{{
Consider a generic imperfect measurement 
$ \Lambda[\rho] = p \rho + \sum_{i} p_i P_{i}[\rho] $.
\begin{itemize}
\item The channel preserves the number of excitations, e.g. 
$$P_i|000101\> = |010010\>.$$ 
It is a unitary symmetry and thus acts on bras and kets independently. 
% It provides 2 quantum numbers. 
\\[.5cm]
\pause
\item Ket-bra interchange symmetry 
$$(|i \>\< j|, \Lambda[|k \>\< l|]) = (|j \>\< i|, \Lambda[|l \>\< k|]).$$ 
% \\[.5cm]
\pause
% $(|i \>\< j|, \Lambda[|k \>\< l|]) = (|i \>\< \bar j|, \Lambda[|k \>\< l|])$. 
% \item It conserves projectors in the computational basis 
\item Partial NOT symmetry: $|3\>=|00101\> \to |\bar 3\> = |22\> = |11010\>$,
$$(|i \>\< j|, \Lambda[|k \>\< l|]) = (|i \>\< \bar j|, \Lambda[|k \>\< \bar l|]).$$\\[.5cm]  
\pause
\item Excitation intersection symmetry: 
$\gamma(13,11)=\gamma(\alert{1}10\alert{1}_2,\alert{1}01\alert{1}_2)=2$. Notice that
$$\gamma(|i\>\<j|)\left[\equiv \gamma(i,j) \right] = \gamma(P_a |i\>\<j| P_a^\dagger).$$
% other quantum number. 
\end{itemize}
% This induces a block structure 
% \begin{equation*}
% \Lambda_p = 
% \bigoplus_{j,k =0}^{ \lfloor q/2 \rfloor } \Lambda_{(j,k)}^{\text{NP } \oplus \sim 4}  
% \bigoplus_{j =0}^{\lfloor q/2 \rfloor } \Lambda_{(j,k)}^{\text{P } \oplus \sim 2} 
% \label{}
% \end{equation*}
\end{frame} % }}}
\begin{frame}{Block structure} % {{{
Consider a generic imperfect measurement device 
$ \Lambda_p[\rho] = p \rho + \sum_{a} p_a P_a \rho  P_a^\dagger $.
\begin{itemize}
\item Unitary conservation of the number of excitations
provides 2 quantum numbers $\alpha$ and $\beta$. \\[.3cm]
\pause
\item Excitation intersection symmetry provides a third number  $\gamma$, compatible
with the previous ones. We thus have blocks  $\Lambda_{(\alpha, \beta, \gamma)}$. 
\\[.3cm]
\item Ket-bra interchange symmetry 
allows to identify
$\Lambda_{(\alpha, \beta, \gamma)} = \Lambda_{(\beta, \alpha, \gamma)}$.
\\[.3cm]
\pause
% $(|i \>\< j|, \Lambda[|k \>\< l|]) = (|i \>\< \bar j|, \Lambda[|k \>\< l|])$. 
% \item It conserves projectors in the computational basis 
\item Partial NOT symmetry: 
allows to identify
$\Lambda_{(\alpha, \beta, \gamma)} = \Lambda_{(\alpha, q+1 -\beta, \gamma)}$.
\end{itemize}
\pause
% This induces a block structure 
\begin{equation*}
\Lambda = 
\bigoplus_{\alpha,\beta,\gamma =0}^{ \lfloor q/2 \rfloor } 
     \Lambda_{(\alpha,\beta,\gamma)}^{\oplus d(\alpha,\beta)}  
% \bigoplus_{j =0}^{\lfloor q/2 \rfloor } \Lambda_{(j,k)}^{\text{P } \oplus \sim 2} 
\label{}
\end{equation*}
The number of blocks is $(q+1)(q+2)(q+3)/6$.
% \begin{equation*}
% \Lambda = 
% \bigoplus_{j,k =0}^{ \lfloor q/2 \rfloor } \Lambda_{(j,k)}^{\text{NP } \oplus \sim 4}  
% \bigoplus_{j =0}^{\lfloor q/2 \rfloor } \Lambda_{(j,k)}^{\text{P } \oplus \sim 2} 
% \label{}
% \end{equation*}
\end{frame} % }}}
\begin{frame}{Spectra of superoperators} % {{{
% $\Lambda$ is a linear operator \pause $\longrightarrow$ eigenvalues and eigenvectors (sometimes) 
% \pause
% \begin{equation*}
% \Lambda_{lm;l'm'} \pause 
% =(|l\>\<m|, \Lambda(|l'\>\<m'|) 
% \pause= \tr [|m\>\<l|\Lambda(|l'\>\<m'|)] \pause = \<\<lm|\Lambda|l'm'\>\> 
% \label{}
% \end{equation*}
\begin{itemize}
\setlength\itemsep{1em}
% \item $\Lambda$ preserves hermiticity \pause $\to$ 
\item Complex eigenvalues of $\Lambda$ come in conjugate pairs 
\pause
\item $\Lambda$ is a contractive map $|\lambda| \le 1$
\pause
\item To each symmetry block corresponds an eigenvalue $\lambda=1$
\pause
\item Blocks $\Lambda_{(\alpha,\alpha,\alpha)}$ yield physical states.
\pause
\item Other blocks yield coherences (and structure to the steady state). 
\pause
% \item To each symmetry block not corresponding to states $\longrightarrow$ $\lambda=1$
\item Our operators are diagonalizable
\end{itemize}


% Hablar de las coherencias

% Hablar de los estados invariantes, realmente estados. 
% Consider a generic imperfect measurement device (as we define it). 
\end{frame} % }}}
\begin{frame}{Generic spectra of fuzzy measurements } % {{{
% \includegraphics[]{spectraAll}
\begin{itemize} 
\item Consider a random measuring 
% $\Lambda_p[\rho] = p \rho + \sum_{i} p_i P_{i}[\rho] $
$ \Lambda_p[\rho] = p \rho + \sum_{i} p_i P_i \rho  P_i^\dagger $.
\item I shall show results for a typical case 
\item Also the ensemble average
\end{itemize}
\pause
\begin{minipage}{.45\textwidth}
\begin{tikzpicture}
 %nodes
\draw[draw=white] (-2.7,-4.5) rectangle ++(5.5,5);
%  \node[punkt] (market) {Market (b)};
% \filldraw [fill=Peach, draw=black] (11.1,5) rectangle (0.3,0.3);
 % We make a dummy figure to make everything look nice.
 \node at (0, 0)  {\includegraphics[width=.99\textwidth]{spectraAll}};
 \node at (0, -2.5)  {%
\only<2>{\includegraphics[width=.99\textwidth]{DensityTalks1}}%
\only<3>{\includegraphics[width=.99\textwidth]{DensityTalks2}}%
\only<4>{\includegraphics[width=.99\textwidth]{DensityTalks3}}%
\only<5->{\includegraphics[width=.99\textwidth]{DensityTalks4}}%
};
%  \node (t) {\includegraphics[width=5cm]{spectraAll}};
%  \includegraphics[width=\textwidth]{cubo_o.pdf}};
%  \node (t) {\includegraphics[width=.3\textwidth, viewport=65 90 310 330, clip]{cubo_o.pdf}};
%  \node (o) at (-1.3, -.8) {\tiny $\rho$};
%  \node (o) at (-1.3, .9) {\tiny $\rho + \epsilon e_2$};
%  \node (o) at (.5, -1.7)  {\tiny $\rho+ \epsilon e_1$};
%  \node (o) at (0, 0)  {\tiny $\rho+ \epsilon e_3$};
\end{tikzpicture}
\end{minipage}%
\begin{minipage}{.5\textwidth} 
\begin{itemize}
\item Spectrum $\Lambda_{(1,1,\gamma)}$ $q=6$ $p=0.4$
\item Non-trivial coherences $\to$ coherent states  %(Klimov)
\item Spectral gap $\sim 1-p$
\item Non-1s rescale when changing p
\end{itemize}
\begin{itemize}
\pause
\item There is an exact real part
\pause
\item Asymptotically flat distribution
\pause
\item Symmetric complex part
\pause
\item Disc-like distribution
\pause
\item Self averaging (not shown)
% Self averaging
\end{itemize}

% 1 Muestro el espectro tipico, para una p

% Ones for symmestries with physical states
% Spectral gap $\sim 1-p$
% Non-ones just rescaling when changing p
% Self averaging
\end{minipage}%

% Hablar de las coherencias

% Hablar de los estados invariantes, realmente estados. 
% Consider a generic imperfect measurement device (as we define it). 
\end{frame} % }}}
\begin{frame}{Other fuzzy measurements} % {{{
For other interesting cases, say just 2-body errors, the spectrum is purely real. 
We look again at $\Lambda_{(1,1,\gamma)}$, $p=0.4$ for 6 qubits.\\[.5cm] 
 \pause
\begin{minipage}{.45\textwidth}
Random 2-body errors:\\
$\Lambda_p[\rho] = p \rho + (1-p) \sum_{i>j} p_{ij} S_{i,j}[\rho] $

\begin{tikzpicture}
 \node at (0, 0)  {\includegraphics[width=.99\textwidth]{gAll2}};
%  \node (t) {\includegraphics[width=5cm]{spectraAll}};
%  \node (o) at (-1.3, -.8) {\tiny $\rho$};
\end{tikzpicture}
\end{minipage}\hfill \pause
\begin{minipage}{.45\textwidth}
Random chain errors:\\
$\Lambda_p[\rho] = p \rho + (1-p) \sum_{i} p_{i} S_{i,i+1}[\rho] $
\begin{tikzpicture}
 \node at (0, 0)  {\includegraphics[width=.99\textwidth]{gChain}};
%  \node (t) {\includegraphics[width=5cm]{spectraAll}};
%  \node (o) at (-1.3, -.8) {\tiny $\rho$};
\end{tikzpicture}
\end{minipage}
 \pause
\begin{itemize}
\item Richer (more complex) structure 
\item No self averaging 
\item More ``damaging'' errors
\end{itemize}


\end{frame} % }}}
\begin{frame}{\enges{Volume contraction}{Contracción del volumen}} % {{{
\begin{itemize}
\onslide<1->{
\item 
\enges{The operator state is a vector space.}{El espacio de operadores es un espacio vectorial} $\rho \to \vec \rho$. 
}
\onslide<2->{
\item 
\enges{We build a hyper-cube with the following corners}{Construimos un cuboide con esquinas }
$\vec \rho$, $\{\vec \rho + \epsilon e_i\}_{i=1,\dots d }$ 
% \item 
% \enges{Its volume is thus}{Su volumen es entonces }
% $V_0 = \epsilon^{d}$ 
}
\onslide<3->{
\item 
\enges{We map each of the states,}{Mapeamos cada uno de los estados,} 
$\vec \Lambda[\rho]$, $\{\vec \Lambda[\rho + \epsilon e_i]\}$ 
}
\onslide<4->{
\item 
\enges{We find that the volume is }{Encontramos que su volumen es} $V_\Lambda = \epsilon^{d} \det |\vec \Lambda[e_i]|$ 
}
\end{itemize}

\begin{center}
\onslide<2->{
\begin{minipage}{.25\textwidth}
\begin{tikzpicture}
 %nodes
%  \node[punkt] (market) {Market (b)};
 % We make a dummy figure to make everything look nice.
 \node (t) {\includegraphics[width=\textwidth]{cubo_o.pdf}};
%  \node (t) {\includegraphics[width=.3\textwidth, viewport=65 90 310 330, clip]{cubo_o.pdf}};
 \node (o) at (-1.3, -.8) {\tiny $\rho$};
 \node (o) at (-1.3, .9) {\tiny $\rho + \epsilon e_2$};
 \node (o) at (.5, -1.7)  {\tiny $\rho+ \epsilon e_1$};
 \node (o) at (0, 0)  {\tiny $\rho+ \epsilon e_3$};
\end{tikzpicture}
\end{minipage}
}
\hspace{.5cm}
\onslide<3->{
\begin{minipage}{.25\textwidth}
\begin{tikzpicture}
 %nodes
%  \node[punkt] (market) {Market (b)};
 % We make a dummy figure to make everything look nice.
 \node (t) {\includegraphics[width=\textwidth]{cubo_f.pdf}};
%  \node (t) {\includegraphics[width=.3\textwidth, viewport=65 90 310 330, clip]{cubo_o.pdf}};
 \node (o) at (-1.3, -.7) {\tiny $\Lambda[\rho]$};
 \node (o) at (-.8, 1.0) {\tiny $\Lambda[\rho + \epsilon e_2]$};
 \node (o) at (-.3, -1.0)  {\tiny $\Lambda[\rho+ \epsilon e_1]$};
 \node (o) at (0.2, -.4)  {\tiny $\Lambda[\rho+ \epsilon e_3]$};
\end{tikzpicture}
\end{minipage}
}
\end{center}
% \hspace{.5cm}
% \onslide<4->{
% \begin{minipage}{.30\textwidth}
%  \includegraphics[width=\textwidth]{vs.pdf}
% \end{minipage}
% \pause
% \pause
% \pause
% \pause
% \begin{itemize}
% \item No matter how small the imperfection, measuring a state is exponentially difficult in $n$
% \pause
% \item There is an invariant set that coincides (for many detectors) with multi-particle coherent states. 
% \end{itemize}
% }

\end{frame} % }}}
\begin{frame}{\enges{Volume contraction}{Contracción del volumen}} % {{{

In our case, we can estimate this contraction as
$$
\sigma = \frac{V_{\rm observable}}{V_{\rm total}} 
\pause
= \det \Lambda 
= \prod_i \lambda_i 
\pause
\approx p^{\left[2^{2q} -\frac{(q+1)(q+2)(q+3)}{6}\right]}
$$
\pause

\begin{center}

\includegraphics[scale=0.6]{gamma_sigmaFinal.pdf}
% \begin{tikzpicture}
%  %nodes
% %  \node[punkt] (market) {Market (b)};
%  % We make a dummy figure to make everything look nice.
%  \node (t) {\includegraphics[width=.4\textwidth]{shrink_density-crop.pdf}};
% %  \node (t) {\includegraphics[width=.3\textwidth, viewport=65 90 310 330, clip]{cubo_o.pdf}};
%  \node (o) at (-2.6, -.0) {\rotatebox{90}{$\log \log \sigma$}};
%  \node (o) at (2.5, -.7) { $q$};
%  \node (o) at (1.5, -.9)  {\tiny {\color{red} $\bullet$} $p=0.10$};
%  \node (o) at (1.5, -1.1)  {\tiny {\color{blue} $\bullet$} $p=0.51$};
%  \node (o) at (1.5, -1.3)  {\tiny {\color{green} $\bullet$} $p=0.99$};
% \end{tikzpicture}
\end{center}
\pause
\begin{itemize}
\item Single channel realization.
\item The observable space, given a detector with small imperfections,
is doubly-exponentially suppressed in the number of particles. 
\item This is a purely many-body effect. 
\item Smalles positive number I have ever calculated: $10^{-348 788}$
\end{itemize}
% \hspace{.5cm}
% \onslide<4->{
% \begin{minipage}{.30\textwidth}
%  \includegraphics[width=\textwidth]{vs.pdf}
% \end{minipage}
% \pause
% \pause
% \pause
% \pause
% \begin{itemize}
% \item No matter how small the imperfection, measuring a state is exponentially difficult in $n$
% \pause
% \item There is an invariant set that coincides (for many detectors) with multi-particle coherent states. 
% \end{itemize}
% }

\end{frame} % }}}
\begin{frame}{Uhlman's theorem} % {{{
\begin{itemize}
\item 
Consider a hermitian operator $A$, probabilities $p_i$ and unitaries $U_i$.
\item  \pause
Define $B=\sum_i p_i U_i A U_i^\dagger$. 
\item  \pause
$B$ is majorized by $A$, meaning that for any convex function $f$ of the spectra, 
$f(B) \le f(A)$. 
\end{itemize}
\pause
\begin{itemize}
\item 
Uhlman's theorem fits our map $ \Lambda_p[\rho] = p \rho + \sum_{i} p_i P_{i}\rho P_i^\dagger $,

\item 
\pause

Important and meaningful quantities, such as purity, entanglement, and most
information measures are convex, so they diminish under {\it any} fuzzy
measurement, as one would expect from any intuitive idea of a ``fuzzy measurement''.

\end{itemize}

% \hfill
% \vfill
% See poster by A. Rosado. 

\end{frame} % }}}
% }}}
\section{Coarse graining} %{{{  
\begin{frame}{\enges{Motivation}{Motivación del coarse graining}} % {{{
\begin{minipage}{.48\textwidth}
\begin{itemize}
\item 
\enges{We cannot measure everything}{No podemos medir todo}
\item 
\enges{We ignore internal uninteresting structure. }
{Ignoramos estructura interna }
% \item 
% Queremos ignorar algunos aspectos superfluos. 
\end{itemize}
\end{minipage}
\begin{minipage}{.48\textwidth}
\includegraphics[width=\textwidth]{cg_molecula}\\
 \tiny
An all-atom representation and MARTINI coarse grain representation of a DPPC bilayer
(Christopher Rowley)
% https://commons.wikimedia.org/wiki/File:All_atom_vs_coarse_grain_DPPC_bilayer.png
\end{minipage}


\begin{center}
\enges{There are no good general theoretical tools to attack this.}{
No hay herramientas teóricas  generales desarrolladas para eso.}
\end{center}

% En mecanica cuantica lo queremos hacer porque puede dar simplicidad al sistema
% porque simplemente no podemos medir. 
% En general la idea de escalar y de hacer zoom out es clave. 
% Como hacerlo?




% Explicación con ejemplos. Muy lenta. 

% Consider now the situation when the measurement apparatus cannot resolve the
% whole system, thus measuring only partial parts of the system. This can be
% thought similar as taking a picture, ,
% the camera is exposed to a big number of light sources, but the detectors have
% determined time and relaxation windows. Therefore the camera just access a
% partial part of light sources, generating an approximate image of what is
% actually there. To illustrate this in the quantum realm take again a two qubits
\end{frame} % }}}
\begin{frame}{\enges{The map}{El mapeo en cuestión}} % {{{
\enges{The main observation stems from the assumption that from 
a set of particles, we can only measure one, at random.}{
La observación principal se basa en suponer que al medir un sistema, 
compuesto de dos (o más partes), sólo medimos una de ellas. }

\begin{align*}
\<\text{\enges{Expected value}{Valor esperado}} \> 
 &= \onslide<2->{\frac12 \tr (\sigma \otimes \openone \rho_{AB} ) + \frac12 \tr (\openone \otimes \sigma  \rho_{AB})} \\ 
 \onslide<3->{&= \frac12 \tr \sigma \rho_A + \frac12 \tr \sigma \rho_B } \\ 
 \onslide<4->{&=  \tr \sigma \frac{\rho_A +\rho_B}{2}} \\ 
\end{align*}

% \begin{equation}
% \rho_\text{eff}=p\rho_1+(1-p)\rho_2.
% \end{equation}
% For general states it straightforward (see appendix) to prove that the linear channel that do this is 
\onslide<5->{
\begin{equation*}
\cgtwo{p}[\rho]=\frac{\tr_B \rho +\tr_A \rho}{2}
\end{equation*}
}
% A central fact here is that this channel is trivially related with \fuzzytwo{p}
% \begin{equation}
% \cgtwo{p}[\rho]=\tr_2 \fuzzytwo{p}[\rho].
% \label{eq:fuzzy_cg_relation}
% \end{equation}
% This equation shows how a simple coarse-graining description can be obtained by performing a fuzzy measurement and reducing the degrees of freedom of the system.
\end{frame} % }}}
\begin{frame}{\enges{A couple of examples}{Ejemplos de como se transforman}} % {{{

% \newcommand{\GHZ}{\text{GHZ}}
% \newcommand{\ket}[1]{{| #1 \rangle}}
% \newcommand{\bra}[1]{{\langle #1 \vert}}
% \newcommand{\proj}[2]{{\vert #1 \rangle \langle #2 \vert}}
% \newcommand{\cgtwo}[1]{\ensuremath{\Lambda_#1^\text{cg}} }
\begin{itemize}
\item 
$|\GHZ(n)\> = \frac{1}{\sqrt{2}} 
\left( \ket{0}^{\otimes n} + \ket{1}^{\otimes n} \right) $
\pause
\begin{equation*}
\cgtwo{.} \left[|\GHZ(n)\>\<\GHZ(n)|\right]=
\frac{1}{2}\left( |0\>\<0|^{\otimes m}+ |1\>\<1|^{\otimes m} \right),\, m<n,
\end{equation*}
which has no quantum correlations. 
\pause
\item $|\text{W}(n)\> = \frac{1}{\sqrt{n}} \sum_{k=0}^{n-1}
|10\dots0\> + |01\dots0\> + \dots + |00\dots1\> $ 
\pause
\begin{equation*}
\cgtwo{.}[|\text{W}(n)\>\<\text{W}(n)|]= 
\frac{m}{n} |\text{W}(m)\> \<\text{W}(m)| + 
\frac{n-m}{n} |0\>\<0|^{\otimes m}
\end{equation*}
attenuated quantum correlations. 
\pause
\item $|\text{Bell}\>=(|00\>+|11\>)/\sqrt{2}$, twice: \raisebox{-.5cm}{\includegraphics[width=1.0cm]{bell1}}\\
\pause
% , and $\bigotimes_{i=1}^4 \mcH_{i}$\\
% $|\text{Bell}\> \otimes |\text{Bell}\>$: 1 and 2 [3 and 4] are entangled
\begin{center}
\raisebox{-1.0cm}{\includegraphics[width=2cm]{bell2}} $\longrightarrow \frac{ \openone}{4}$
\hspace{1cm}
\pause
\raisebox{-1.0cm}{\includegraphics[width=2cm]{bell3}} $\longrightarrow 
\rho_{\text{Werner}}^{p^2 + (1-p)^2}$
\end{center}
% \begin{equation*}
% % \begin{multline*}
% \cgtwo{{p,(1,2)}}\otimes \cgtwo{{p,(3,4)}}[(|\text{Bell}\>\<\text{Bell}|)^{\otimes 2}]=
% \frac{\openone}{4},
% % \frac{1}{4} \big( \prj{0}\otimes \openone  + 
% %  \mele{01}{10} \\ + \mele{10}{01} + \openone \otimes \prj{0}\big)
% % \prj{11} \right), \\
% % (\cgtwo{1/2} \otimes \cgtwo{1/2}) (\mele{W}{W}) &= \frac{1}{4} \left(2 \prj{00} + \prj{01} + \mele{01}{10} + \mele{10}{01} + \prj{10} \right).
% % \end{multline*}
% \end{equation*}
% whereas 
% \begin{equation*}
% % \begin{multline*}
% \cgtwo{{p,(1,3)}}\otimes \cgtwo{{p,(2,4)}}[(|\text{Bell}\>\<\text{Bell}|)^{\otimes 2}]=
% \rho_{\text{Werner}}(p^2 + (1-p)^2)
% % \frac{1}{4}
% % \begin{pmatrix}
% % 1+\alpha & 0 &0&2\alpha \\
% % 0 & 1-\alpha & 0 & 0 \\
% % 0 & 0 & 1-\alpha & 0 \\
% % 2\alpha &0 &0  &1+\alpha
% % \end{pmatrix}
% \end{equation*}

\end{itemize}

\end{frame} % }}}
\begin{frame}{\enges{Entanglement dynamics}{Ejemplos de como se transforman}} % {{{
% En la de la mitad en la que uno solo se puede equivocar de una manera y el otro en la manera de la izquierdo. p=0.75


% Ok, que quiero mostrar. 
% Que el coarse graining tiene sentido, y que puede predecir experimentos. 
$$
\hat{H}=-J_{ex}\sum_j(\hat{\sigma}_j^+\hat{\sigma}_{j+1}^- + \hat{\sigma}_j^-\hat{\sigma}_{j+1}^+),a \qquad |\psi\> = |\dots00100\dots\>
$$
\pause
\begin{minipage}{.48\textwidth}
% \enges{Example}{Ejemplo }
% Excitation in the central spin. 
\includegraphics[width=\textwidth]{CG-Entanglement/siteEPlot}

% \includegraphics{siteEPlot.pdf}
% \caption{{\bf Entanglement evolution.} Concurrence $C(\rho^{(2)}_{A,-A})$ for
% symmetric sites with $A=1,2,3,4$ for the single spin impurity dynamics as a
% function of time in units of $\hbar/J_{ex}$. The concurrence
% Eq.\eqref{eq:microC} (red continuos line) of the exact reduced state
% $\rho^{(2)}_{AB}$  is shown along the concurrence of the coarse grained reduced
% states $\rho^{CG1(2)}_{AB}$ and $\rho^{CG2(2)}_{AB}$, Eqs.\eqref{eq:CG1C} and
% Eqs.\eqref{eq:CG2C}, respectively (blue dashed line and green dotted line).}
\begin{itemize}
\item Entanglement waves
\item Entanglement decreases 
\item \enges{Tendency remains}{Permanece la tendencia}
% \item \enges{What is the maximum observable entanglement?}{¿Cual es el máximo $C$ observable?}
\end{itemize}
% \alert{TALK ABOUT ENTANGLENT}
\end{minipage}\hfill\pause
\begin{minipage}{.48\textwidth}
\includegraphics[width=\textwidth, page=3, clip, viewport=50 590 300 695]{bibliografia/entanglement_waves}\\[-.4cm]
{\tiny Fukuhara,\dots, Bloch, Gross: Phys. Rev. Lett. 115, 035302 }\\[.5cm] \pause
\includegraphics[width=\textwidth, clip, viewport=20 0 400 142]{KCWPlot_CP}\\[-.2cm]
{\tiny FM $p=0.9$ }
% \includegraphics[width=6cm]{chainl1}\\
% \includegraphics[width=6cm]{Concl1}
\end{minipage}
\end{frame} % }}}
\begin{frame}{Additional information} % {{{
The mathematical structure of coarse graining is similar, as
far as we can tell, 
$$\cgtwo{p}=\tr_A \circ \Lambda^{\text{FM}}.$$

\vspace{1cm}
However, some subtleties arise, as we now deal with non-diagonal operators. 
\end{frame} % }}}
% }}}
\section*{Conclusions} %{{{  
\begin{frame}{Conclusions} % {{{
\begin{itemize}
% \enges{ }{asdfadf 
\item We provided a pragmatic framework for coarse graining and fuzzy measurements
based on what we can physically observe. 
% \item We provided a pragmatic fuzzy measurement framework, closely connected to 
% coarse graining. 
\item The framework can be adapted to several physical situations. 
\item {\it Arbitrarily} small errors in identification have {\it very}
damaging errors.
% \item We are now calculating interesting things about the map. 
% \item Some of these maps have a semigroup structure. 
\item The many-particle limit selects a non-trivial set of states.
% \item We think that they do not rotate (what does this mean?)
% \item Does coarse graining is always fuzzy + partial trace?
% \item We want to study inclusion relations of the map families. 
\end{itemize}

TODO:
\begin{itemize}
\item Characterize well the invariant set and relate to classical limit.
\item Can we define coarse grained dynamics?
\item Similar framework when we apply operations on the wrong particles?
\end{itemize}


\begin{tikzpicture}
 \node at (5, 0)  { \includegraphics[width=.45\textwidth]{experiment_measure} };
 \node at (0, 0)  { \includegraphics[width=.45\textwidth]{experiment_measure} };
\draw[line width=1mm,draw=green] (-.7,-1.2) rectangle ++(3,2.1);
\draw[line width=1mm,draw=green] (3,-1.2) rectangle ++(4.2,2.4);
%  \node (t) {\includegraphics[width=5cm]{spectraAll}};
%  \node (o) at (-1.3, -.8) {\tiny $\rho$};
\end{tikzpicture}
% \end{minipage}\hfill \pause
% \begin{minipage}{.45\textwidth}
% Random chain errors:\\
% $\Lambda_p[\rho] = p \rho + (1-p) \sum_{i>j} p_{ij} S_{i,j}[\rho] $
% \begin{tikzpicture}
%  \node at (0, 0)  {\includegraphics[width=.99\textwidth]{gChain}};
% %  \node (t) {\includegraphics[width=5cm]{spectraAll}};
% %  \node (o) at (-1.3, -.8) {\tiny $\rho$};
% \end{tikzpicture}
\end{frame} % }}}

% }}}

