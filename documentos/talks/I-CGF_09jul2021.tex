%%%%% slides template
% Giovanni Ramirez (ramirez@ecfm.usac.edu.gt)
% 20200828
% Require: images/ecfmByN.png images/usacByN.jpg
%
%
% Use this option to print notes only
% \documentclass[xcolor=dvipsnames,handout,notes=only]{beamer}%
%
% Use this option to print notes and slides
% \documentclass[xcolor=dvipsnames,handout,notes=show]{beamer}%
%
% Use this option to get a printer-friendly version
% \documentclass[xcolor=dvipsnames,handout]{beamer}%
%
% Unse this option to get slides only
\documentclass[xcolor=dvipsnames,presentation]{beamer}%
%
%%%%%


%%%%% theme and coloring
% a list of themes can be found here https://hartwork.org/beamer-theme-matrix/
% I like a simple Malmoe with some modifications and color-schemes like
% dolphin for a white background, beetle for a gray background or dove for
% plain slides
%
% theme definition
\usetheme{Malmoe}
%
% definition of the presentation mode: full slides
\mode<presentation> {
% coloring for slides other options: beetle, dove
  \usecolortheme{dolphin}
% to include a slide with the table of contents
 \AtBeginSection[] {
  \begin{frame}
    \tableofcontents[currentsection]
  \end{frame} }
}
%
% definition of the handout mode: printer-friendly
\mode<handout>{
% coloring: dove is white
  \usecolortheme{dove}
% this is to print 2 slides in 1 page, other values are allowed
  \usepackage{pgfpages}
  \renewcommand\textbullet{\ensuremath{\bullet}}
  \pgfpagesuselayout{2 on 1}[letterpaper,border shrink=5mm]
% set the fontsize of the notes
  \setbeamerfont{note page}{size=\scriptsize}
% disable the slide with the table of contents
  \AtBeginSection[]{}
}
%%%%%

%%%%% personalisation
% clear all headers
\setbeamertemplate{headline}{}
% disable navigation buttons
\setbeamertemplate{navigation symbols}{}
% configure the blocks used for equations, theorems, etc
\setbeamertemplate{blocks}[rounded][shadow=true]
% clear the footline
\setbeamertemplate{footline}{}
% set the footline
\setbeamertemplate{footline}{
  \hbox{%
% left block with the shortname, it is set in the \author command
    \begin{beamercolorbox}[wd=.40\paperwidth,ht=2.25ex,dp=1ex,center]%
      {author in head/foot}%
      \usebeamerfont{author in head/foot}\insertshortauthor
    \end{beamercolorbox}%
% center block with the short title, it is set in the \title command
    \begin{beamercolorbox}[wd=.50\paperwidth,ht=2.25ex,dp=1ex,center]%
      {title in head/foot}%
      \usebeamerfont{title in head/foot}\insertshorttitle
    \end{beamercolorbox}%
% right block with the number of slides
    \begin{beamercolorbox}[wd=.10\paperwidth,ht=2.25ex,dp=1ex,right]%
      {date in head/foot}%
      \usebeamerfont{date in head/foot}
      \insertframenumber{} / \inserttotalframenumber\hspace*{1em}
    \end{beamercolorbox}}%
  \vskip0pt%
}
%%%%%

%%%%% packages
%
% language, decimal point mark
\usepackage[spanish,es-nodecimaldot]{babel}%
%
% to use only T1 scalable fonts
\usepackage[T1]{fontenc}%
\let\Tiny=\tiny% this is a proper configuration of the tiny font size
%
% to use UTF coding, other options latin1
\usepackage[utf8]{inputenc}%
%
\usepackage{array}
% special math symbols
\usepackage{latexsym,amsfonts,amsmath}%
%
% to include graphics
\usepackage{graphicx}%
% path to images
\graphicspath{ {../images/} }
%
% to use tiks: nice to draw diagrams
\usepackage{tikz}%
\usetikzlibrary{arrows,shapes,positioning}
%
% to use SI units and physics notation
\usepackage{physics}%
\usepackage{multirow}
\usepackage{xcolor}
%
% to use hypenation
\usepackage{ragged2e}%
\let\raggedright=\RaggedRight%
%
\usepackage{dsfont}
% to configure the hyperref to set metadata in the PDF file
\hypersetup{%
%pdfpagemode=FullScreen,% to open the file in fullscreen
breaklinks={true},% to allow links in more than one line
%pdfstartview={Fit},% to set the slide to fit the screen
%pdfpagemode=FullScreen,% to set it to full screen
pdfauthor={Giovanni Ramirez Garcia (ramirez@ecfm.usac.edu.gt)},% author name
pdftitle={Quantum Optics aplied to Quantum Computing},% title of the talk
pdfsubject={Keynote for the talk at Lat0 group},% subject of the talk
pdfkeywords={quantum optics, quantum computing, quantum algorithms, quantum
  circuits}% keywords to categorise the contents
}
%%%%%

%%%%% Talk data
%
%title
\title[Canales cuánticos PCE]% this is the short title
{\bf Mapeos proyectivos en sistemas de qubits}% this is the long title
%
% author
\author[José Alfredo de León (ECFM-USAC)]% this is the short name
{\bf José Alfredo de León${}^1$, 
Carlos Pineda${}^2$,\\
David Dávalos${}^2$, 
Alejandro Fonseca${}^3$}% this is the long name
%
% affiliation
\institute{${}^1$Escuela de Ciencias Físicas y Matemáticas, U. de 
San Carlos de Guatemala\newline%
${}^2$Instituto de Física, U. Nacional Autónoma de México\newline
${}^3$Departamento de Física, U. Federal de Pernambuco}%
%
% date and information about the conference
\date{I Congreso Guatemalteco de Física\\09 de julio de 2021}
%%%%%


\begin{document}
% % % % % % % % % % % % % % % % % % % % % % % % % % % % % % % % % % % % % %
% Title and Outline
% % % % % % % % % % % % % % % % % % % % % % % % % % % % % % % % % % % % % %
\begin{frame}[plain]
  \titlepage %
  \begin{tikzpicture}[x=1mm,y=1mm,overlay,remember picture]
    \pgftransformshift{\pgfpointanchor{current page}{center}}
%    \node[inner sep=0pt] (usac) at (-35,-30) %
%    {\includegraphics[height=12mm]{logos/ecfmByN}};
%    \node[inner sep=0pt] (usac) at (0,-30) %
%    {\includegraphics[height=12mm]{logos/ifunam}};
%    \node[inner sep=0pt] (ecfm) at (35,-30) %
%    {\includegraphics[height=12mm]{logos/u-pernambuco}};
  \end{tikzpicture}
\end{frame}

\begin{frame}{Outline}
  \tableofcontents[hidesubsections] 
\end{frame}


% % % % % % % % % % % % % % % % % % % % % % % % % % % % % % % % % % % % % %
% Introduction and Motivation
% % % % % % % % % % % % % % % % % % % % % % % % % % % % % % % % % % % % % %
\section{Introducción}
\label{sec:Intro}

\begin{frame}{Motivación}{¿por qué estudiar canales cuánticos?}
	\only<1>{
	Sistema ideal:
	\vspace{-.5cm}
	\begin{columns}
	\begin{column}{.5\textwidth}
	\begin{center}
		\includegraphics[width=6cm]{sistema_cerrado.pdf}
	\end{center}
	\end{column}
	\hfill
	\begin{column}{.45\textwidth}
	\begin{center}
		\includegraphics[width=5cm]{dynamics_closed}
	\end{center}
	\end{column}
	\end{columns}
	}
%\end{frame}
%
%\begin{frame}{Sistema abierto}
	\only<2>{
	Sistemas reales: abiertos
	\vspace{-.1cm}
	\begin{center}
		\includegraphics[width=5cm]{sistema_abierto01.pdf}
		\hfill
		\includegraphics[width=5cm]{sistema_abierto02.pdf}
	\end{center}
	\begin{center}
		\hspace{.5cm}
		\includegraphics[width=5cm]{dynamics_open}
	\end{center}
  }
  
%  \only<2>{
%	\vspace{-.1cm}
%	\begin{center}
%		\includegraphics[height=7cm]{sistema_abierto02.pdf}
%	\end{center}
%  }    
  
%  \only<1>{
%  \vspace{-.6cm}
%  Sin embargo, los sistemas cuánticos reales son sistemas 
%  abiertos a interacciones con su entorno.
%  
%  \vfill
%	}
%	
%	\only<2>{
%  \vspace{-.6cm}
%  \alert{
%  Sin embargo, los sistemas cuánticos reales son sistemas 
%  abiertos a interacciones con su entorno.
%  }
%  
%  \vfill
%	}

\end{frame}


\section{Formalismo}
\begin{frame}{Matriz de densidad}
{¿cómo representar a los estados de un sistema abierto?}
   \only<1>{
  La herramienta más apropiada para representar al estado de un 
  sistema cuántico abierto es la matriz de densidad $\rho$. \vfill
  
  Por ejemplo, supongamos un sistema cuyo estado es $\ket{\psi}$,
  su matriz de densidad se escribe
  \begin{align*}
  \rho=\dyad{\psi}{\psi}.
  \end{align*}
  }

	\only<2>{
	Una matriz $\rho$ representa a un estado cuántico si y sólo si 	
	\begin{enumerate}
		\item $\Tr (\rho)=1$,
		\item $\rho^{\dagger}=\rho$,
		\item $\rho\geq 0$.
	\end{enumerate}	
	}
\end{frame}


\begin{frame}{Canales cuánticos}%
  {¿cómo describir la dinámica de los sistemas abiertos?}
  
  La teoría de los canales cuánticos es un formalismo para describir 
  la evolución de los sistemas abiertos. \vfill
  
  \only<1>{
	Una operación lineal $\mathcal{E}$ que actúa sobre $\rho$ 	como 
	\begin{align*}
	\mathcal{E}(\rho)=\rho'
	\end{align*}
	es un canal cuántico si
	\begin{enumerate}
	\item Preserva las características de la matriz de densidad,
	\item Es una operación completamente positiva.
	\end{enumerate}
	}
	
	\only<2>{
	Una operación lineal $\mathcal{E}$ que actúa sobre $\rho$ como 
	\begin{align*}
	\mathcal{E}(\rho)=\rho'
	\end{align*}
	es un canal cuántico si
	\begin{enumerate}
	\item \alert{Preserva las características de la matriz de densidad},
	\item Es una operación completamente positiva.
	\end{enumerate}
	}
	
	\only<3>{
	Una operación lineal $\mathcal{E}$ que actúa sobre $\rho$ 	como 
	\begin{align*}
	\mathcal{E}(\rho)=\rho'
	\end{align*}
	es un canal cuántico si
	\begin{enumerate}
	\item Preserva las características de la matriz de densidad,
	\item \alert{Es una operación completamente positiva. ¿¿¿???}
	\end{enumerate}
	}
\end{frame}	

\begin{frame}{Completa positividad}{¿Para qué o qué?}
	Supongamos una operación $\mathcal{E}$ que actúa sobre 
	el sistema A.
	
	\only<1->{
	\begin{center}
	\includegraphics[height=6cm]{CP2}
	\end{center}	
	}
	
	No es suficiente que 	$\mathcal{E}(\rho_A)\geq0$, \alert{también 
	$\qty(\mathcal{E}\otimes \mathds{1}_k)\qty[\rho_{total}]\geq0$.}
\end{frame}

\begin{frame}{Canal cuántico}
	\begin{center}
		\includegraphics[width=7cm]{QuantumChannel}
	\end{center}
	\vspace{-.2cm}
%	\only<1>{
%	Una operación $\mathcal{E}$ es un canal cuántico si y sólo si 
%	\begin{align*}
%	 \qty(\mathcal{E}\otimes \mathds{1})\qty[\dyad{\text{Bell}}{\text{Bell}}]
%	\hspace{.5cm}	 
%	 \text{(matriz de Choi de }\mathcal{E})
%	\end{align*}
%	es (1) de traza unitaria, (2) Hermítica y (3) positiva.}
	\only<1>{
	Una operación $\mathcal{E}$ es un canal cuántico si y sólo si 
	\begin{align*}
	 \qty(\mathcal{E}\otimes \mathds{1})\qty[\dyad{\text{Bell}}{\text{Bell}}]
	\hspace{.5cm}	 
	 \alert{\text{(}\propto  \text{matriz de Choi de }\mathcal{E})}
	\end{align*}
	es (1) de traza unitaria, (2) Hermítica y (3) positiva.	
	}
\end{frame}


\begin{frame}{Resumen}{Sistemas cuánticos abiertos}
	\begin{enumerate}
		\item Estados cuánticos: matriz de densidad $\rho$
		\item Dinámica: canales cuánticos
	\end{enumerate}
\end{frame}

% % % % % % % % % % % % % % % % % % % % % % % % % % % % % % % % % % % % % %
% Qubits
% % % % % % % % % % % % % % % % % % % % % % % % % % % % % % % % % % % % % %

\section{Qubits}

\begin{frame}{Qubit}
	Un qubit es un sistema cuántico de dos niveles.
	
	\begin{columns}
	\hspace{1cm}
	\begin{column}{0.7\textwidth}
		\begin{align*}
			\rho &= \frac{\mathds{1}+r_1\sigma_x+r_2\sigma_y+r_3\sigma_z}{2},
		\end{align*}	
		$\qty(r_1,r_2,r_3)$ especifican las coordenadas cartesianas de un 
 		punto en la esfera de Bloch,
 	\begin{align*}
 		\alert{\rho = \qty(1,r_1,r_2,r_3)}.
 	\end{align*}
	\end{column}\hspace{-1cm}
	\begin{column}{0.5\textwidth}  
  		\begin{center}
    		\includegraphics[width=0.7\textwidth]{bloch-sph}      
     \end{center}
		\end{column}
	\end{columns}
\end{frame}

\begin{frame}{Rotación de la esfera de Bloch}
	Las rotaciones de la esfera de Bloch con canales cuánticos de 1 qubit.
	\begin{center}
	\begin{tabular}{m{2.5cm} m{1.5cm} m{2.5cm}}
		\includegraphics[width=3cm]{rotation_01}
		& \hfill \LARGE{$\longmapsto$} \hfill
		& \includegraphics[width=3cm]{rotation_02}
	\end{tabular}
	\end{center}
	
	\begin{align*}
	\qty(1,r_1,r_2,r_3) \longmapsto \qty(1,r_1,-r_3,r_2).
	\end{align*}
\end{frame}

\begin{frame}{Canal \textit{bit-flip} de 1 qubit}
	El \textit{bit-flip} actúa sobre la esfera de Bloch como
	\begin{center}
	\begin{tabular}{m{2.5cm} m{1.5cm} m{2.5cm}}
		\includegraphics[width=3cm]{unit_sph}
		& \hfill \LARGE{$\longmapsto$} \hfill
		0
%		& \includegraphics[width=3cm]{bit_flip_p0_3} 
	\end{tabular}
	\end{center}
	
	Transforma a las componentes $r_i$ de la matriz de densidad como
	\begin{align*}
	\qty(1,r_1,r_2,r_3) \longmapsto \qty(1,r_1,\qty(1-2p)r_2,\qty(1-2p)r_3).
	\end{align*}
\end{frame}

\begin{frame}{Matriz de densidad de $n$ qubits}
	En la base de productos tensoriales de las matrices de Pauli, 
	la matriz de densidad de $n$ qubits se escribe
	\begin{align*}
		\rho = \frac{1}{2^n}\sum _{j_1,\ldots,j_n=0}^3 r_{j_1,\ldots,j_n}
		\sigma_{j_1}\otimes \ldots
		\otimes\sigma_{j_n},\hspace{.8cm} r_{0,\ldots,0}=1.
	\end{align*}
	\alert{Llamaremos ``componentes de Pauli'' a las $r_{j_1,\ldots,j_n}$.}
\end{frame}

% % % % % % % % % % % % % % % % % % % % % % % % % % % % % % %
% Operaciones PCE
% % % % % % % % % % % % % % % % % % % % % % % % % % % % % % % 
\section{Operaciones PCE}
\label{sec:Theory}

\begin{frame}{Motivación (1/2)}
	Un caso particular del canal \textit{bit-flip} es cuando
	\begin{center}
	\begin{tabular}{m{2.5cm} m{1.5cm} m{2.5cm}}
		\includegraphics[width=3cm]{unit_sph}
		& \hfill \LARGE{$\longmapsto$}
		& \includegraphics[width=3cm]{bit_flip_p0_5}
	\end{tabular}
	\end{center}
	
	Las componentes de Pauli $r_i$ se transforman como
	\begin{align*}
	\qty(1,r_1,r_2,r_3)\longmapsto \qty(1,r_1,0,0).
	\end{align*}
	
	\alert{¿Son canales cuánticos todas las operaciones que borran
	cualesquiera de las componentes de Pauli de 1 qubit?}

\end{frame}


\begin{frame}{Motivación (2/2)}
	\only<1>{\alert{No.}}
	Las operaciones $\Lambda$ que borran dos componentes de Pauli no 
	son canales cuánticos.
	\begin{center}
	\begin{tabular}{m{2.5cm} m{1.3cm} m{2.5cm}}
		\includegraphics[width=3cm]{unit_sph}
		& \hfill \LARGE{$\longmapsto$}
		& \includegraphics[width=3.5cm]{unit_disk_xy}
	\end{tabular}
	\end{center}
	\vspace{-1cm}
	\begin{align*}
	\qty(1,r_1,r_2,r_3)\longmapsto\qty(1,r_1,r_2,0).
	\end{align*}
	
	\alert{$\qty(\Lambda\otimes \mathds{1})\qty[\dyad{\text{Bell}}{\text{Bell}}]\ngeq0$, 
	por consiguiente	$\Lambda$ no es completamente positiva.}
		
	\vspace{.5cm}
\end{frame}


\begin{frame}{Operaciones PCE}
	Una operación PCE (\textit{Pauli component erasing}) 
	es una operación lineal que transforma a las 	componentes 
	de Pauli de una matriz de densidad $\rho$ de $n$ qubits como
	\begin{align*}
	r_{j_1,\ldots,j_n}\longmapsto \tau_{j_1,\ldots,j_n}r_{j_1,\ldots,j_n},
	\hspace{.8cm}\tau_{j_1,\ldots,j_n}=0,1.
	\end{align*}
	\alert{Una operación PCE borra o deja invariantes las component's de Pauli. }
	
	\hfill
	\includegraphics[height=2cm]{1qubit_pce_01}
	\hspace{1.8cm}
	\includegraphics[height=2cm]{2qubit_pce_01}
	\hfill
	\includegraphics[height=2.5cm]{3qubit_pce_01}
	\hspace{1cm} 

	\vfill	
	
	\textbf{Problema:} ¿cuáles son las características del suconjunto 
	de canales cuánticos de las operaciones PCE?
\end{frame}


\begin{frame}{Canales cuánticos PCE}{Eigenvalores matriz de Choi} 
Los eigenvalores de $\qty(\mathcal{E}\otimes
\mathds{1})\qty[\dyad{\text{Bell}}{\text{Bell}}]$ son
\begin{align*}
%\qty(\bigotimes_{}^na) \vec{\tau} \geq \vec{0},\\
\vec{\lambda} = \frac{1}{2^n}
\underbrace{\qty(a\otimes \ldots\otimes a)}_{n\text{ veces}} \vec{\tau}
\end{align*}
con
\begin{align*}
a=\mqty(1&1&1&1\\1&1&-1&-1\\1&-1&1&-1\\1&1&-1&-1)
\end{align*}
y $\vec{\tau}$ un vector de $4^n$ componentes cuyas 
componentes son los 0's y 1's que definen a la operación PCE $\mathcal{E}$.
\alert{¡¡Esta es la diagonalización exacta de una matriz de dimensión $4^n$!!}
\end{frame}

\begin{frame}{Canales PCE de 1 y 2 qubits}
\begin{columns}
\begin{column}{0.4\textwidth}
De 8 operaciones posibles:
\begin{figure}
\includegraphics[height=2.2cm]{1qubitPCE}
\end{figure}
5 son canales cuánticos:
\begin{figure}
\includegraphics[height=2.2cm]{1qubit}
\end{figure}
\end{column} 
\begin{column}{0.6\textwidth}
De $\sim$32,000, 67 son canales cuánticos:
\begin{figure}
\includegraphics[width=.9\textwidth]{2qubits_01}\\
\includegraphics[width=.9\textwidth]{2qubits_02}\\
\includegraphics[width=.9\textwidth]{2qubits_03}
\end{figure}
\end{column}
\end{columns}
\end{frame}


\begin{frame}{Canales cuánticos PCE}
{Regla $2^k$}
De todas las operaciones PCE, sólo aquellas que dejan 
invariantes $2^k$ componentes de Pauli siguen siendo 
candidatos a canales cuánticos (\alert{sólo cantidatos}).

\vspace{.4cm}

\small{\textbf{\underline{2 qubits, 8 componentes invariantes:}}}

\vspace{.15cm}

\small{
\begin{columns}
\begin{column}{0.5\textwidth}
\hspace{.3cm}
No canales cuánticos:
\hspace{.3cm}
\begin{center}
\includegraphics[height=1.8cm]{2qubit_pceOp_8components}
\end{center}
\vspace{1.6cm}
\end{column} 
\begin{column}{0.5\textwidth}
Canales cuánticos:
\begin{center}
\includegraphics[height=3.8cm]{2qubit_pceQCh_8components}
\end{center}
\end{column}
\end{columns}}
\vspace{2cm}
\end{frame}



%\begin{frame}{Canales cuánticos PCE}{Regla espejo}
%El número de canales PCE según la cantidad de componentes 
%de Pauli invariantes obedece una regla `espejo'.
%
%\vfill
%
%\only<1>{\small{\textbf{\underline{2 qubits:}}}}
%%\begin{table}[]
%%\centering
%%\begin{tabular}{c|lllll}
%%\multirow{3}{*}{\textbf{canales PCE}} &                                &                                & \multicolumn{1}{c}{35}         &                                &                                 \\
%%                                                      &                                & \multicolumn{1}{c}{15}         &                                & \multicolumn{1}{c}{15}         &                                 \\
%%                                                      & \multicolumn{1}{c}{1}          &                                &                                &                                & \multicolumn{1}{c}{1}           \\
%%\hline 
%%\textbf{componentes}                                  & \multicolumn{1}{c}{\textbf{1}} & \multicolumn{1}{c}{\textbf{2}} & \multicolumn{1}{c}{\textbf{4}} & \multicolumn{1}{c}{\textbf{8}} & \multicolumn{1}{c}{\textbf{16}}
%%\end{tabular}
%%\end{table}
%%
%%\begin{table}[]
%%\begin{tabular}{c|lllll}
%%\multirow{3}{*}{\textbf{operaciones PCE}} &                                &                                & \multicolumn{1}{c}{455}        &                                &                                 \\
%%                                                      &                                & \multicolumn{1}{c}{15}         &                                & \multicolumn{1}{c}{6435}       &                                 \\
%%                                                      & \multicolumn{1}{c}{1}          &                                &                                &                                & \multicolumn{1}{c}{1}           \\
%%\hline
%%\textbf{componentes}                                  & \multicolumn{1}{c}{\textbf{1}} & \multicolumn{1}{c}{\textbf{2}} & \multicolumn{1}{c}{\textbf{4}} & \multicolumn{1}{c}{\textbf{8}} & \multicolumn{1}{c}{\textbf{16}}
%%\end{tabular}
%%\begin{table}[]
%%\footnotesize
%%\begin{tabular}{cccccccc}
%%\multirow{4}{*}{\textbf{canales PCE}} &            &            &            & 1395       &             &             &             \\
%%                                                      &            &            & 651        &            & 651         &             &             \\
%%                                                      &            & 63         &            &            &             & 63          &             \\
%%                                                      & 1          &            &            &            &             &             & 1           \\
%%\textbf{componentes}                                  & \textbf{1} & \textbf{2} & \textbf{4} & \textbf{8} & \textbf{16} & \textbf{32} & \textbf{64}
%%\end{tabular}
%%\end{table}
%%\end{table} 
%%
%%\begin{table}[]
%%\footnotesize
%%\begin{tabular}{cccccccc}
%%\multirow{4}{*}{\textbf{operaciones PCE}} &            &            &            & $5.5\times 10^7$ &                     &                         &             \\
%%                                          &            &            & 39711      &             & $1.2\times 10^{14}$ &                         &             \\
%%                                          &            & 63         &            &             &                     & $9.2\times 10^{17}$ &             \\
%%                                          & 1          &            &            &             &                     &                         & 1           \\
%%\textbf{componentes}                      & \textbf{1} & \textbf{2} & \textbf{4} & \textbf{8}  & \textbf{16}         & \textbf{32}             & \textbf{64}
%%\end{tabular}
%%\end{table}
%\begin{table}[H]
%\resizebox{\textwidth}{!}{%
%\begin{tabular}{l|cccccccccccccccc}
%\cline{2-17}
%\multicolumn{1}{l|}{}                               & \multicolumn{16}{c|}{\textbf{Number of Pauli components left invariant}}                                                                                                                                                                                                                                                                                                                                                                                                                                                                                             \\[.5ex]
%\multicolumn{1}{l|}{}                               & \multicolumn{1}{c}{\textbf{1}} & \multicolumn{1}{c}{\textbf{2}} & \multicolumn{1}{c}{\textbf{3}} & \multicolumn{1}{c}{\textbf{4}} & \multicolumn{1}{c}{\textbf{5}} & \multicolumn{1}{c}{\textbf{6}} & \multicolumn{1}{c}{\textbf{7}} & \multicolumn{1}{c}{\textbf{8}} & \multicolumn{1}{c}{\textbf{9}} & \multicolumn{1}{c}{\textbf{10}} & \multicolumn{1}{c}{\textbf{11}} & \multicolumn{1}{c}{\textbf{12}} & \multicolumn{1}{c}{\textbf{13}} & \multicolumn{1}{c}{\textbf{14}} & \multicolumn{1}{c}{\textbf{15}} & \multicolumn{1}{c|}{\textbf{16}} \\ \cline{2-17} 
%\multicolumn{1}{c|}{\textbf{\# of PCE operations}}   & 1                               & 15                              & 105                             & 455                             & 1365                            & 3003                            & 5005                            & 6435                            & 6435                            & 5005                             & 3003                             & 1365                             & 455                              & 105                              & 15                               & 1                                \\[.4ex]
%\multicolumn{1}{c|}{\textbf{\# of PCE qtm channels}} & 1                               & 15                              & 0                               & 35                              & 0                               & 0                               & 0                               & 15                              & 0                               & 0                                & 0                                & 0                                & 0                                & 0                                & 0                                & 1                               
%\end{tabular}%
%}
%\end{table}
%
%\begin{table}[H]
%\centering
%\resizebox{.5\textwidth}{!}{%
%\begin{tabular}{l|ccccc}
%\cline{2-6}
%\multicolumn{1}{l|}{}                               & \multicolumn{5}{c|}{\textbf{Number of $\boldsymbol{r_{ij}}$ inv.}} \\[.5ex]
%\multicolumn{1}{l|}{}                               & \multicolumn{1}{c}{\textbf{1}} & \multicolumn{1}{c}{\textbf{2}} &  \multicolumn{1}{c}{\textbf{4}} & \multicolumn{1}{c}{\textbf{8}} &  \multicolumn{1}{c|}{\textbf{16}} \\ \cline{2-6} 
%\multicolumn{1}{c|}{\textbf{\# of PCE qtm channels}} & 1                               & 15                               & 35                          & 15                        & 1                               
%\end{tabular}%
%}
%\end{table}
%
%\end{frame}





\begin{frame}{Canales cuánticos PCE}{Generadores}
Los canales cuánticos PCE pueden escribirse como concatenación 
de canales generadores,
\begin{align*}
\Phi = \underbrace{\mathcal{E}_{j_1}\circ \mathcal{E}_{j_2}\circ \ldots
\circ \mathcal{E}_{j_l}}_{\text{máximo }2^n},
\end{align*}
con $\mathcal{E}_{j_i}$ los generadores.

\vfill

Los generadores $\mathcal{E}_{j_i}$ son las únicas formas que están 
físicamente permitidas de borrar la mitad de las componentes de Pauli 
de un sistema de $n$ qubits $+\ \mathds{1}$.
\end{frame}

\begin{frame}{Ejemplo: 2 qubits}
{Canales PCE a partir de generadores}
%\onslide<1->{
\underline{Generadores:}
\begin{center}
\includegraphics[width=8cm]{2qubits_pce_generators}
\end{center}
%}

%\onslide<2->{
\hfill
\begin{tabular}{m{0.8cm} m{.1cm} m{0.8cm} m{.05cm} m{0.8cm}}
\includegraphics[height=1.1cm]{E2}  
& $\bigcirc$ 
& \includegraphics[height=1.1cm]{E5} 
& $=$
& \includegraphics[height=1.1cm]{E2E5} 
\end{tabular} \hspace{1.3cm}
%}

%\onslide<3->{
\hfill
\begin{tabular}{m{0.8cm} m{.1cm} m{0.8cm} m{.1cm} m{0.8cm} m{.1cm} m{0.8cm}}
\includegraphics[height=1.1cm]{E4}
& $\bigcirc$ 
& \includegraphics[height=1.1cm]{E2}  
& $\bigcirc$ 
& \includegraphics[height=1.1cm]{E5} 
& $=$
& \includegraphics[height=1.1cm]{E4E2E5} 
\end{tabular} \hspace{1.3cm}
%}

%\onslide<4->{
\hfill
\begin{tabular}{m{0.8cm} m{.1cm} m{0.8cm} m{.1cm} m{0.8cm} m{.1cm} m{0.8cm} m{.1cm} m{0.8cm}}
\includegraphics[height=1.1cm]{E12}
& $\bigcirc$ 
& \includegraphics[height=1.1cm]{E4}
& $\bigcirc$ 
& \includegraphics[height=1.1cm]{E2}  
& $\bigcirc$ 
& \includegraphics[height=1.1cm]{E5} 
& $=$
& \includegraphics[height=1.1cm]{E12E4E2E5} 
\end{tabular} \hspace{1.3cm}
%}
\end{frame}



\begin{frame}{Resumen}
	\only<1>{
	\begin{itemize}
		\item Una operación PCE es una operación que actúa sobre un sistema
		de $n$ qubits y que borra o deja invariantes a sus componentes de Pauli.
		\item El problema que estudiamos es caracterizar a las operaciones PCE 
		que son canales cuánticos (las que satisfacen la condición de completa
		positividad).
	\end{itemize}
	}
	\only<2>{
	\begin{itemize}
		\item Diagonalización de la matriz de Choi: $\vec{\lambda}=a^{\otimes n}\vec{\tau}$
		\item Una operación PCE $\Phi$ se puede entender como la concatenación
		de canales cuánticos que describen las únicas formas permitidas por 
		la mecánica cuántica de borrar la mitad todas componentes de Pauli del
		estado de un sistema de qubits.
	\end{itemize}
	}
\end{frame}

% % % % % % % % % % % % % % % % % % % % % % % % % % % % % % % % % % % % % %
% Let's have good manners
% % % % % % % % % % % % % % % % % % % % % % % % % % % % % % % % % % % % % %
\begin{frame}
  \begin{center}
    ¡Muchas gracias!

    \vfill

    Contacto:

    José Alfredo de León

    deleongarrido.jose@gmail.com
  \end{center}
\end{frame}

\end{document}

%%% Local Variables:
%%% mode: latex
%%% TeX-master: t
%%% End:
