%%%%% slides template
% Giovanni Ramirez (ramirez@ecfm.usac.edu.gt)
% 20200828
% Require: images/ecfmByN.png images/usacByN.jpg
%
%
% Use this option to print notes only
% \documentclass[xcolor=dvipsnames,handout,notes=only]{beamer}%
%
% Use this option to print notes and slides
% \documentclass[xcolor=dvipsnames,handout,notes=show]{beamer}%
%
% Use this option to get a printer-friendly version
% \documentclass[xcolor=dvipsnames,handout]{beamer}%
%
% Unse this option to get slides only
\documentclass[xcolor=dvipsnames,presentation]{beamer}%
%
%%%%%


%%%%% theme and coloring
% a list of themes can be found here https://hartwork.org/beamer-theme-matrix/
% I like a simple Malmoe with some modifications and color-schemes like
% dolphin for a white background, beetle for a gray background or dove for
% plain slides
%
% theme definition
\usetheme{Malmoe}
%
% definition of the presentation mode: full slides
\mode<presentation> {
% coloring for slides other options: beetle, dove
  \usecolortheme{dolphin}
% to include a slide with the table of contents
 \AtBeginSection[] {
  \begin{frame}
    \tableofcontents[currentsection]
  \end{frame} }
}
%
% definition of the handout mode: printer-friendly
\mode<handout>{
% coloring: dove is white
  \usecolortheme{dove}
% this is to print 2 slides in 1 page, other values are allowed
  \usepackage{pgfpages}
  \renewcommand\textbullet{\ensuremath{\bullet}}
  \pgfpagesuselayout{2 on 1}[letterpaper,border shrink=5mm]
% set the fontsize of the notes
  \setbeamerfont{note page}{size=\scriptsize}
% disable the slide with the table of contents
  \AtBeginSection[]{}
}
%%%%%

%%%%% personalisation
% clear all headers
\setbeamertemplate{headline}{}
% disable navigation buttons
\setbeamertemplate{navigation symbols}{}
% configure the blocks used for equations, theorems, etc
\setbeamertemplate{blocks}[rounded][shadow=true]
% clear the footline
\setbeamertemplate{footline}{}
% set the footline
\setbeamertemplate{footline}{
  \hbox{%
% left block with the shortname, it is set in the \author command
    \begin{beamercolorbox}[wd=.40\paperwidth,ht=2.25ex,dp=1ex,center]%
      {author in head/foot}%
      \usebeamerfont{author in head/foot}\insertshortauthor
    \end{beamercolorbox}%
% center block with the short title, it is set in the \title command
    \begin{beamercolorbox}[wd=.50\paperwidth,ht=2.25ex,dp=1ex,center]%
      {title in head/foot}%
      \usebeamerfont{title in head/foot}\insertshorttitle
    \end{beamercolorbox}%
% right block with the number of slides
    \begin{beamercolorbox}[wd=.10\paperwidth,ht=2.25ex,dp=1ex,right]%
      {date in head/foot}%
      \usebeamerfont{date in head/foot}
      \insertframenumber{} / \inserttotalframenumber\hspace*{1em}
    \end{beamercolorbox}}%
  \vskip0pt%
}
%%%%%

%%%%% packages
%
% language, decimal point mark
\usepackage[spanish,es-nodecimaldot]{babel}%
%
% to use only T1 scalable fonts
\usepackage[T1]{fontenc}%
\let\Tiny=\tiny% this is a proper configuration of the tiny font size
%
% to use UTF coding, other options latin1
\usepackage[utf8]{inputenc}%
%
\usepackage{array}
% special math symbols
\usepackage{latexsym,amsfonts,amsmath}%
%
% to include graphics
\usepackage{graphicx}%
% path to images
%\graphicspath{ {../images/} }
%
% to use tiks: nice to draw diagrams
\usepackage{tikz}%
\usetikzlibrary{arrows.meta}
%
% to use SI units and physics notation
\usepackage{physics}%
\usepackage{multirow}
\usepackage{xcolor}

\usepackage{pifont}% http://ctan.org/pkg/pifont
\newcommand{\cmark}{\ding{51}}%
\newcommand{\xmark}{\ding{55}}%
\newcommand{\Mod}[1]{\ (\mathrm{mod}\ #1)}
%
% to use hypenation
\usepackage{ragged2e}%
\let\raggedright=\RaggedRight%
%
\usepackage{dsfont}
% to configure the hyperref to set metadata in the PDF file
\hypersetup{%
%pdfpagemode=FullScreen,% to open the file in fullscreen
breaklinks={true},% to allow links in more than one line
%pdfstartview={Fit},% to set the slide to fit the screen
%pdfpagemode=FullScreen,% to set it to full screen
pdfauthor={Giovanni Ramirez Garcia (ramirez@ecfm.usac.edu.gt)},% author name
pdftitle={Quantum Optics aplied to Quantum Computing},% title of the talk
pdfsubject={Keynote for the talk at Lat0 group},% subject of the talk
pdfkeywords={quantum optics, quantum computing, quantum algorithms, quantum
  circuits}% keywords to categorise the contents
}
%%%%%

%%%%% Talk data
%
%title
\title[Canales cuánticos PCE y de Weyl]% this is the short title
{\bf Canales cuánticos PCE y de Weyl}% this is the long title
%
% author
\author[José Alfredo de León]% this is the short name
{José Alfredo de León, Tomás Basile}
%{\bf José Alfredo de León, Tomás Basile, 
%Carlos Pineda,
%David Dávalos, 
%Alejandro Fonseca}% this is the long name
%
% affiliation
%\institute{${}^1$Escuela de Ciencias Físicas y Matemáticas, U. de 
%San Carlos de Guatemala\newline%
%${}^2$Instituto de Física, U. Nacional Autónoma de México\newline
%${}^3$Departamento de Física, U. Federal de Pernambuco}%
%
% date and information about the conference
%\date{I Congreso Guatemalteco de Física\\09 de julio de 2021}
%%%%%


\begin{document}
% % % % % % % % % % % % % % % % % % % % % % % % % % % % % % % % % % % % % %
% Title and Outline
% % % % % % % % % % % % % % % % % % % % % % % % % % % % % % % % % % % % % %
\begin{frame}[plain]
  \titlepage %
  \begin{tikzpicture}[x=1mm,y=1mm,overlay,remember picture]
    \pgftransformshift{\pgfpointanchor{current page}{center}}
%    \node[inner sep=0pt] (usac) at (-35,-30) %
%    {\includegraphics[height=12mm]{logos/ecfmByN}};
%    \node[inner sep=0pt] (usac) at (0,-30) %
%    {\includegraphics[height=12mm]{logos/ifunam}};
%    \node[inner sep=0pt] (ecfm) at (35,-30) %
%    {\includegraphics[height=12mm]{logos/u-pernambuco}};
  \end{tikzpicture}
\end{frame}

\begin{frame}{El problema de los Pauli Component Erasing (PCE)}
No todas las proyecciones de la esfera de Bloch son evoluciones físicas
de sistemas abiertos.
% \begin{minipage}{.48\textwidth}
\begin{center}
\begin{tikzpicture}
 %nodes
% \draw[draw=white] (-2.7,-4.5) rectangle ++(5.5,5);
\draw[draw=white] (-1.2,-4.1) rectangle ++(6.5,5.5);
%  \node[punkt] (market) {Market (b)};
% \filldraw [fill=Peach, draw=black] (11.1,5) rectangle (0.3,0.3);
 % We make a dummy figure to make everything look nice.
%  \node at (0, 0)  {\includegraphics[width=.99\textwidth]{spectraAll}};
\node at (0, 0)  {%
\only<1>{\includegraphics[width=.2\textwidth,page=1]{../images/blochs-crop}}%
\only<2->{\includegraphics[width=.2\textwidth,page=2]{../images/blochs-crop}}%
};
\only<3->{
\draw [-{Stealth[length=3mm, width=3mm]},line width=3pt] (1.4,0) -- (2.7,0); 
\node at (2.,.3) {\color{green} \cmark};
}
\only<5->{
\node at (2.,-2.5) {\color{red} \xmark};
\draw [-{Stealth[length=3mm, width=3mm]},line width=3pt] (1.4,-2.8) -- (2.7,-2.8); 
}
\node at (4,  0)  {%
\only<3->{\includegraphics[width=.2\textwidth,page=3]{../images/blochs-crop}}%
% \only<2->{\includegraphics[width=.2\textwidth,page=2]{../images/blochs-crop}}%
};
\node at (0,  -2.8)  {%
\only<4->{
\includegraphics[width=.2\textwidth,page=5]{../images/blochs-crop}}%
};
\node at (4,  -2.8)  {%
\only<5->{
\includegraphics[width=.2\textwidth,page=4]{../images/blochs-crop}}%
};
\end{tikzpicture}
% Sistemas naturales, estables\\[.2cm]
% Acá un dibujo en dos partes, con la esfera de Bloch en la primera indicando 
% algun punto que ava a viajar al disco central, luego una segunda con solo el 
% disco. Luego una segunda serie con eso, pero proyectado al eje z. 
\end{center}
\end{frame}

\begin{frame}
\begin{center}
\begin{figure}
\includegraphics[width=0.75\textwidth]{../images/tetrahedron}
\caption{Tetrahedro de 1 qubit. Figura donde podemos ubicar los 
canales cuánticos de 1 qubit. Tomada de Geometry of Quantum States.}
\end{figure}
\end{center}
\end{frame}

\begin{frame}{Canales de Weyl: definición}
Un canal de Weyl es un mapeo \alert{completamente positivo} que transforma a las
componentes del vector de Bloch como
\begin{align*}
r_{\vec \alpha}\mapsto \tau_{\vec\alpha}r_{\vec\alpha},
\qquad \tau_{\vec\alpha}\in\mathbb{C} \quad\text{y}\quad 
\norm{\tau_{\vec \alpha}}\leq 1,
\end{align*}
donde las $\tau_{\vec\alpha}$ deben satisfacer ciertas condiciones 
específicas para que el mapeo satisfaga \alert{la} condición de completa 
positividad.
\end{frame}

\begin{frame}{Resultados interesantes}
\begin{itemize}
\item<1-> Si se restringe el problema a qubits y se impone la restricción
\begin{align*}
r_{\vec \alpha}\mapsto \tau_{\vec\alpha}r_{\vec\alpha},\qquad 
%\qquad \tau_{\vec\alpha}\in\mathbb{C} \quad\text{y}\quad 
\alert{\tau_{\vec \alpha}= 0,1};
\end{align*}
se define un mapa \alert{Pauli erasing component (PCE)}.
Las $\tau_{\vec\alpha}=1$ de los canales cuánticos (mapeos CP)
forman un espacio vectorial.

\item<2-> Los puntos extremos de la generalización del tetrahedro para sistemas
de más de 2 niveles son aquellos mapeos de Weyl definidos por 
\begin{align*}
r_{\vec \alpha}\mapsto \tau_{\vec\alpha}r_{\vec\alpha},\qquad 
%\qquad \tau_{\vec\alpha}\in\mathbb{C} \quad\text{y}\quad 
\alert{\norm{\tau_{\vec \alpha}}= 1}.
\end{align*}
¡Cualquier otro canal se construye como una combinación convexa de estos 
puntos extremos!
\end{itemize}



\end{frame}

\begin{frame}{Canales cuánticos PCE: son subgrupos}
Si para un canal cuántico PCE $\tau_{pq}=\tau_{p'q'}=1$ ($p,q\ne p',q'$), 
entonces $\tau_{p+p',q+´q'}=1$.\\[1ex]

\textbf{Prueba:} consideremos
\scriptsize
\begin{align}
\lambda_{\mu \nu}=\frac{1}{2}\sum_{p,q=0}^1(-1)^{\mu p+\nu q\Mod{2}}\tau_{pq}
\hspace*{2pt}&\Leftrightarrow\hspace*{2pt} 
\tau_{pq}=\frac{1}{2}\sum_{\mu,\nu}(-1)^{-\mu p-\nu q},\label{eq:exp:1} \\
\sum_{\mu,\nu}\lambda_{\mu\nu}&=2.\label{eq:exp:2}
\end{align}

Restamos \eqref{eq:exp:2} de la expresion \eqref{eq:exp:1}, 
considerando $\tau_{pq}=1´$,
\scriptsize{
\begin{align*}
\sum_{\mu,\nu}\lambda_{\mu\nu}\qty[(-1)^{-\mu p-\nu q}-1]=0
\hspace*{10pt}\Leftrightarrow \hspace*{10pt}
(-1)^{-\mu p}&=(-1)^{\nu q},\quad \forall \mu,\nu\vert \lambda_{\mu\nu}\ne 0.
\end{align*}
}
Similarmente se puede deducir una expresión para $p',q'$. Luego,
\scriptsize{
\begin{align*}
\tau_{p+p',q+q'}=\frac{1}{2}\sum_{\mu,\nu}
(-1)^{-\mu(p+p')-\nu(q+q')}\lambda_{\mu\nu}
\onslide<2->{=\frac{1}{2}\sum_{\mu,\nu|\lambda_{\mu\nu}\ne 0}
(-1)^{\nu(q+q')-\nu(q+q')}\lambda_{\mu\nu} = 1.}
\end{align*}
}\normalsize

\alert{Es decir, los índices forman un grupo}
\begin{align*}
(p,q)+(p',q')=(p+p',q+q')
\end{align*}
\end{frame}


\end{document}

%%% Local Variables:
%%% mode: latex
%%% TeX-master: t
%%% End:
