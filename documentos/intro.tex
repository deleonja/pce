\section{Introduction} % {{{

Quantum correlations~\cite{PhysicsPhysiqueFizika.1.195,
PhysRevLett.126.170404,Ollivier2001}, including entanglement~\cite{Horodecki},
are an important resource for a wide variety of tasks that
include teleportation~\cite{PhysRevLett.70.1895}, quantum
computation~\cite{nielsen_chuang_2011}, and others~\cite{Sapienza2019}.
However, this resource is also extremely
delicate~\cite{nielsen_chuang_2011,Schlosshauer2004}, especially 
for multi-particle systems~\cite{Horodecki}; that is why an important part
of the efforts of the community implementing quantum technologies is 
devoted to tackle this issue from an experimental \cite{Mooney_2021,Briegel1998} and
theoretical \cite{Bennett1996,Georgescu2020,RevModPhys.87.307} point of view. 
The process by which quantum correlations are unintentionally dissipated
is called {\it decoherence}~\cite{Schlosshauer2004,sep-qm-decoherence}.
One of the main tools to study the effects of decoherence are 
quantum channels. Quantum channels can describe quantum
noise~\cite{davalosdivisibility,ruskai}, open quantum systems
dynamics~\cite{nielsen_chuang_2011,breuerbook}, and recently even coarse
graining~\cite{Duarte2017,pinedacg}. 
%
One of the main difficulties 
in characterizing quantum channels is that, like for quantum states, 
the number of parameters required for their description increases quite 
rapidly with Hilbert-space dimension.
% \sout{to characterize quantum channels is that,
% similar to quantum states, the parameters needed to describe them
% increases overwhelmingly rapidly with the dimension of the Hilbert space.}
%
Moreover, such parameters are constrained in a complicated way by 
physical conditions, such as complete positivity ~\cite{zimansbook}.
% Such parameters are not fully independent, the complete positivity condition imposes the regions that have physical meaning~\cite{zimansbook}.
% 
Describing in detail families of channels having a given property provides
insight into the jungle of quantum operations.
For the qubit case there are several studies concerning the unital case, for which
non-trivial properties can be described using only three parameters, which in
turn form the well-known tetrahedron of Pauli
channels~\cite{Rybar2012,ruskai,davalosdivisibility}. 
% For general
% finite-dimensional channels, several families of channels have been defined.
More generally, in~Ref.~\cite{nathanson2007pauli} the authors study families of 
convex combinations of quantum-classical channels that relate to
% have a close relation with 
unital qubit channels with positive eigenvalues, and give a
generalization of the Bloch sphere. 
% {\blue We will show later that some of the channels studied here are also
% quantum-classical.}\ddnote{Esta frase no me gusta, pero no se me occure nada
% mejor en tanto a la escritura. Propongo omitirla, despues se volvera a
% mencionar esto pero en donde corresponde.}
Similarly, a generalization of Pauli channels based on mutually unbiased
measurements is introduced and studied in Ref.~\cite{Siudzinska2020}. Other
studies of channels beyond the qubit can be
found~\cite{pub.1117408526,pub.1060516327,pub.1014660854,Fonseca2019}. 
% \cpnote{Me gusta la idea de ir sobre un par de familias, pero no veo para que 
% mencionar a las MUBs aca, ni paraque introducir la sigla. Ojala podamos 
% discutir que familias mencionamos acá}\ddnote{atendido como fue acordado en la reunion :)}

In this paper we present a generalization of idempotent Pauli channels---\ie{},
the qubit flip operations (bit, phase, and bit-phase when the flip probability is $1/2$), total depolarizing qubit
channel and the identity channel---to the case of $N$ qubits. 
% \cpnote{Mencionar decoherencia} 
The generalization is done by extending
Pauli observables to \textit{Pauli strings} (tensor products of Pauli
matrices)~\cite{KIMURA2003339,Lawrence2002}. The resulting maps are unital and
diagonal in the Pauli strings' basis. 
We shall in the following refer to 
such maps as {\em Pauli component erasing} (\pce{}) {\it maps}.

The main task which we perform in this paper is the identification of the
conditions which an arbitrary \pce{} map must satisfy in order to be completely
positive. The answer turns out to involve a strikingly simple and unexpected
mathematical structure that is exploited to gain deeper understanding
on aforementioned channels, as we show in section \ref{sec:vector_spaces}. This
structure allows us, for example, to describe such channels with a much reduced
set of parameters (as compared to specifying a list of all erased Pauli
components) or to define an
interesting semigroup structure on the set of all \pce{} channels. 
Additionally, these channels are, in a sense, the simplest possible channels,
and as such can be used as building blocks of more general channels. For 
instance, one can combine them (through convex superposition) or compose them
with unitary transformations.
To summarize succinctly the final result, we show that it is possible
to assign to every Pauli string a simple \pce{} channel, obtained by extending
the system with an ancilla of a single qubit, acting on the combined system by
a unitary involving the Pauli string and tracing over the ancilla. It then
follows from our results that {\em all\/} \pce{} channels arise from such
channels by composition.

%In addition, these channels describe the decoherence process in a similar way
%as the aforementioned
%ones for the single-qubit case.
%Additionally, we characterize the algebraic
%properties of the set of \pce{} channels, showing that it is equipped
%with a semigroup structure, which is later used to derive physical
%implementations.


%
\par
The paper is organized as follows. In section~\ref{sec:pce_maps} we recall
the properties of quantum channels needed to proceed with the definition 
of 
% the tools to unders
% refreshing of quantum channels theory and the definition of 
\pce{} maps. 
In section~\ref{sec:math} we diagonalize analytically the Choi matrix for
arbitrary \pce{} maps and characterize their complete positivity by
interpreting \pce{} quantum channels as finite vector subspaces.
% In
% section~\ref{sec:math} we diagonalize analytically the Choi matrix for
% arbitrary \pce{} maps and characterize \janote{the} completely positive maps by interpreting
% \pce{} maps \janote{quantum channels?} as finite vector spaces. 
We study the
generators of the semigroup structure associated to the set of \pce{} channels in section~\ref{sec:generators}, and we use
them to derive meaningful physical interpretations of \pce{} channels in
section~\ref{sec:kraus}, as well as Kraus operators of the generators.
% 
To finish, we conclude and discuss future perspectives and possible generalizations in
section~\ref{sec:conclusions}.
% }}}


