\section{Introduction} % {{{
%
The identification of families of quantum channels among the whole convex set of quantum channels associated to a Hilbert space, is crucial to understand quantum noise~\cite{davalosdivisibility,ruskai}, open quantum systems dynamics~\cite{nielsen_chuang_2011,breuerbook}, and recently even coarse graining~\cite{Duarte2017,pinedacg}. One of the main difficulties is that, similar to quantum states, the parameters needed to describe quantum channels increases overwhelmingly rapid respect to the dimension of the Hilbert space. For a Hilbert space of dimension $d$, the space of quantum channels is parametrized with $d^4-d^2$ real parameters which are not fully independent (the complete positivity condition imposes the regions that physical meaning, see Ref.~\cite{zimansbook}). Therefore, even the qubit case is difficult to analyze having $12$ parameters corresponding to $6$ angles, $3$ components of the displacement vector and $3$ describing the compression of the Bloch sphere~\cite{ruskai}. Thus, understanding families of channels sharing a common property usually gives a good insight about the jungle of quantum operations. For the qubit case there are several studies concerning the unital case, which non-trivial properties can be described using only $3$ parameters, which in turn form the well known tetrahedron of Pauli channels~\cite{Rybar2012,ruskai,davalosdivisibility}. For general finite-dimensional channels, several families of channels have been defined. In~Ref.~\cite{nathanson2007pauli} authors define families of unital channels as convex combinations of quantum-classical channels, in turn defined with mutually unbiased basis (MUB), are introduced. These channels have properties in common with unital qubit channels with positive eigenvalues, and give a generalization of the Bloch sphere in terms of the generators of the MUBs. Similarly, a generalization of Pauli channels based on mutually unbiased measurements is introduced and studied in Ref.~\cite{Siudzinska2020}. Other studies of channels beyond the qubit can be found~\cite{pub.1117408526,pub.1060516327,pub.1014660854}.

\par
In this work we present a generalization of Pauli channels (convex combinations of Pauli unitaries) that are also idempotent, \ie{} the qubit flip operations (bit, phase and bit-phase), total depolarizing qubit channel and the identity channel, to the case of $N$ qubits. The generalization is done by extending Pauli observables to \textit{Pauli strings} (tensor products of Pauli matrices), the resulting maps are unital and diagonal in the Pauli strings' basis. We call such maps \textit{Pauli component erasing maps} (\pce{}). We derive necessary and sufficient conditions for such mappings to be completely positive trace preserving, thus, showing which idempotent operations are valid quantum channels (\pce{} channels). Additionally, we characterize the algebraic properties of the set of \pce{} channels, showing that is equipped with a semigroup structure, which is later used to derive physical implementations.
%
\par
The paper is organized as follows, in section~\ref{sec:pce_maps} we give a refreshing of quantum channels theory and the definition of \pce{} maps, being the generalization of idempotent Pauli channels aforementioned. In section~\ref{sec:math} we diagonalize analytically the Choi matrix for arbitrary \pce{} maps and characterize completely positive maps by interpreting \pce{} maps as discrete vector spaces. In section~\ref{sec:generators} we study generators of the semigroup structure of the set of \pce{} channels, and we use them to derive meaningful physical interpretations of \pce{} channels in section~\ref{sec:kraus}, as well Kraus operators of the generators.
%
To finish, we conclude in section~\ref{sec:conclusions}.
% }}}


