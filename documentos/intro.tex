\section{Introduction} % {{{

Quantum correlations \cite{}, including entanglement~\cite{Horodecki}, are an important
resource for a wide variate of tasks that include teleportation~\cite{PhysRevLett.70.1895},
quantum computation~\cite{}, and others~\cite{}.
However, this resource is also extremely delicate~\cite{}, specially 
for multi-particle systems~ \cite{},  and important part
of the efforts of the community implementing quantum technologies is 
devoted to tackle this issue from an experimental \cite{} and
theoretical \cite{} point of view. 
The process by which quantum correlations are unintentionally dissipated
is called decoherence~\cite{}.
One of the main tools to study the effects of decoherence are 
quantum channels. \cpnote{Ver como pegar con el siguiente bicho. Quizá 
modificar la primera frase del siguiente parrafo o pasar a este.}\par
\ddnote{Para pegar los dos parrafos, propongo la siguiente modificacion a la primera frase}. {\blue More generally, quantum channels model quantum noise~\cite{davalosdivisibility,ruskai}, open quantum systems
dynamics~\cite{nielsen_chuang_2011,breuerbook}, and recently even coarse
graining~\cite{Duarte2017,pinedacg}. [...] Luego continuar con el resto}
%
The identification of families of quantum channels among the whole convex set
of quantum channels associated to a Hilbert space, is crucial to understand
quantum noise~\cite{davalosdivisibility,ruskai}, open quantum systems
dynamics~\cite{nielsen_chuang_2011,breuerbook}, and recently even coarse
graining~\cite{Duarte2017,pinedacg}. One of the main difficulties is that,
similar to quantum states, the parameters needed to describe quantum channels
increases overwhelmingly rapidly with the dimension of the Hilbert space.
Such parameters are not fully independent, the complete positivity condition imposes the regions that have physical meaning~\cite{zimansbook}.\cpnote{quiza quitar (QQ), o modificiar 
a decir que el numero de parametros crece exponencialmente con el numero 
de particulas.}\ddnote{Atendido, dejo lo esencial y relacionado con el articulo}. Therefore, even the
qubit case is difficult to
analyze having $12$ parameters corresponding to $6$ angles, $3$ components of
the displacement vector and $3$ describing the compression of the Bloch
sphere~\cite{ruskai}\cpnote{QQ}\ddnote{quitemoslo}. Thus, describing in detail small families of channels having a given property may provide insight into the jungle of quantum operations.
For the qubit case there are several studies concerning the unital case, for which
non-trivial properties can be described using only $3$ parameters, which in
turn form the well known tetrahedron of Pauli
channels~\cite{Rybar2012,ruskai,davalosdivisibility}\cpnote{QQ}\ddnote{quitemoslo}. For general
finite-dimensional channels, several families of channels have been defined.
In~Ref.~\cite{nathanson2007pauli} the authors define families of unital channels as
convex combinations of quantum-classical channels, having properties
in common with unital qubit channels with positive eigenvalues, and give a
generalization of the Bloch sphere. {\blue We will show later that some of the channels studied here are also quantum-classical.}\ddnote{Esta frase no me gusta, pero no se me occure nada mejor en tanto a la escritura. Propongo omitirla, despues se volvera a mencionar esto pero en donde corresponde.}
Similarly, a generalization of Pauli channels based on mutually unbiased
measurements is introduced and studied in Ref.~\cite{Siudzinska2020}. Other
studies of channels beyond the qubit can be
found~\cite{pub.1117408526,pub.1060516327,pub.1014660854}.
\cpnote{Me gusta la idea de ir sobre un par de familias, pero no veo para que 
mencionar a las MUBs aca, ni paraque introducir la sigla. Ojala podamos 
discutir que familias mencionamos acá}\ddnote{atendido como fue acordado en la reunion :)}

\par
In this work we present a generalization of Pauli channels (convex combinations
of Pauli unitaries) that are also idempotent, \ie{} the qubit flip operations
(bit, phase and bit-phase), total depolarizing qubit channel and the identity
channel, to the case of $N$ qubits. The generalization is done by extending
Pauli observables to \textit{Pauli strings} (tensor products of Pauli
matrices), the resulting maps are unital and diagonal in the Pauli strings'
basis. We call such maps \textit{Pauli component erasing maps} (\pce{}). We
derive necessary and sufficient conditions for such mappings to be completely
positive trace preserving, thus, showing which idempotent operations are valid
quantum channels (\pce{} channels). Additionally, we characterize the algebraic
properties of the set of \pce{} channels, showing that is equipped with a
semigroup structure, which is later used to derive physical implementations.
%
\par
The paper is organized as follows, in section~\ref{sec:pce_maps} we give a
refreshing of quantum channels theory and the definition of \pce{} maps, being
the generalization of idempotent Pauli channels aforementioned. In
section~\ref{sec:math} we diagonalize analytically the Choi matrix for
arbitrary \pce{} maps and characterize completely positive maps by interpreting
\pce{} maps as discrete vector spaces. In section~\ref{sec:generators} we study
generators of the semigroup structure of the set of \pce{} channels, and we use
them to derive meaningful physical interpretations of \pce{} channels in
section~\ref{sec:kraus}, as well Kraus operators of the generators.
%
To finish, we conclude in section~\ref{sec:conclusions}.
% }}}


