\section{Generators} % {{{
%\todo{Francois escribe una primera version del primer punto, y luego quizá Jose
%Alfredo o Carlos la terminan}

%\janote{Mis sugerencias para esta sección son:\newline\indent (1) Es ilustrativo y bonito
%explicar a punta de las figuras PCE cómo generar todos los canales PCE 
%de 2 qubits. Sólo es necesario mostrar cómo se generan elementos representativos
%de cada clase de equivalencia. [Ya se habló de las clases de equivalencia a
%esta altura del artículo?] Esto es algo que se puede poner en un apéndice.\newline\indent
%(2) Para el caso de 1 qubit, nuestra noción de generadores es compatible 
%con los resultados de Siudzinska \cite{siudzinska2019geometry} sobre 
%el espacio en el tetraedro de los canales cuánticos que son accesibles 
%mediante generadores locales de tiempo. Quizás una referencia es 
%apropiada.\newline\indent
%(3) Hablar sobre la relación entre los generadores y filas o columnas de la $A$.
%Esto lo podemos discutir con más detalles, Carlos. 
%}

%\esqueletoja{By standard theorems of linear algebra, any subspace $W$ 
%in \eqref{eq:19}, associated with a unique PCE quantum channel, can be 
%extended to a maximal subspace of dimension $2N-1$ by adjoining appropiate
%additional basis elements.}

%\colorbox{yellow}{\textcolor{blue}{Esqueleto}}
%\textcolor{jacolor}{
%\begin{itemize}
%\item Siguiendo la discusión de los espacios vectoriales de la sección anterior, 
%introducir de forma consecuente los generadores como subespacios maximales. 
%Es decir, darle una retocada a lo que ya escribió Francois, pero que armonice 
%con lo que viene de la sección anterior.
%\item Ilustrar el algoritmo para construir todos los PCEs como 
%concatenaciones de los generadores. Comentar al final del párrafo que, para 
%1 qubit, nuestro descubrimiento de los PCEGs es compatible con los resultados 
%de Siudzinska~\cite{siudzinska2019geometry} sobre el espacio en el tetrahedro 
%en el que ella encontró que viven los canales cuánticos que son accesibles 
%mediante generadores. **en algún lado aquí se puede decir que en el apéndice 
%se muestra con los tableros como generar todo el caso de 2 qubits
%\item Ya que expusimos en el párrafo anterior la utilidad de los generadores, 
%hablar de la relación de los generadores con la $A$, como una forma de 
%obtenerlos. El enunciado principal podría ser algo como "The matrix $A$ 
%contains all the information of PCE quantum channels". Es una frase fuerte, 
%pero creo que con un poco de meditación es cierto; de la $A$ puedes sacar 
%los generadores, y con ellos todos los PCEs, se acabó. 
%\item Hablar de las simetrías de los generadores. (No sé a qué se refiere quien 
%escribió eso. Toca hablarlo con Carlos.)
%\end{itemize}}

\janote{\colorbox{yellow}{¡¡¡Léeme Carlos!!!:} Las frases nuevas que hacen
parte del esqueleto y que no estaban ayer, martes, son las que aparecen 
en las líneas 412, 422 y 425.  a todonote at line \lineref{lne:label1}} 

There exists a subset of PCE quantum channels that generate 
all PCE's, and the existence of these PCE generators may be explained with 
the vector space point of view.
\janote{Este párrafo lo voy a retocar. El enunciado de arriba es el primer enunciado
del nuevo párrafo. Ojo que hay que introducir la notación $\mcG$.}
By standard theorems of linear algebra, any subspace $W$ can be extended to a
maximal (non-trivial) subspace of dimension $2N-1$ by adjoining appropriate
additional basis elements. This can clearly be done in different ways. We
therefore arrive to the set of maximal extensions of $W$. Clearly, the
intersection of all the elements of this set reduces to $W$ itself, leading to
the result that all PCE's can be obtained as intersections of maximal PCE's. 
\janote{Referenciar \Fref{fig:examplesPCE} y apéndice con una ilustración 
del caso de 2 qubits usando los $\mcG$. Mostrar en ese apéndice todos
los generadores de 2 qubits}

PCE generators may be characterized with a unique vector index
$\vec\alpha$, and therefore, denoted as $\pceg$. \linelabel{lne:label1}

For the case of 1 qubit, the existence of PCE generators $\pceg$ is  
verified by Siudzińska~\cite{siudzinska2019geometry}.
\janote{Esto puede
ser un párrafo, o puede ser un comentario final dentro del párrafo antrerior.}

The matrix $\san$ [see \eref{eq:san}] encodes all the information
of PCE generators $\pceg$ and, therefore, of the PCE quantum channels. \janote{Aquí
tendrá que ir una prueba que aún se está cocinando}

PCE generators $\pceg$ (and so $\san$'s rows and columns) obey a symmetry rule
that defines uniquely every PCE generator.

PCE generators $\pceg$ may be classified by its local action on 
all individual qubits in the system.


-------------------------------------------------------------------
The set of all PCE quantum channels may be sorted in subsets, 
which we will call equivalence classes, whose
elements are equivalent via particle swaps and local-basis element
permutations. 
\janote{El párrafo que pretendía ser este tendrá que ir en otra parte. 
Aún hay que decidirlo con los demás.}

%Quizá en un apendice poner todos los generadores de dos qubits. 
%\janote{Poner el comentario en algún lado de que hay que hacer una
%referencia a un apéndice.}
%
%\janote{falta escribir un párrafo sobre la clasificación: (la clasificación 
%es de que cada generador está caracterizado por un $\alpha$)}
%También poner como se pueden calsificar los generadores y por aih poner la clasificacion de los 
%generadores en la figura.  o en el texto
%
%\begin{itemize}
%\item Clasificacion de los generadores, es decir que para dos qubits, lo podemos indicar con la acción sobre cada qubit individual
%\item Quizá mostrar algunas propiedades de simetrías. 
%\item Notar lo de los generadores y como sale la estructura jerárquica 
%\end{itemize}
% }}}


