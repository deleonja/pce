\section{Generators} % {{{
%\todo{Francois escribe una primera version del primer punto, y luego quizá Jose
%Alfredo o Carlos la terminan}

We now discuss the existence of a generator set for all PCE quantum channels, 
how to label each of them uniquely as $\pceg$ according to its local action 
on every qubit in the system, and, finally, we will discuss a symmetry of
both PCE quantum channel generators and $A$, the matrix that diagonalizes
the Choi matrix $\mcD_N$ of a PCE map.

There exists a subset of PCE quantum channels that generates the entire set; 
the nature of these generators may be studied, as we
shall see, with the properties of the aforementioned vector space.
By standard theorems of linear algebra, any proper subspace $W$, see \Eref{eq:19}, 
can be extended to a maximal non-trivial subspace of dimension $2N-1$ by
adjoining appropriate additional basis elements. This can be done in
different ways. We therefore arrive to the set of maximal extensions of $W$,
where every maximal subspace corresponds to a PCE quantum channel that
preserves half of the Pauli components. The intersection of all the elements of
this set reduces to $W$ itself,
and since intersection of subspaces translates to composition of PCE channels,
this implies that all PCE quantum channels can be obtained as
compositions of PCE channels corresponding to maximally non-trivial subspaces, plus
the identity map. In other words, the set of PCE quantum channels that preserve half 
of the components plus the identity map, are a generator set for 
all PCE channels.
Consider \Fref{fig:examplesPCE}; subfigures (c), (d) and (e) represent nontrivial 
PCE generators and the composition of any two of them yield the PCE channel
corresponding to (b). 


PCE generators may be characterized by its local action on every qubit 
in the system. This action can be encoded using a vector index $\vec\alpha$, 
as in \eqref{eq:pauli_strings}, hence each of the different $4^N$ 
vector indices may be uniquely related to each of the PCE generators
and thus denoted as $\mcG_{\vec\alpha}$.
% \cpnote{
The proof is simplified if one uses the Kraus representation developed in 
\sref{sec:kraus}, so we postpone the demonstration to appendix \ref{sec:2_qubits}.
\janote{referir a fig 5} For single qubits, the identity corresponds to $\mcG_0$, shown in 
\fref{fig:one:qubit:examples}(b), whereas $\mcG_3$ is shown in in 
\fref{fig:one:qubit:examples}(c).
% }
The 2-qubits PCE generator represented in subfigure (c)
of \Fref{fig:examplesPCE} acts on the first qubit (first column) as mapping its
Bloch sphere to de $x$-axis, and on the second qubit (first row) acts an identity,
hence it is labeled $\mcG_{(1,0)}$.
See \Fref{fig:appex:twoQ_PCEGs} for the notation of all 2-qubit PCE generators.

A reflection symmetry is identified for both PCE generators $\pceg$ and
the matrix $A$ that diagonalizes the Choi matrix $\mcD_N$ of a PCE map\cpnote{Creo que aun 
no tenemos el link entre el generador y la matriz A}.
Consider the map $\Sigma^{(k)}$ that reflects an index $\valpha$ with 
respect to the $k$-th axis. This map leaves all components of $\valpha$
invariant, except the $k$-th component, which is transformed according to 
\begin{equation}
% \Sigma: 
0\mapsto3,\quad3\mapsto0,\quad1\mapsto2,\quad2\mapsto1. 
\label{eq:Sigma}
\end{equation}
% and the associated 
All $N$ maps have the following properties:
\begin{enumerate}
\item $\Sigma^{(k)}(\vec\alpha)\oplus \Sigma^{(k)}(\vec\beta)=\vec\alpha\oplus \vec\beta$.
\item $\Sigma^{(k)}(\vec\alpha)\neq\vec\alpha$.
\end{enumerate}
From the first property, we now obtain
\begin{equation}
\valpha=\Sigma^{(k)}(\valpha) \oplus \Sigma^{(k)}(\vec0), 
\label{eq:Sigma}
\end{equation}
which means that if $\Sigma^{(k)}(\vec0)$ belongs
to a channel then 
$\tau_\valpha = \tau_{\Sigma^{(k)}(\vec\alpha)}$, where we used 
the fact that for channels, the non-zero elements are closed
under $\oplus$. In other words, the components of a PCE channel
are symmetric under reflection over the $k$-th axis. \cpnote{Quizá 
referencia a algunas figuras}. Now consider the case in which 
 $\Sigma^{(k)}(\vec0)$ does not belong to generator. Then 
$\tau_\valpha \ne \tau_{\Sigma^{(k)}(\vec\alpha)}$, since
the case $\tau_\valpha = \tau_{\Sigma^{(k)}(\vec\alpha)}=1$
is forbidden due to \eref{eq:Sigma} and the case 
$\tau_\valpha = \tau_{\Sigma^{(k)}(\vec\alpha)}=0$ is also forbidden 
because for generators, the codimension of the associated vector space 
is 1. 


% 
% 
% The symmetry for a PCE generator $\pceg$ is encoded in the 
% $N$-dimensional array of all $\taus$. The Cartesian grid associated with it 
% is either symmetric or anti-symmetric under reflection of the 
% indices in the $k$-th position in $\vec\alpha$. 
% Consider the map
% \begin{equation}
% % \Sigma: 
% 0\mapsto3,\quad3\mapsto0,\quad1\mapsto2,\quad2\mapsto1
% \label{eq:Sigma}
% \end{equation}
% Indeed, any generator is a subspace containing half the elements of the full vector space.  
% Map \eref{eq:Sigma} applied to the $k$-th qubit has the following properties
% \begin{enumerate}
% \item $\Sigma(\vec\alpha)\oplus \Sigma(\vec\beta)=\vec\alpha\oplus \vec\beta$.
% \item $\Sigma(\vec\alpha)\neq\vec\alpha$
% \end{enumerate}
% From the first property, we now obtain
% \begin{equation}
% \vec\alpha=\Sigma(\vec\alpha) \oplus \Sigma(\vec0)
% \end{equation}
% We now see that, depending on whether $\Sigma(\vec0)$ belongs (respectively does not belong) to the generator,
% $\vec\alpha$ belongs to the generator if and only if $\Sigma(\vec\alpha)$ belongs (respectively does not belong) to the generator.  
% 

\janote{Aquí iría la prueva de la simetría de reflexión que está en el horno aún.}
For instance, see that the 2-qubit PCE generators 
$\mcG_{(1,0)}$, $\mcG_{(3,2)}$ and $\mcG_{(2,2)}$
represented in subfigures (c), (d) and (e) of \Fref{fig:two:qubit:examples}, 
respectively, are either symmetric or anti-symmetric under reflection with 
respect to lines that divide the diagram in half vertically and horizontally. 
The same symmetry is identified for the rows and columns of $A$ if one 
labels each position in a row or column $A^{(i)}$ as $A^{(i)}_{\vec\alpha}$.
\janote{Aquí habría que agregar la prueva de simetría para la $A$.}

\begin{figure} % {{{
\centering
\includegraphics[width=\columnwidth]{oneQ_PCEGs-crop}
\caption{Connection between rows and columns of $a$ and single qubit PCE generators.
The rows or columns of $a$ and the diagrams of the generators are either 
symmetric or anti-symmetric under reflection.
\janote{Propuesta de figura para mostrar que generadores 
y la a obedecen las mismas simetrías.}
}
\label{fig:A_and_PCEGs_connection}
\end{figure} % }}}

\ddnote{Tengo entendido que esto se queria probar usando las simetrias, esto ya esta probado en la secc V}
Given that both matrix $\san$ and PCE generators $\pceg$ have the same 
symmetries, it is worth pointing out that $\san$ (and thus $a$) [see \eref{eq:san}] 
encodes all the information of PCE generators $\pceg$ and, therefore, of all PCE quantum channels. 
From the tensor power of matrix $a$ one can infer the $\{\taus\}$ of a PCE
generator $\pceg$ by taking any row or column $A^{(i)}$ and mapping 
all entries with $-1$ to $0$. In \Fref{fig:A_and_PCEGs_connection} we illustrate
this connection between $a$ and single qubit PCE generators.
In \appref{sec:2_qubits} we illustrate this program for 2-qubits PCE channels.

\cpnote{Sienot que no tiene flujo. Quizá se le pueda dar flujo usando el final 
del primer parrafo}

- - - - - - - - - - - - - - - - - - - - - - - - - - - - - - - - - - - -

The set of all PCE quantum channels may be sorted in subsets, 
which we will call equivalence classes, whose
elements are equivalent via particle swaps and local-basis element
permutations. 
\janote{El párrafo que pretendía ser este tendrá que ir en otra parte. 
Aún hay que decidirlo con los demás.}

% }}}


