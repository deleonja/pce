\section{Generators} % {{{
\todo{Francois escribe una primera version del primer punto, y luego quizá Jose
Alfredo o Carlos la terminan}

\janote{Mis suferencias para esta sección son:\newline\indent (1) Es ilustrativo y bonito
explicar a punta de las figuras PCE cómo generar todos los canales PCE 
de 2 qubits. Sólo es necesario mostrar cómo se generan elementos representativos
de cada clase de equivalencia. [Ya se habló de las clases de equivalencia a
esta altura del artículo?] Esto es algo que se puede poner en un apéndice.\newline\indent
(2) Para el caso de 1 qubit, nuestra noción de generadores es compatible 
con los resultados de Siudzinska \cite{siudzinska2019geometry} sobre 
el espacio en el tetraedro de los canales cuánticos que son accesibles 
mediante generadores locales de tiempo. Quizás una referencia es 
apropiada.\newline\indent
(3) Hablar sobre la relación entre los generadores y filas o columnas de la $A$.
Esto lo podemos discutir con más detalles, Carlos. 
}

By standard theorems of linear algebra, any subspace $W$ can be extended to a
maximal (non-trivial) subspace of dimension $2N-1$ by adjoining appropriate
additional basis elements. This can clearly be done in different ways. We
therefore arrive to the set of maximal extensions of $W$. Clearly, the
intersection of all the elements of this set reduces to $W$ itself, leading to
the result that all PCE's can be obtained as intersections of maximal PCE's. 

Quizá poner las reglas de simetría de los generadores. 

Quizá en un apendice poner todos los generadores de dos qubits. 


También poner como se pueden calsificar los generadores y por aih poner la clasificacion de los 
generadores en la figura.  o en el texto



\begin{itemize}
\item Clasificacion de los generadores, es decir que para dos qubits, lo podemos indicar con la acción sobre cada qubit individual
\item Quizá mostrar algunas propiedades de simetrías. 
\item Notar lo de los generadores y como sale la estructura jerárquica 
\end{itemize}
% }}}


