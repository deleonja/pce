\section{Generators} % {{{
%\todo{Francois escribe una primera version del primer punto, y luego quizá Jose
%Alfredo o Carlos la terminan}

%\janote{Mis sugerencias para esta sección son:\newline\indent (1) Es ilustrativo y bonito
%explicar a punta de las figuras PCE cómo generar todos los canales PCE 
%de 2 qubits. Sólo es necesario mostrar cómo se generan elementos representativos
%de cada clase de equivalencia. [Ya se habló de las clases de equivalencia a
%esta altura del artículo?] Esto es algo que se puede poner en un apéndice.\newline\indent
%(2) Para el caso de 1 qubit, nuestra noción de generadores es compatible 
%con los resultados de Siudzinska \cite{siudzinska2019geometry} sobre 
%el espacio en el tetraedro de los canales cuánticos que son accesibles 
%mediante generadores locales de tiempo. Quizás una referencia es 
%apropiada.\newline\indent
%(3) Hablar sobre la relación entre los generadores y filas o columnas de la $A$.
%Esto lo podemos discutir con más detalles, Carlos. 
%}

%\esqueletoja{By standard theorems of linear algebra, any subspace $W$ 
%in \eqref{eq:19}, associated with a unique PCE quantum channel, can be 
%extended to a maximal subspace of dimension $2N-1$ by adjoining appropiate
%additional basis elements.}

%\colorbox{yellow}{\textcolor{blue}{Esqueleto}}
%\textcolor{jacolor}{
%\begin{itemize}
%\item Siguiendo la discusión de los espacios vectoriales de la sección anterior, 
%introducir de forma consecuente los generadores como subespacios maximales. 
%Es decir, darle una retocada a lo que ya escribió Francois, pero que armonice 
%con lo que viene de la sección anterior.
%\item Ilustrar el algoritmo para construir todos los PCEs como 
%concatenaciones de los generadores. Comentar al final del párrafo que, para 
%1 qubit, nuestro descubrimiento de los PCEGs es compatible con los resultados 
%de Siudzinska~\cite{siudzinska2019geometry} sobre el espacio en el tetrahedro 
%en el que ella encontró que viven los canales cuánticos que son accesibles 
%mediante generadores. **en algún lado aquí se puede decir que en el apéndice 
%se muestra con los tableros como generar todo el caso de 2 qubits
%\item Ya que expusimos en el párrafo anterior la utilidad de los generadores, 
%hablar de la relación de los generadores con la $A$, como una forma de 
%obtenerlos. El enunciado principal podría ser algo como "The matrix $A$ 
%contains all the information of PCE quantum channels". Es una frase fuerte, 
%pero creo que con un poco de meditación es cierto; de la $A$ puedes sacar 
%los generadores, y con ellos todos los PCEs, se acabó. 
%\item Hablar de las simetrías de los generadores. (No sé a qué se refiere quien 
%escribió eso. Toca hablarlo con Carlos.)
%\end{itemize}}
%
%Un párrafo de intro a las tres ideas 
%\begin{itemize}
%\item Existen los generadores 
%\item Clasificación de los generadores 
%\item Simetrías
%\end{itemize}

We now discuss the existence of a generator set for all PCE quantum channels, 
how to label each of them uniquely as $\pceg$ according to its local action 
on every qubit in the system, and, finally, we will discuss a symmetry of
both PCE quantum channel generators and $A$, the matrix that diagonalizes
the Choi matrix $\mcD_N$ of a PCE map.

There exists a subset of PCE quantum channels that generates the entire set; 
the nature of these generators may be studied, as we
shall see, with the properties of the aforementioned vector space.
By standard theorems of linear algebra, any proper subspace $W$, see \Eref{eq:19}, 
can be extended to a maximal non-trivial subspace of dimension $2N-1$ by
adjoining appropriate additional basis elements. This can be done in
different ways. We therefore arrive to the set of maximal extensions of $W$,
where every maximal subspace corresponds to a PCE quantum channel that
preserves half of the Pauli components. The intersection of all the elements of
this set reduces to $W$ itself,
and since intersection of subspaces translates to composition of PCE channels,
this implies that all PCE quantum channels can be obtained as
compositions of PCE channels corresponding to maximally non-trivial subspaces, plus
the identity map. In other words, the set of PCE quantum channels that preserve half 
of the components plus the identity map, are a generator set for 
all PCE channels.
% and the complete vector space (corresponding to the identity channel),
% which we call generators. 
% \cpnote{Aca tenemos un semigrupo y no se si incluir a la identidad como 
% generador. De alguna manera es excepcional. Es problematico incluirla y es 
% problematico no incluirla.}
Consider \Fref{fig:examplesPCE}; subfigures (c), (d) and (e) represent nontrivial 
PCE generators and the composition of any two of them yield the PCE channel
corresponding to (b). 

%\cpnote{Arreglar}
PCE generators may be characterized by its local action on every qubit 
in the system, and thus each of them be denoted as $\mcG_{\vec\alpha}$, 
with $\vec\alpha$ each of the $4^{N}$ different vector indices. For 1-qubit
the PCE generators are the identity map plus the channels that map the Bloch 
sphere to each Cartesian axis, as their associated vector subspaces are trivial
to prove and their dimension $2(1)-1=1$ is met. 
\janote{Here goes the bla bla for the discussion of the proof for the $N$-qubit
\pce{} generators}
Therefore, we can identify the index $\alpha_k$ in $\vec\alpha$ by the local action
of the PCE generator on the $k$-th qubit. 
See, for example, that the 2-qubits PCE generator represented in subfigure (c)
of \Fref{fig:examplesPCE} acts on the first qubit (first column) as mapping its
Bloch sphere to de $x$-axis, and on the second qubit (first row) acts an identity,
hence it is labeled $\mcG_{(1,0)}$.
See \Fref{fig:appex:twoQ_PCEGs} for the notation of all 2-qubit PCE generators.

%\cpnote{Mucha lora y no se prueba nada}
A reflection symmetry is identified for both PCE generators $\pceg$ and
the matrix $A$ that diagonalizes the Choi matrix $\mcD_N$ of a PCE map.
%PCE generators $\pceg$ obey a symmetry
%that defines uniquely every PCE generator. 
The symmetry for a PCE generator $\pceg$ is encoded in the 
$N$-dimensional array of all $\taus$, the Cartesian grid associated with it 
is either symmetric or anti-symmetric under reflection of the 
indices in the $k$-th position in $\vec\alpha$ as
\begin{align}
\alpha_{0}\mapsto\alpha_3,\quad\alpha_{3}\mapsto\alpha_0,
\quad\alpha_1\mapsto\alpha_2, \quad\alpha_2\mapsto\alpha_1.
\end{align}
\janote{Aquí iría la prueva de la simetría de reflexión que está en el horno aún.}
For instance, see that the 2-qubit PCE generators 
$\mcG_{(1,0)}$, $\mcG_{(3,2)}$ and $\mcG_{(2,2)}$
represented in subfigures (c), (d) and (e) of \Fref{fig:two:qubit:examples}, 
respectively, are either symmetric or anti-symmetric under reflection with 
respect to lines that divide the diagram in half vertically and horizontally. 
The same symmetry is identified for the rows and columns of $A$ if one 
labels each position in a row or column $A^{(i)}$ as $A^{(i)}_{\vec\alpha}$.
\janote{Aquí habría que agregar la prueva de simetría para la $A$.}

\begin{figure} % {{{
\centering
\includegraphics[width=\columnwidth]{oneQ_PCEGs-crop}
\caption{Connection between rows and columns of $a$ and 1-qubit PCE generators.
The rows or columns of $a$ and the diagrams of the generators are either 
symmetric or anti-symmetric under reflection.
\janote{Propuesta de figura para mostrar que generadores 
y la a obedecen las mismas simetrías.}
}
\label{fig:A_and_PCEGs_connection}
\end{figure} % }}}

%\cpnote{Mucha lora y no se prueba nada, again}
Given that both matrix $\san$ and PCE generators $\pceg$ have the same 
symmetries, it is worth pointing out that $\san$ (and thus $a$) [see \eref{eq:san}] 
encodes all the information of PCE generators $\pceg$ and, therefore, of all PCE quantum channels. 
From the tensor power of matrix $a$ one can infer the $\{\taus\}$ of a PCE
generator $\pceg$ by taking any row or column $A^{(i)}$ and mapping 
all entries with $-1$ to $0$. In \Fref{fig:A_and_PCEGs_connection} we illustrate
this connection between $a$ and 1-qubit PCE generators.
In \appref{sec:2_qubits} we illustrate this program for 2-qubits PCE channels.

\cpnote{Sienot que no tiene flujo. Quizá se le pueda dar flujo usando el final 
del primer parrafo}

- - - - - - - - - - - - - - - - - - - - - - - - - - - - - - - - - - - -

The set of all PCE quantum channels may be sorted in subsets, 
which we will call equivalence classes, whose
elements are equivalent via particle swaps and local-basis element
permutations. 
\janote{El párrafo que pretendía ser este tendrá que ir en otra parte. 
Aún hay que decidirlo con los demás.}

%Quizá en un apendice poner todos los generadores de dos qubits. 
%\janote{Poner el comentario en algún lado de que hay que hacer una
%referencia a un apéndice.}
%
%\janote{falta escribir un párrafo sobre la clasificación: (la clasificación 
%es de que cada generador está caracterizado por un $\alpha$)}
%También poner como se pueden calsificar los generadores y por aih poner la clasificacion de los 
%generadores en la figura.  o en el texto
%
%\begin{itemize}
%\item Clasificacion de los generadores, es decir que para dos qubits, lo podemos indicar con la acción sobre cada qubit individual
%\item Quizá mostrar algunas propiedades de simetrías. 
%\item Notar lo de los generadores y como sale la estructura jerárquica 
%\end{itemize}
% }}}


