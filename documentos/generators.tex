\section{Generators} % {{{
\todo{Francois escribe una primera version del primer punto, y luego quizá Jose
Alfredo o Carlos la terminan}

\janote{Mis suferencias para esta sección son:\newline\indent (1) Es ilustrativo y bonito
explicar a punta de las figuras PCE cómo generar todos los canales PCE 
de 2 qubits. Sólo es necesario mostrar cómo se generan elementos representativos
de cada clase de equivalencia. [Ya se habló de las clases de equivalencia a
esta altura del artículo?] Esto es algo que se puede poner en un apéndice.\newline\indent
(2) Para el caso de 1 qubit, nuestra noción de generadores es compatible 
con los resultados de Siudzinska \cite{siudzinska2019geometry} sobre 
el espacio en el tetraedro de los canales cuánticos que son accesibles 
mediante generadores locales de tiempo. Quizás una referencia es 
apropiada.\newline\indent
(3) Hablar sobre la relación entre los generadores y filas o columnas de la $A$.
Esto lo podemos discutir con más detalles, Carlos. 
}

\colorbox{yellow}{\textcolor{blue}{Esqueleto}}
\textcolor{jacolor}{
\begin{itemize}
\item Siguiendo la discusión de los espacios vectoriales de la sección anterior, 
introducir de forma consecuente los generadores como subespacios maximales. 
Es decir, darle una retocada a lo que ya escribió Francois, pero que armonice 
con lo que viene de la sección anterior.
\item Ilustrar el algoritmo para construir todos los PCEs como 
concatenaciones de los generadores. Comentar al final del párrafo que, para 
1 qubit, nuestro descubrimiento de los PCEGs es compatible con los resultados 
de Siudzinska~\cite{siudzinska2019geometry} sobre el espacio en el tetrahedro 
en el que ella encontró que viven los canales cuánticos que son accesibles 
mediante generadores. **en algún lado aquí se puede decir que en el apéndice 
se muestra con los tableros como generar todo el caso de 2 qubits
\item Ya que expusimos en el párrafo anterior la utilidad de los generadores, 
hablar de la relación de los generadores con la $A$, como una forma de 
obtenerlos. El enunciado principal podría ser algo como "The matrix $A$ 
contains all the information of PCE quantum channels". Es una frase fuerte, 
pero creo que con un poco de meditación es cierto; de la $A$ puedes sacar 
los generadores, y con ellos todos los PCEs, se acabó. 
\item Hablar de las simetrías de los generadores. (No sé a qué se refiere quien 
escribió eso. Toca hablarlo con Carlos.)
\end{itemize}}

By standard theorems of linear algebra, any subspace $W$ can be extended to a
maximal (non-trivial) subspace of dimension $2N-1$ by adjoining appropriate
additional basis elements. This can clearly be done in different ways. We
therefore arrive to the set of maximal extensions of $W$. Clearly, the
intersection of all the elements of this set reduces to $W$ itself, leading to
the result that all PCE's can be obtained as intersections of maximal PCE's. 

Quizá poner las reglas de simetría de los generadores. 

Quizá en un apendice poner todos los generadores de dos qubits. 


También poner como se pueden calsificar los generadores y por aih poner la clasificacion de los 
generadores en la figura.  o en el texto



\begin{itemize}
\item Clasificacion de los generadores, es decir que para dos qubits, lo podemos indicar con la acción sobre cada qubit individual
\item Quizá mostrar algunas propiedades de simetrías. 
\item Notar lo de los generadores y como sale la estructura jerárquica 
\end{itemize}
% }}}


