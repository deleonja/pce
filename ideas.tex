\documentclass[11pt,letterpaper]{article}
\usepackage[utf8]{inputenc}
\usepackage[spanish]{babel}
\usepackage{amsmath}
\usepackage{amsfonts}
\usepackage{amssymb}
\usepackage{graphicx}
\usepackage[dvispnames]{xcolor}
\usepackage{bbold}
\usepackage[left=2cm,right=2cm,top=2cm,bottom=2cm]{geometry}

\newcommand{\mcG}{\mathcal{G}}
\newcommand{\one}{\mathbb{1}}

\title{Ideas}
\begin{document}
\maketitle

Un documento con algunas ideas no concretadas. Falta jugar para ver qué sale.

\section{FM (y CG?) + PCEs}
\begin{itemize}
\item Proceso a considerar
	\begin{enumerate}
		\item Evolución bajo Hamiltoniano $H$
		\item Abrir el sistema para medirlo y ocurre decoherencia descrita por PCEs
		\item Medida FM
	\end{enumerate}
\end{itemize}

Para 2 qubits:

Decoherencia: 
\begin{align}\label{eq:decoherence}
D(\rho) = p'\rho + (1-p')\mathcal G_{\vec \alpha}(\rho),
\end{align}
con $p'$ la probabilidad de que sobre el estado $\rho$ no ocurra ningún proceso
de decoherencia, y $\mathcal G_{\vec \alpha}$ un generador PCE {\color{red} (obviamente este no es un proceso de decoherencia general modelado por PCEs, pero es una primera aproximación)}. 

FM: 
\begin{align}\label{eq:FM}
\mathcal F(\rho) = p\rho+ (1 - p)S(\rho),
\end{align}
con $p$ la probabilidad de medir al estado real y $S$ el operador de swap. 

Entonces si abrimos el sistema luego de que ha evolucionado bajo un Hamiltoniano $H$, ocurre un proceso de decohrencia \eqref{eq:decoherence} y luego se hace una FM, el estado final sería
\begin{align}\label{eq:all}
\mathcal F \Big[ D(\rho) \Big] = p \rho + pp'\Big\{ S\Big(\mcG_{\vec\alpha}(\rho)\Big) - S(\rho) \Big\} + (1-p)S\Big(\mcG_{\vec\alpha}(\rho)\Big),
\end{align}
naturalmente, si ningún proceso de decoherencia ocurre \textit{i.e.} $\mcG_{\vec\alpha} = \one$, entonces \eqref{eq:all} se reduce a \eqref{eq:FM}.
\begin{itemize}
\item ¿Puede ser positivo el efecto de la decoherencia en algún sentido? Tomografía, por ejemplo. Suponiendo que queremos saber cuál es el estado después de la evolución del sistema bajo $H$
\item Caso límite: $p=0$ \textit{i.e.} el detector siempre confunde a las partículas 1 y 2. En este caso \eqref{eq:all} se reduce a 
\begin{align}\label{eq:limit:case:1}
\mathcal F \Big[ D(\rho) \Big] = S\Big(\mcG_{\vec\alpha}(\rho)\Big).
\end{align} 
\end{itemize}

\end{document}